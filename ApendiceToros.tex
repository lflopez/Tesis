En esta pequeña sección vamos a estudiar transformaciones entre toros complejos y clasificaremos los automorfismos de un toro que fijan un punto (isomorfismo de grupos).

\begin{defn}
\label{latiz}
Sean $\omega,\tilde{\omega}\in\C$ dos números complejos que son linealmente independientes como vectores en $\mathbb{R}^{2}$. Al conjunto $L=\omega\mathbb{Z}\oplus\tilde{\omega}\mathbb{Z}$ lo llamaremos la \emph{retícula} generada por $\omega$ y $\tilde{\omega}$
\end{defn}

Observemos que por definición una retícula $L$ es un subgrupo aditivo de $\C$, como $\C$ es conmutativo podemos considerar el grupo cociente $X_{L}:=\C/L$. Este conjunto es bien sabido que es homeomorfo a un toro y con la proyección canónica $\Pi\colon\C\rightarrow L$ podemos hacer de $X_{L}$ una superficie de Riemann.\\

Sean $X,Y$ dos toros con retículas asociadas $L$ y $M$ respectivamente. Si $F\colon X\rightarrow Y$ es una función holomorfa, entonces, por la fórmula de Riemann-Hurwitz, $F$ no tiene puntos de ramifiación y por lo tanto es una aplicación cubriente. Como la proyección canónica $\Pi_{L}\colon\C\rightarrow X$ es la  cubriente universal de $X$ tenemos que $F\circ\Pi\colon\C\rightarrow Y$ es una aplicación cubriente y como $\Pi_{L}\colon\C\rightarrow Y$ es la cubriente universal de $Y$ debe de existir una función holomorfa $G$ que hace conmutar el siguiente diagrama.\\

%Diagrama  

A continuación obtendremos una expresión para la función $G$. Para cualquier retícula $M$ y cualquier número complejo $a\in\C$, la traslación $z\mapsto a+z$ respeta las clases de equivalencia de $\C/M$ y por lo tanto la traslación baja a una traslación en el cociente $\C/M$. Así, con una traslación adecuada, siempre podemos suponer que $G(0)=0$. Si $l\in L$, por la conmutatividad del diagrama tenemos que $G(z-l)=G(z)$ (mod$M$) y así, para $l$ fijo la función $G(z-l)-G(z)\colon\C\rightarrow M\subset\C$ toma valores en $M$. Como $M$ es discreto y $G$ es continua tenemos que $G(z-l)-G(z)$ es constante y por lo tanto $G(z-l)'=G(z)'$. Esto quiere decir que todos los valores de $G'$ son tomados en una región contenida en el paralelogramo formado por dos generadores de $M$ y por lo tanto $G'$ es acotada y por el teorema de Liuoville, $G'$ es constante. Así, $G(z)=\gamma z$ para algún $\gamma\in\C$. Como $G(z-l)=G(z)$ (mod$M$) tenemos que $G(l)=G(0)=0$ (mod$M$) y entonces $\gamma L\subset M$. Si $F$ es un isomorfismo tenemos que $\gamma^{-1}M\subset L$ y entonces $M\subset\gamma L$, por lo tanto concluímos que $\gamma L = M$.\\

Ahora vamos a clasificar los automorfismos de un toro en sí mismo. Sea $F\colon X\rightarrow X$ un biholomorfismo que fija al origen (automorfsimo de grupos), en la cubriente universal esta transformación adquiere la expresión $G(z)=\gamma z$ y además se cumple que $\gamma L=L$. Afirmamos que $\gamma$ tiene que ser una raíz de la unidad. En efecto, si $|\gamma|<1$ y $l\in L$ es un elemento de la retícula de longitud mínima, $|\gamma l|<|\gamma|$, y por lo tanto $\gamma l\notin L$, esto claramente no pude suceder. Si $|\gamma|>1$ podemos hacer un argumento similar. Si el argumento de $\gamma$ no es conmesurable con $\pi$, la órbita de un punto en la circunferencia unitaria se hace densa en ella y en consecuencia no puede ocurrir que $\gamma L = L$. Por lo tanto, $\gamma$ es una raíz de la unidad.

Si $l\in L\setminus \{ 0\}$ y $\gamma\in\C\setminus\mathbb{R}$, entonces $l$ y $\gamma l$ generan a $L$ y en consecuencia $\gamma^{2} l$ debe de ser una combinación entera de ellos dos.
\begin{equation*}
\gamma^{2} l =m\gamma l +nl,
\end{equation*}
\noindent entonces,
\begin{equation*}
\gamma^{2}-m\gamma-nl=0.
\end{equation*}
\noindent Por lo tanto,
\begin{equation}
\label{TorosPosibles}
\gamma=\frac{m\pm \sqrt{m^{2}-4n}}{2}.
\end{equation}

\noindent Como $\gamma$ es una raíz de la unidad los únicos valores posibles para $m$ son $-2,-1,-0,1$ y $2$. Esto nos dice que $\gamma$ puede ser una raíz cuadrada, cuarta ó sexta de la unidad. Ahora que sabemos que valores puede tomar $\gamma$, la condición $\gamma L =L$ nos obligará a poner algunas restricciones a $L$. Si $\gamma$ es una raíz cuadrada de la unidad no hay ningun problema pues $-L=L$. Cuando $\gamma$ es una raiz cuarta de la unidad los generadores de $L$ son ortogonales (y en consecuencia $L\simeq \mathbb{Z}\oplus i\mathbb{Z}$), si $\gamma$ es una raíz sexta entonces los generadores forman un ángulo de $\tfrac{\pi}{3}$ ($L\simeq \mathbb{Z}\oplus e^{\tfrac{\pi i}{3}}\mathbb{Z}$).

Denotemos por $j=e^{\tfrac{2\pi i}{3}}$, $L=\mathbb{Z}\oplus e^{\tfrac{\pi i}{3}}\mathbb{Z}$
y $\Gamma=\mathbb{Z}\oplus j\mathbb{Z}$. Observemos que $z\mapsto jz$ lleva a la retícula $L$ en la retícula $\Gamma$ y en consecuencia $\C/L\simeq\C/\Gamma$. El párrafo anterior nos dice que si un toro $X$ tiene un automorfismo de orden tres, entonces $X\simeq\C/\Gamma$ ($j$ es una raíz sexta de la unidad y tiene orden tres).
