En esta sección veremos que existe un subconjunto denso $E$ de parámetros $\alpha\in\overline{\C}$ tal que si $\alpha\in E$, entonces la foliación $\Foli_{\alpha}$ tiene una primera integral racional. En caso de que dicha primera integral exista sabemos que el grupo de holonomía global $G_{\alpha}$ de la foliación $\tilde{\Foli}_{\alpha}=\Pi^{*}(\Foli_{\alpha})$ es finito (proposición \ref{Prop:EquivalenciasIntegrabilidad} y corolario \ref{Coro:Equivalencias}). Vamos a acotar por abajo el grado de la primera integral en términos de la cardinalidad del grupo de holonomía global y mostraremos que para una natural $k_{0}$ fijo, los parámetros $\alpha\in E$ tales que la primera integral de la foliación $\Foli_{\alpha}$ es de grado menor que $k_{0}$, forman un subonjunto finito.\\

Por la proposición \ref{Prop:EquivalenciasIntegrabilidad} sabemos que la foliación $\Foli_{\alpha}$ tiene primera integral racional si y sólo si existe $n\in\mathbb{N}$ tal que $nA(\alpha)\in\mathbb{Z}\oplus j\mathbb{Z}=\Gamma$, donde $g_{\alpha}(z)=jz+A(\alpha)$ es uno de los dos generadores del grupo de holonomía global $G_{\alpha}$. La condición $nA(\alpha)\in\mathbb{Z}\oplus j\mathbb{Z}$ es equivalente a que $A(\alpha)\in\mathbb{Q}\oplus j\mathbb{Q}$. El conjunto $A(\alpha)\in\mathbb{Q}\oplus j\mathbb{Q}$ es un subconjunto denso del toro $\C/\Gamma$. Veremos que la función $A\colon\C\rightarrow\C/\Gamma$ es holomorfa, no constante y por lo tanto localmente abierta. Esto implica que la preimagen de $\mathbb{Q}\oplus j\mathbb{Q}$, es densa en $\C$. Así pues, vamos a estudiar algunas propiedades de la función $A\colon\C\rightarrow \C/\Gamma$.\\

Primero veamos que $A\colon\C\rightarrow\C/\Gamma$ es holomorfa. La transformación $g_{\alpha}=jz+A(\alpha)$ está generada por un lazo $\gamma\in\pi_{1}(\C\setminus\{0,1\},a)$. Si fijamos este lazo obtenemos una función $\Phi_{\gamma}\colon\C\times T_{a}\rightarrow T_{a}$, donde $T_{a}=h^{-1}(a)\simeq\C/\Gamma$ es una transversal a todas las foliaciones $\tilde{\Foli}_{\alpha},\, \alpha\in\C$. Esta función está definida como $\Phi_{\gamma}(\alpha,p)=g_{\alpha}$, esto quiere decir que para $\alpha\in\C$ fijo, $\Phi_{\gamma}(\alpha,z)=jz+A(\alpha)$ es uno de los dos generadores del grupo de holonomía global $G_{\alpha}$. Como las soluciones de la ecuación diferencial que genera a cada una de estas foliaciones en una vecindad de cada punto dependen de manera holomorfa de las condiciones iniciales y del parámetro $\alpha$ (teorema \ref{Teo:ExistenciaUnicidad}), tenemos que $A(\alpha)$ es holomorfa.\\

Para ver que $A(\alpha)$ es no constante veremos que los grupos de holonomía de las foliaciones $\tilde{\Foli}_{0}$ y $\tilde{\Foli}_{1}$ tienen distinto número de elementos y por lo tanto $A(0)\neq A(1)$.\\

Por la proposición \ref{Prop:2} tenemos que $$H_{1}=\frac{(x-j^{2})(y-j)(y-j^{2}x)}{(x-j)(y-j^{2})(y-jx)}=\frac{p_{1}}{q_{1}},$$ es una primera integral racional de $\Foli_{1}$. Por el teorema de Bézout tenemos que $\{p_{1}=0\}$ interseca a $\{q_{1}=0\}$ en nueve puntos y cada uno de estos puntos está en todas las curvas de nivel de $H_{1}$. Con un razonamiento similar, la curva de nivel $H_{\infty}^{-1}(a)=\Pi(T_{a})$ siempre pasa por los nueve puntos de intersección de $\{p_{\infty}=y^{3}-1=0\}$ y $\{q_{\infty}=x^{3}-1=0\}$. Con esto no es difícil convencerse (ver figura...) que $H^{-1}_{1}(c)$ interseca a $H^{-1}_{\infty}(a)$ en los seis puntos $(1,j),(1,j^{2}),(j,1),(j,j),(j^{2},1)$ y $(j^{2},j^{2})$. Así, de los nueve puntos de intersección de $H^{-1}_{1}(c)$ y $H^{-1}_{\infty}(a)$, sólo tres de ellos no son alguno de los doce puntos que explota la transformación $\Pi\colon M\rightarrow\CP$. Las intersecciones en cada uno de estos seis puntos es transversal y entonces, al explotar, estas intersecciones desaprecen y por lo tanto el transformado estricto de $H^{-1}_{1}(c)$ bajo $\Pi$ (que es una hoja de la foliación $\tilde{\Foli_{1}}$) interseca a $T_{a}=\Pi^{-1}(H^{-1}_{\infty}(a))$ en tres puntos. Esto implica que el grupo de holonomía global $G_{1}$ sólo tiene tres elementos y así $f_{1}(z)=g_{1}(z)=jz$. Por lo tanto, $A(0)=0$.\\

A continuación veremos que $A(0)\neq 0$ (mod $\Gamma$). Notemos primero que $$H_{0}(x,y)=\frac{x^{3}(y^{3}-1)}{y^{3}(x^{3}-1)}=\frac{p_{0}}{q_{0}},$$ es una primera integral racional de $\Foli_{0}$. En efecto, $$q_{0}^{2}dH_{0}=q_{0}dp_{0}-p_{0}dq_{0}=3x^{2}y^{2}[x(x^{3}-1)\, dy-y(y^{3}-1)\, dx].$$

La recta $\{x=0\}$ que es parte de la curva de nivel cero de $H_{0}$ corta a $H_{\infty}(a)$ en tres puntos $(0,y_{k})$, donde $y_{1},y_{2},y_{3}$ son las raíces de $a=y^{3}-1$. $H_{0}^{-1}(0)$ tiene multiplicidad tres a lo largo de la recta $\{x=0\}$ así que para $b$ cercana a $0$, $H_{0}^{-1}(b)$ interseca a $H_{\infty}^{-1}(a)$ en nueve puntos  pues cada uno de los puntos de intersección $(0,y_{k})$, al variar $b$, se separa en tres puntos cercanos a $(0,y_{k})$ y además, para $b$ suficientemente pequeña ninguno de estos nueve puntos coincide con los puntos que explota $\Pi$. Por lo tanto, al explotar, el transformado estricto de $H_{0}^{-1}(b)$ interseca a la tansversal $T_{a}=h^{-1}(a)$ en al menos nueve puntos y así, el grupo $G_{0}$ tiene al menos nueve elementos. Esto quiere decir que $A(0)\neq 0$. Ahora podemos probar el siguiente teorema.

\begin{Teorema}
\label{Teo:DensidadPrimerasIntegrales}
Para la familia $\Foli_{\alpha}$ de foliaciones de $\CP$ existe un subconjunto denso de parámetros $E\subset\overline{\C}$ tal que si $\alpha\in E$, la foliación $\Foli_{\alpha}$ tiene una primera integral racional.
\end{Teorema}
\begin{proof}
Como la función $A\colon\C\rightarrow\C/\Gamma$ es holomorfa y no constante, $E=A^{-1}(\mathbb{Q}\oplus j\mathbb{Q})$ es un subconjunto denso de $\C$. Si $\alpha\in\ E$ tenemos que $A(\alpha)\in\mathbb{Q}\oplus j\mathbb{Q}$ y por lo tanto existe $n\in\mathbb{N}$ tal que $nA(\alpha)\in\Gamma=\mathbb{Z}\oplus j\mathbb{Z}$. De acuerdo al teorema \ref{Prop:EquivalenciasIntegrabilidad} la foliación $\Foli_{\alpha}$ tiene una primera integral racional.
\end{proof}

Si $\alpha\in E\subset\overline{\C}$ entonces la foliación $\Foli_{\alpha}$ tiene una primera integral racional de grado $d(\alpha)$ y para el grupo de holonomía global $G_{\alpha}$ de la foliación $\tilde{\Foli}_{\alpha}$ existe un $k=k(\alpha)\in\mathbb{N}$ tal que  $G_{\alpha}$ tiene $3k(\alpha)^{2}$ elementos (corolario \ref{Coro:Equivalencias} y proposición \ref{Prop:EquivalenciasIntegrabilidad}).

\begin{Lema}
\label{Lema:Cota}
Si el grupo de holonomía global $G_{\alpha}$ tiene $k(\alpha)^{2}$ elementos, entonces el grado $d(\alpha)$ de la primera integral de la foliación $\Foli_{\alpha}$ cumple $k(\alpha)^{2}\leq d(\alpha)$.
\end{Lema} 
\begin{proof}
Sea $L$ una hoja de la foliación $\Foli_{\alpha}$ y $\tilde{L}$ su transformado estricto bajo $\Pi$, por el teorema de Bézout tenemos que,
$$3d(\alpha)=\#\{L\cap H_{\infty}^{-1}(a)\}\geq\#\{\tilde{L}\cap h_{\infty}^{-1}(a)\}=\#G_{\alpha}=3k(\alpha)^{2}.$$
\end{proof}

\begin{Teorema}
\label{Teo:GradosGrandes}
Dado un natural $n_{0}\in\mathbb{N}$ el subconjunto $E_{n_{0}}:=\{\alpha\in E;\, k(\alpha)\leq n_{0}\, \}\cup\{\infty \, \}$ es finito. Es decir, las foliaciones $\Foli_{\alpha}$ de $\CP$ con primera integral racional de grado $d(\alpha)\leq n_{0}$ son un número finito. 
\end{Teorema}
\begin{proof}
Si el conjunto $E_{n_{0}}$ fuera infinito tendríamos que para algún número $r\in\{\, 1,\ldots,n_{0} \, \}$, el conjunto $E_{r}:=\{\, \alpha\in E_{n_{0}};\, k(\alpha)=p \, \}\cup\{\, \infty\, \}$ es infinito. Del corolario \ref{Coro:Equivalencias} tenemos que para todo $\alpha,\beta\in E_{r}$, $A(\alpha)=A(\beta)$ (mod $\Gamma$) y entonces el subconjunto $E_{r}$ tiene una sucesión convergente $\{\, a_{i} \, \}_{i\in\mathbb{N}}$ tal que $A(a_{i})=A(\alpha)=A(\beta)$ para toda $i\in\mathbb{N}$ esto implica que $A(z)$ es constante lo cual es una contradicción.
\end{proof}

\section{El género de las hojas de $\Fol{\alpha}$.}

Cuando una curva algebraica es suave hay una conocida fórmula que relaciona el grado de la curva algebraica con su género. En el caso de que la curva algebraica $C$ sea singular, debemos desingularizarla para así obtener una superficie de Riemann compacta $S$ y una aplicación holomorfa $\varphi\colon S\rightarrow C$ que es un biholomorfimso fuera de $S\setminus\varphi^{-1}(sing(C))$, donde $sing(C)$ es el conjunto de puntos singulares de la curva $C$ y luego calcular el género de la superficie $S$.

En nuestro caso, si $\alpha\in\{\, 1,j,j^{2},\infty \, \}$ las proposiciones \ref{Prop:3} y \ref{Prop:FibraDelHaz} nos dicen que las hojas irreducibles que no están contenidas en las nueve líneas que deja invariante la foliación $\Fol{\alpha}$ son toros biholomorfos a $\C/\mathbb{Z}\oplus j\mathbb{Z}=\C/\Gamma$. Cuando $\alpha\notin\{\, 1,j,j^{2},\infty \, \}$ la proposición \ref{Prop:2} nos asegura que al explotar mediante $\Pi\colon M\rightarrow \CP$ a los doce puntos radiales de la foliación $\Fol{\alpha}$, cualquier hoja $L$ de la foliación que no pase por ninguno de los nueve puntos silla $q_{i}(\alpha)$ se transforma en una superficie de Riemann $\tilde{L}$. En el caso particular en que la foliación $\Fol{\alpha}$ tiene primera integral racional ($\alpha\in E$),  vamos a construir una aplicación cubriente $\pi\colon \tilde{L}\rightarrow \C/\Gamma$ donde $\tilde{L}$ es una hoja ``genérica'' de la foliación $\tilde{\Fol{\alpha}}$. Esto implica que la hoja $L=\Pi({\tilde{L}})$ de la foliación $\Fol{\alpha},\, \alpha\in E$ tiene género uno pues por la proposición \ref{Prop:EquivalenciasIntegrabilidad} la hoja $\tilde{L}$ es una superficie de Riemann compacta.\\

Para poder definir la aplicación cubriente vamos necesitar algunas definiciones. 
\begin{defn}
\label{Def:PuntoGenerico}
Sea $G=\langle jx,jx+A\rangle$ un subrgupo de biholomorfismos de $\C/\Gamma$ generado por dos elementos, decimos que $b\in\C/\Gamma$ es \emph{genérico} para el grupo $G$ si el único elemento $f\in G$ que fija a $b$ es la identidad. Es decir, $b\in\C/\Gamma$ es genérico si su estabilizador, $G_{b}$ es trivial.
\end{defn}

En el caso particular en que $\alpha\in E$, el grupo $G_{\alpha}$ que actúa en $\C/\Gamma$ es finito, así, hay un número finito de estabilizadores y cada estabilizador fija un número finito de puntos. Por lo tanto, el conjunto de puntos de $\C/\Gamma$ que no son genéricos para $G_{\alpha}$ es un conjunto finito y así, el conjunto de puntos genéricos es denso y no numerable.

\begin{defn}
\label{Def:HojaGenerica}
 Diremos que una hoja $\tilde{L}$ de la foliación $\tilde{\Fol{\alpha}}$ es \emph{genérica} si $\tilde{L}\cap T_{a}$ tiene un punto genérico, donde $T_{a}=h^{-1}(a)\simeq\C/\Gamma$ es una transversal a todas las hojas de la foliación $\tilde{\Fol{\alpha}}$.
\end{defn}

Notemos que la condición de ser una hoja genérica no depende del punto $b\in\tilde{L}\cap T_{a}$. En efecto, si denotamos por $L$ a $\tilde{L}\setminus h^{-1}(0,1,\infty)$, la restricción de $h$ a $L$ es una aplicación cubriente y así, $L\cap T_{a}=\tilde{L}\cap T_{a}$ es la fibra $h^{-1}(a)$ de esta aplicación cubriente. Como la acción de monodromía de toda aplicación cubriente es transitiva, todos los puntos tienen estabilizadores conjugados. Así, si un punto $b\in L\cap T_{a}$ es genérico, todos los puntos en $\tilde{L}\cap T_{a}$ son genéricos.

Como en el caso en que $\alpha\in E$ el grupo de holonomía global $G_{\alpha}$ es finito, los puntos $b\in T_{a}$ son genéricos son densos y no numerables. Por lo tanto, si $\alpha\in E$ el conjunto de hojas genéricas de la foliación $\Fol{\alpha}$ es denso y no numerable.\\

Una hoja $\tilde{L}_{1}$ de la foliación $\tilde{\Fol{1}}$ es biholomorfa a $\C/\Gamma$. Dado $\alpha\in E$ y una hoja genérica $\tilde{L}$ de la foliación $\Fol{\alpha}$, vamos a construir una aplicación cubriente $\pi\colon\tilde{L}\rightarrow\tilde{L}_{1}$. Denotemos por $h_{0}$ y $h_{1}$ a las restricciones de $h=H\circ\Pi$ a las hojas $\tilde{L}$ y $\tilde{L}_{1}$ respectivamente y denotemos por $L$ y $L_{1}$ a $\tilde{L}\setminus\{\, 0,1,\infty \, \}$ y $\tilde{L}_{1}\setminus\{\, 0,1,\infty \, \}$.

Escojamos un punto $b\in L\cap T_{a}$ y un punto $b_{1}\in L_{1}\cap T_{a}$ y definamos $\pi(b)=b_{0}$, vamos a definir $\pi$ de tal manera que $h_{0}=h_{1}\circ\pi$. Sean $x\in\tilde{L}$ y $\tilde{\gamma}\colon [0,1]\rightarrow L$ un camino en $L$ que una $b$ con $x$. Al aplicar $h_{0}$, $\tilde{\gamma}$ se convierte en una curva $\gamma\colon [0,1]\rightarrow\C\setminus\{\, 0,1\, \}$ que podemos levantar a una curva $\hat{\gamma}\colon [0,1]\rightarrow  L_{1}$ mediante la aplicación cubriente $h_{1}$. Definimos $\pi(x)=\hat{\gamma}(1)$.

Veamos que $\pi(x)$ no depende del camino $\tilde{\gamma}$ que une a $b$ con $x$. Como es usual, basta ver que si $\tilde{\gamma}$ es un camino cerrado basado en $b$, $\hat{\gamma}(1)=b_{1}$. La transformación de holonomía $k\colon\C/\Gamma\rightarrow\C/\Gamma$ del grupo $G_{\alpha}$ asociada a la curva $\gamma=h_{0}\circ\tilde{\gamma}$ deja fijo al punto $b$, y como la hoja $\tilde{L}$ es genérica, la transformación $k$ tiene que ser la identidad. Así, si escribimos a $\gamma$ como producto de los dos generadores del grupo fundamental de $\C\setminus\{\, 0,1\, \}$, $\gamma=\gamma_{1}^{m_{1}}\circ\gamma_{2}^{n_{1}}\circ\cdots\circ\gamma_{1}^{m_{r}}\circ\gamma_{2}^{m_{r}}$ podemos escribir a $k$ como un producto de los generadores $f_{\alpha}$ y $g_{\alpha}$ del grupo de holonomía global $G_{\alpha}$ de la foliación $\tilde{\Fol{\alpha}}$, obteniendo:

$$k(z)=f_{\alpha}^{m_{1}}\circ g_{\alpha}^{n_{1}}\circ\cdots\circ f_{\alpha}^{m_{r}}\circ g_{\alpha}^{n_{r}}=cz+d=z,$$

\noindent donde $c=j^{m_{1}+\cdots+m_{r}}$. Como $c=1$ tenemos que $3$ divide a $m_{1}+\cdots+m_{r}$. En la sección anterior vimos que en el grupo $G_{1}$, los dos generadores $f_{1}$ y $g_{1}$ coinciden ($f_{1}(z)=jz$). Así, la transformación de holonomía del grupo $G_{1}$ asociada al lazo $\gamma$ es $K(z)=j^{m_{1}+\cdots+m_{r}}=j^{3p}=z$ y por lo tanto $\hat{\gamma}(b)=b_{1}$.

Hemos definido $\pi\colon L\rightarrow L_{1}$ de tal forma que $h_{0}=h_{1}\circ\pi$, y como $h_{1}$ es invertible de manera local, localmente tenemos que $\pi=h_{1}^{-1}\circ h_{0}$. Por lo tanto, es holomorfa y más aún, es un homeomorfismo local.\\

Ahora vamos a extender a $\pi$ al conjunto $\tilde{L}\cap h^{-1}(0,1,\infty)$. El conjunto $h^{-1}(0,1,\infty)$ está formado por tres divisores $D_{1},D_{2}$ y $D_{3}$ que se obtienen de explotar los puntos, en coordenadas homogéneas, $[1:0:0],[0:1:0]$ y $[0:0:1]$ y por el transformado estricto $\tilde{\mathit{l}_{k}}$ de las nueve rectas $\mathit{l}_{k}$ que deja invariante toda foliación $\Fol{\alpha}$ de la familia. Como la foliación $\tilde{\Fol{\alpha}}$ deja invariante al transformado estricto de estas nueve rectas y la hoja $\tilde{L}$ no pasa por ningún punto silla $\tilde{q_{k}(\alpha)}$ (estas hojas no son genéricas), entonces las intersecciones tanto de $\tilde{L}$ como de $\tilde{L}_{1}$ con el conjunto $h^{-1}(0,1,\infty)$ están contenidas en el conjunto $D_{1}\cup D_{2}\cup D_{3}$. Vamos a definir $\pi$ para los puntos en $\tilde{L}\cap D_{1}$, donde $D_{1}$ es el divisor obtenido al explotar el punto $(0,0)$, para definirla en los otros divisores podemos usar el corolario \ref{Coro2Prop1} y la definición de $\pi$ para los puntos en $\tilde{L}\cap D_{1}$.\\

La hoja $\tilde{L}_{1}$ queda descrita por $\tilde{h}_{1}^{-1}(e)=(H_{1}\circ\Pi)^{-1}(e)$, $e\in\C\setminus\{\, 0,1 \, \}$, donde $H_{1}(x,y)=\tfrac{(x-j^{2})(y-j)(y-j^{2}x)}{(x-j)(y-j^{2})(y-jx)}$ es una primera integral de la foliación $\Fol{1}$ de $\CP$ y $\Pi\colon M\rightarrow \CP$ es la transformación que explota los doce puntos radiales de la foliación $\Fol{\alpha}$. Usando la carta $(x,u)$ de $M$ en la cual $\Pi(x,u)=(x,ux)$ tenemos que $\tilde{h}_{1}(x,u)=\tfrac{(x-j^{2})(ux-j)(u-j^{2})}{(x-j)(ux-j^{2})(u-j)}$. Al evaluar en el divisor $D_{1}=\{\, x=0\, \}$ obtenemos $\tilde{h}_{1}(0,u)=\tfrac{u-j^{2}}{u-j}$ y por lo tanto, la intersección de $\tilde{L}_{1}=h^{-1}(e)$ con el divisor $D_{1}=\{\, x=0\, \}$ es el punto $(0,\tfrac{j^{2}-je}{1-e}):=p_{0}$. Así, la hoja $\tilde{L}_{1}$ interseca en único punto $p_{0}$ y para todo $q\in\tilde{L}\cap D_{1}$ definimos $\pi(q)=p_{0}$.\\

Veamos que $\pi$ también es holomorfa y un homeomorfismo local en los puntos $q\in\tilde{L}\cap h^{-1}(1)$. Por el lema \ref{Lema:ExpresionLocalDeLaAplicacionCubriente} para cada punto $q\in\tilde{L}\cap h^{-1}(1)$ y $p_{0}\in\tilde{L}$ existen coordenadas $z$ y $w$ de $\tilde{L}$ y $\tilde{L}_{1}$ alrededor de $q$ y $p$ respectivamente tales que $h_{0}(z)=1+z^{3}$ y $h_{1}(w)=1+w^{3}$. Como $h_{0}=h_{1}\circ\pi$ fuera del conjunto $D_{1}\cap\tilde{L}$ tenemos que $z^{3}=(w(\pi(z)))^{3}$ para toda $z\neq 0$. Así, $w(\pi(z))=j^{i}z$ y por lo tanto $\pi$ es holomorfa y un homemorfismo local alrededor de cada punto $q\in\tilde{L}\cap h^{-1}(1)$.\\

Para ver que $\pi$ es una aplicación cubriente basta mostrar que cumple la propiedad de levantamiento de curvas. Sea $\tilde{\gamma}\colon [0,1]\rightarrow\tilde{L}_{1}$ una curva en la hoja $\tilde{L}_{1}$ y consideremos la curva $\gamma:=h_{1}\circ\tilde{\gamma}$. Usando a $h_{0}$ podemos levantar a la curva $\gamma$ a una curva $\hat{\gamma}\colon [0,1]\rightarrow\tilde{L}_{0}$. Afirmamos que $\hat{\gamma}$ es un levantamiento de $\tilde{\gamma}$. En efecto, como $\hat{\gamma}$ es un levantamiento de $\gamma$ tenemos que $h_{1}\circ\tilde{\gamma}=\gamma=h_{0}\circ\hat{\gamma}$, pero por construcción, $h_{0}=h_{1}\circ\pi$ y así, $h_{1}\circ\tilde{\gamma}=(h_{1}\circ\pi)\circ\hat{\gamma}$ y como $h_{1}$ es un homeomorfismo local tenemos que punto a punto $\tilde{\gamma}=\pi\circ\hat{\gamma}$. Hemos probado así, la siguiente proposición.

\begin{Proposicion}
\label{Prop:GeneroDeLasSoluciones}
Si la foliación $\Fol{\alpha}$ de $\CP$ tiene primera integral racional, entonces casi cualquier hoja de la foliación tiene género uno.
\end{Proposicion}

Como vimos en la sección anterior, podemos escoger un parámetro $\alpha$ de tal forma que el grado de la primera integral de la foliación $\Fol{\alpha}$ sea arbitrariamente grande, pero como afirma la proposición anterior, el género de las soluciones es uno.
