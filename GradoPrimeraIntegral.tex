En esta sección veremos que existe un subconjunto denso $E$ de parámetros $\alpha\in\overline{\C}$ tal que si $\alpha\in E$, entonces la foliación $\Foli_{\alpha}$ tiene una primera integral racional. En caso de que dicha primera integral exista sabemos que el grupo de holonomía global $G_{\alpha}$ de la foliación $\tilde{\Foli}_{\alpha}=\Pi^{*}(\Foli_{\alpha})$ es finito (proposición \ref{Prop:EquivalenciasIntegrabilidad}). Vamos a acotar por abajo el grado de la primera integral en términos de la cardinalidad del grupo de holonomía global y mostraremos que para una natural $k_{0}$ fijo, los parámetros $\alpha\in E$ tales que la primera integral de la foliación $\Foli_{\alpha}$ es de grado menor que $k_{0}$, forman un subonjunto finito.\\

Por la proposición \ref{Prop:EquivalenciasIntegrabilidad} sabemos que la foliación $\Foli_{\alpha}$ tiene primera integral racional si existe $n\in\mathbb{N}$ tal que $nA(\alpha)\in\mathbb{Z}\oplus j\mathbb{Z}=\Gamma$, donde $g_{\alpha}(z)=jz+A(\alpha)$ es uno de los dos generadores del grupo de holonomía global $G_{\alpha}$. La condición $nA(\alpha)\in\mathbb{Z}\oplus j\mathbb{Z}$ es equivalente a que $A(\alpha)\in\mathbb{Q}\oplus j\mathbb{Q}$. El conjunto $A(\alpha)\in\mathbb{Q}\oplus j\mathbb{Q}$ es un subconjunto denso del toro $\C/\Gamma$. Veremos que la función $A\colon\C\rightarrow\C/\Gamma$ es holomorfa, no constante y por lo tanto localmente abierta. Esto implica que la preimagen de $\mathbb{Q}\oplus j\mathbb{Q}$, es densa en $\C$. Así pues, vamos a estudiar algunas propiedades de la función $A\colon\C\rightarrow \C/\Gamma$.\\

Primero veamos que $A\colon\C\rightarrow\C/\Gamma$ es holomorfa. La transformación $g_{\alpha}=jz+A(\alpha)$ está generada por un lazo $\gamma\in\pi_{1}(\C\setminus\{0,1\},a)$. Si fijamos este lazo obtenemos una función $\Phi_{\gamma}\colon\C\times T_{a}\rightarrow T_{a}$, donde $T_{a}=h^{-1}(a)\simeq\C/\Gamma$ es una transversal a todas las foliaciones $\tilde{\Foli}_{\alpha},\, \alpha\in\C$. Esta función está definida como $\Phi_{\gamma}(\alpha,p)=g_{\alpha}(p)+A(\alpha)$, esto quiere decir que para $\alpha\in\C$, $\Phi_{\gamma}(\alpha,z)=g_{\alpha}(z)+A(\alpha)$. Como las soluciones de la ecuación diferencial que genera a cada una de estas foliaciones en una vecindad de cada punto dependen de manera holomorfa de las condiciones iniciales y del parámetro $\alpha$ (teorema \ref{Teo:ExistenciaUnicidad}), tenemos que $A(\alpha)$ es holomorfa.\\

Para ver que $A(\alpha)$ es no constante veremos que los grupos de holonomía de las foliaciones $\tilde{\Foli}_{0}$ y $\tilde{\Foli}_{1}$ tienen distinto número de elementos y por lo tanto $A(0)\neq A(1)$.\\

Por la proposición \ref{Prop:2} tenemos que $$H_{1}=\frac{(x-j^{2})(y-j)(y-j^{2}x)}{(x-j)(y-j^{2})(y-jx)}=\frac{p_{1}}{q_{1}},$$ es una primera integral racional de $\Foli_{1}$. Por el teorema de Bézout tenemos que $\{p_{1}=0\}$ interseca a $\{q_{1}=0\}$ en nueve puntos y cada uno de estos puntos está en todas las curvas de nivel de $H_{1}$. Con un razonamiento similar, la curva de nivel $H_{\infty}^{-1}(a)=\Pi(T_{a})$ siempre pasa por los nueve puntos de intersección de $\{p_{1}=y^{3}-1=0\}$ y $\{q_{1}=x^{3}-1=0\}$. Con esto no es difícil convencerse (ver figura...) que $H^{-1}_{1}(c)$ interseca a $H^{-1}_{\infty}(a)$ en los seis puntos $(1,j),(1,j^{2}),(j,1),(j,j),(j^{2},1)$ y $(j^{2},j^{2})$. Así, de los nueve puntos de intersección de $H^{-1}_{1}(c)$ y $H^{-1}_{\infty}(a)$, sólo tres de ellos no son alguno de los doce puntos que explota la transformación $\Pi\colon M\rightarrow\CP$. Las intersecciones en cada uno de estos seis puntos es transversal y entonces, al explotar, estas intersecciones desaprecen y por lo tanto el transformado estricto de $H^{-1}_{1}(c)$ bajo $\Pi$ (que es una hoja de la foliación $\tilde{\Foli_{1}}$) interseca a $T_{a}=\Pi^{-1}(H^{-1}_{\infty}(a))$ en tres puntos. Esto implica que el grupo de holonomía global $G_{1}$ sólo tiene tres elementos y así $f_{1}(z)=g_{1}(z)=jz$. Por lo tanto, $A(0)=0$.\\

A continuación veremos que $A(0)\neq 0$ (mod $\Gamma$). Notemos primero que $$H_{0}(x,y)=\frac{x^{3}(y^{3}-1)}{y^{3}(x^{3}-1)}=\frac{p_{0}}{q_{0}},$$ es una primera integral racional de $\Foli_{0}$. En efecto, $$q_{0}^{2}dH_{0}=q_{0}dp_{0}-p_{0}dq_{0}=3x^{2}y^{2}[x(x^{3}-1)\, dy-y(y^{3}-1)\, dx].$$

La recta $\{x=0\}$ que es parte de la curva de nivel cero de $H_{0}$ corta a $H_{\infty}(a)$ en tres puntos $(0,y_{k})$, donde $y_{1},y_{2},y_{3}$ son las raíces de $a=y^{3}-1$. $H_{0}^{-1}(0)$ tiene multiplicidad tres a lo largo de la recta $\{x=0\}$ así que para $b$ cercana a $0$, $H_{0}^{-1}(b)$ interseca a $H_{\infty}^{-1}(a)$ en nueve puntos  pues cada uno de los puntos de intersección $(0,y_{k})$, al variar $b$, se separa en tres puntos cercanos a $(0,y_{k})$ y además, para $b$ suficientemente pequeña ninguno de estos nueve puntos coincide con los puntos que explota $\Pi$. Por lo tanto, al explotar, el transformado estricto de $H_{0}^{-1}(b)$ interseca a la tansversal $T_{a}=h^{-1}(a)$ en al menos nueve puntos y así, el grupo $G_{0}$ tiene al menos nueve elementos. Esto quiere decir que $A(0)\neq 0$. Ahora podemos probar el siguiente teorema.

\begin{Teorema}
\label{Teo:DensidadPrimerasIntegrales}
Para la familia $\Foli_{\alpha}$ de foliaciones de $\CP$ existe un subconjunto denso de parámetros $E\subset\overline{\C}$ tal que si $\alpha\in E$, la foliación $\Foli_{\alpha}$ tiene una primera integral racional.
\end{Teorema}
\begin{proof}
Como la función $A\colon\C\rightarrow\C/\Gamma$ es holomorfa y no constante, $E=A^{-1}(\mathbb{Q}\oplus j\mathbb{Q})$ es un subconjunto denso de $\C$. Si $\alpha\in\ E$ tenemos que $A(\alpha)\in\mathbb{Q}\oplus j\mathbb{Q}$ y por lo tanto existe $n\in\mathbb{N}$ tal que $nA(\alpha)\in\Gamma=\mathbb{Z}\oplus j\mathbb{Z}$. De acuerdo al teorema \ref{Prop:EquivalenciasIntegrabilidad} la foliación $\Foli_{\alpha}$ tiene una primera integral racional.
\end{proof}

Si $\alpha\in E\subset\overline{\C}$ entonces la foliación $\Foli_{\alpha}$ tiene una primera integral racional de grado $d(\alpha)$ y para el grupo de holonomía global $G_{\alpha}$ de la foliación $\tilde{\Foli}_{\alpha}$ existe un $k=k(\alpha)\mathbb{N}$ tal que  $G_{\alpha}$ tiene $3k(\alpha)^{2}$ elementos (corolario \ref{Coro:Equivalencias} y proposición \ref{Prop:EquivalenciasIntegrabilidad}).

\begin{Lema}
\label{Lema:Cota}
Si el grupo de holonomía global $G_{\alpha}$ tiene $k(\alpha)^{2}$ elementos, entonces el grado $d(\alpha)$ de la primera integral de la foliación $\Foli_{\alpha}$ cumple $k(\alpha)^{2}\leq d(\alpha)$.
\end{Lema} 
\begin{proof}
Sea $L$ una hoja de la foliación $\Foli_{\alpha}$ y $\tilde{L}$ su transformado estricto bajo $\Pi$, por el teorema de Bézout tenemos que,
$$3d(\alpha)=\#\{L\cap H_{\infty}^{-1}(a)\}\geq\#\{\tilde{L}\cap h_{\infty}^{-1}(a)\}=\#G_{\alpha}=3k(\alpha)^{2}.$$
\end{proof}

\begin{Teorema}
\label{Teo:GradosGrandes}
Dado un natural $n_{0}\in\mathbb{N}$ el subconjunto $E_{n_{0}}:=\{\alpha\in E;\, k(\alpha)\leq n_{0}\, \}\cup\{\infty \, \}$ es finito. Es decir, las foliaciones $\Foli_{\alpha}$ de $\CP$ con primera integral racional de grado $d(\alpha)\leq n_{0}$ son un número finito. 
\end{Teorema}
\begin{proof}
Si el conjunto $E_{n_{0}}$ fuera infinito tendríamos que para algún número $r\in\{\, 1,\ldots,n_{0} \, \}$, el conjunto $E_{r}:=\{\, \alpha\in E_{n_{0}};\, k(\alpha)=p \, \}\cup\{\, \infty\, \}$ es infinito. Del corolario \ref{Coro:Equivalencias} tenemos que para todo $\alpha,\beta\in E_{r}$, $A(\alpha)=A(\beta)$ (mod $\Gamma$) y entonces el suconjunto $E_{r}$ tiene una sucesión convergente $\{\, a_{i} \, \}_{i\in\mathbb{N}}$ tal que $A(a_{i})=r$ para toda $i\in\mathbb{N}$ esto implica que $A(z)$ es constante lo cual es un contradicción.
\end{proof}
