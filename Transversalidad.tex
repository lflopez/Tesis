%Para probar la segunda parte del teorema principal usaremos una curva algebraica que sea transversal a todas las hojas de todas las foliaciones de la familia. Mediante esta tranversal obtendremos un grupo de holonomía. Algunas propiedades de este grupo nos ayudarán a probar la segunda parte del teorema. 

La siguiente porposición nos brinda información de cómo se intersecan hojas de distintas foliaciones de M que son inducidas por $(\Fol[4]{\alpha})_{\alpha\in\overline\C}$ y $\Pi\colon M\rightarrow\CP$, pero antes, recordemos parte de la notación que se usó antes de la proposición \ref{Prop:2}.

\begin{enumerate}


\item Llamemos M a la variedad que obtenemos de explotar y resolver los 12 puntos singulares de $\mathcal{P}$ y denotemos por $\Pi\colon M \rightarrow \CP$ al mapeo que resuelve las singularidades.

\item $\tilde{\Fol{\alpha}}$ será la foliación en $M$ inducida por $\Fol[4]{\alpha}$,  $\mathit{i.e.} \ \Pi^{*}(\Fol[4]{\alpha}) = \tilde{\Fol{\alpha}}$.

\item $D_{i}$ va a ser el divisor asociado a $p_{i} \in \mathcal{P} \ i=1,...,12. \ D_{i} = \Pi^{-1}(p_{i})$.

\item Para cada $l_{j} \in \mathcal{L}$, en el conjunto de las nueve rectas que deja invariante toda foliación de la familia,  denotaremos por $\tilde{l}_{i}$ al transformado estricto bajo $\Pi$ de la recta $l_{i}$. Es decir, $\tilde{l}_{i}= \overline{\Pi^{-1}(l_{i} \setminus \{p_{i1}, p_{i2}, p_{i3}, p_{i4} \})}$, donde $p_{ik} \ k=1,...,4$ son los cuatro puntos de $\mathcal{P}$ que están en $l_{i}$.

\end{enumerate}


\begin{Proposicion}
\label{Prop:Transversalidad}
Si $\alpha\neq\beta$ entonces las foliaciones $\tilde{\Fol{\alpha}}$ y $\tilde{\Fol{\beta}}$ se intersecan transversalmente afuera del conjunto $\tilde{\mathcal{L}} = \tilde{l}_{1}\cup\cdots\cup\tilde{l}_{9}$.
\end{Proposicion}

\begin{proof}
Para demostrar la proposición vamos a usar las expresiones (\ref{CampVect4}) y (\ref{CampVect4Inf}) para mostrar la transversalidad en aquellos lugares donde la transformación $\Pi\colon M\rightarrow \CP$ lo permita. En los lugares donde no podamos hacer esto (los divisores que obtenemos al explotar los doce puntos) vamos a usar las cartas de la explosión para mostrar que las foliaciones se cortan de manera transversal. Supongamos que $\alpha , \beta\neq\infty$, primero probaremos que las foliaciones $\tilde{\Fol{\alpha}},\tilde{\Fol{\beta}}$ son transversales afuera del conjunto $\tilde{\mathcal{L}}\cup D_{1}\cup\cdots\cup D_{12}$. Como $\Pi$ es un biholomorfismo fuera de $D_{1}\cup\cdots\cup D_{12}$, basta demostrar que $\Fol[4]{\alpha},\Fol[4]{\beta}$ son transversales afuera del conjunto $l_{1}\cup\ldots\cup l_{9}$. Si usamos los campos vectoriales que generan a las foliaciones y calculamos su determinante obtenemos:

\begin{equation*}
P_{\alpha}Q_{\beta}-P_{\beta}Q_{\alpha}=(\beta - \alpha)(x^{3}-1)(y^{3}-1)(y^{3}-x^{3}).
\end{equation*}

Todavía nos hace falta probar que son transversales en la recta al infinito. Usando  el corolario \ref{Coro1Prop1} podemos encontrar un automorfismo $S$ de $\CP$ que mande a los puntos $[1:0:0],[0:1:0]$ a otros dos puntos de $\mathcal{P}$ pero que estén en la parte finita y por lo tanto la recta al infinito ahora está en la parte finita. Como las transformaciones que fijan a la configuración cumplen $S^{*}(\Fol[4]{\alpha})=\Fol[4]{\gamma}$, el cálculo anterior muestra que las foliaciones también son transversales en la recta al infinito.\\

Resta ver que $\tilde{\Fol{\alpha}},\tilde{\Fol{\beta}}$ también son transversales en $D_{1}\cup\cdots\cup D_{12}\setminus\tilde{\mathcal{L}}$. Como por cada punto $p_{i}\in\mathcal{P}$ tenemos un divisor $D_{i}$, si usamos el corolario \ref{Coro2Prop1} sólo tenemos que ver que $\tilde{\Fol{\alpha}},\tilde{\Fol{\beta}}$ son transversales en uno de los doce divisores, usemos  $D_{1}=\Pi^{-1}(p_{1}=(0,0))$. Si explotamos el origen usando la carta $(u,x)$ en la cual  $\Pi(u,x)=(x,ux)$, las tres rectas de $\mathcal{L}$ que pasan por el origen ($y=x,y=jx,y=j^{2}x$) se transforman en las rectas $u=1,u=j,u=j^{2}$, el divisor $D_{1}$ queda descrito por $x=0$ y el campo vectorial que genera a $\tilde{\Fol{\alpha}}$ en esta carta es:

\begin{align*}
\dot{u}&= \alpha(u^{3}-1) + xh_{1}(u,x)\\
\dot{x}&= 1 + xh_{2}(u,x).
\end{align*}  

\noindent donde $h_{1}$ y $h_{2}$ son polinomios. Por lo tanto en el divisor $D_{1}$ las pendientes de $\tilde{\Fol{\alpha}},\tilde{\Fol{\beta}}$ son $\tfrac{du}{dx}=\alpha(u^{3}-1)$ y $\tfrac{du}{dx}=\beta(u^{3}-1)$ respectivamente; esto quiere decir que las foliaciones son transversales en $D_{1}\setminus\tilde{\mathcal{L}}$ (si usamos la otra carta, en la cual $\Pi(y,v)=(vy,y)$, obtenemos un resultado análogo).

\noindent Cuando $\beta=\infty$ procedemos de manera análoga, en los lugares donde $\Pi$ es un biholomorfismo tenemos:
\begin{equation*}
P_{\alpha}Q_{\infty}-P_{\infty}Q_{\alpha}=(x^{3}-1)(y^{3}-1)(x^{3}-y^{3}),
\end{equation*}
\noindent este determinante sólo se anula en las nueve rectas que dejan invariantes ambas foliaciones y por lo tanto las foliaciones son transversales fuera de los 12 divisores y las nueve rectas invariantes. Al explotar el punto $(0,0)$ y usar la misma carta $(u,x)$ que antes, la foliación $\tilde{\Foli}_{\infty}$ esta generada por el campo vectorial,
\begin{align*}
\dot{u}=&u^{3}-1\\
\dot{x}=&-(x^{3}-1)u^{2}x.
\end{align*}
\noindent Esta última expresión muestra que el punto $(0,0)$ es un punto no dicrítico pues el divisor $D_{1}$ es invariante para la foliación $\tilde{\Foli}_{\infty}$ mientras que para la foliación $\tilde{\Foli}_{\alpha}$, el punto $(0,0)$ es dicrítico y por lo tanto las foliaciones son transversales. 
\end{proof}


La foliación $\Fol{\infty}$ tiene por primera integral a $H_{\infty}=\tfrac{y^{3}-1}{x^{3}-1}$ (proposición \ref{Prop:3}), si usamos coordenadas homogéneas, las hojas de la foliación quedan descritas por $L_{c}=\{\tfrac{y^{3}-z^{3}}{x^{3}-z^{3}}=c\}$. Para $c=0$, $L_{c}=\{(y-z)(y-jz)(y-j^{2}z)=0\}$ y entonces el punto $[1:0:0]$ es un punto singular. Si $c=\infty$, $L_{c}=\{(x-z)(x-jz)(x-j^{2}z)=0\}$ y así, $[0:1:0]$ es un punto singular. Para $c=1$, $L_{c}=\{(y-x)(y-jx)(y-j^{2}x)=0\}$ y entonces $[0:1:0]$ es otro punto singular. Para todas las demás $c\in\C\setminus\{\, 0,1\, \}$, $L_{c}=\{y^{3}-z^{3}-c(x^{3}-z^{3})=0\}$ y calculando el gradiente correspondiente, es fácil ver que estas curvas son suaves y como todas están dadas por un polinomio de grado tres entonces tienen género uno (ver \cite{FischerGerd}).\\

Observemos que los doce puntos singulares radiales están contenidos en las curvas de nivel singulares de $H_{\infty}$, por lo tanto, si consideramos la función $h=H_{\infty}\circ\Pi\colon M\rightarrow\overline{\C}$ y tomamos un valor regular $c\in\overline{\C}\setminus(0,1,\infty)$ de la función $H_{\infty}$, entonces $h^{-1}(c)$ es topológicamente equivalente a un toro.

\begin{Obs}
\label{Obs:FibrasSingulares}
El conjunto $h^{-1}(1)$ está formando por el divisor $D_{1}$ que resulta de explotar el punto $(0,0)$ y el transformado estricto bajo $\Pi$ de las rectas $\{\, (y-x)(y-jx)(y-j^{2}x)=0\, \}$. En efecto, $H_{\infty}^{-1}(1)=\{\, (y-x)(y-jx)(y-j^{2}x)=0\, \}$ y como se vio al final de la proposición \ref{Prop:Transversalidad} el divisor $D_{1}$ es invariante para la foliación $\tilde{\Fol{\infty}}$ y este divisor se proyecta bajo $\Pi$ al punto $(0,0)\in H_{\infty}^{-1}(1)$. Los otros puntos singulares que explota la transformación $\Pi$ y que están en las rectas $\{\, (y-x)(y-jx)(y-j^{2}x)=0\, \}$ son puntos singulares radiales (ver proposición \ref{Prop:3}) y por lo tanto al explotar estos puntos, las hojas de la foliación $\tilde{\Fol{\infty}}$ cortan de manera transversal a los divisores correspondientes y así, los divisores asociados a estos puntos singulares no están contenidos en $h^{-1}(1)$. De manera análoga $h^{-1}(0)$ y $h^{-1}(\infty)$, (cada uno de ellos) está formado por otro divisor junto con el transformado estricto de otras tres rectas que deja invariante cualquier foliación de la familia. Es decir, $h^{-1}(0,1,\infty)=D_{1}\cup D_{2}\cup D_{3}\cup_{i=1}^{9}\tilde{l_{i}}$, donde $D_{1},D_{2},D_{3}$ son los divisores que se obtienen al explotar los puntos (en coordenadas homogéneas) $[0:0:1][0:1:0][1:0:0]$ respectivamente y $\tilde{l}_{i}$ es el transformado estricto de alguna de las nueve rectas que deja invariante cualquier foliación $\Fol{\alpha}$ de la familia.
\end{Obs}

 Si $V=M\setminus h^{-1}(0,1,\infty)$ y $\Omega=\C\setminus\{0,1\}$ entonces $h\colon V\rightarrow\Omega$ satisface las hipótesis del lema \ref{Lema:HazTopologico} y por lo tanto, $(V,h,\Omega)$ es un haz topológico con fibra $T$ isomorfa a un toro. Como $\tilde{\Foli_{\alpha}^{4}}$ es transversal a $\tilde{\Foli_{\infty}^{4}}$, si fijamos una fibra no singular $T_{a}=h^{-1}(a)$ obtenemos el grupo de holonomía global $G_{\alpha}$ de $T_{\alpha}$ asociado a la foliación $\tilde{\Foli_{\alpha}^{4}}$ y al haz fibrado $(V,h,\Omega)$ (ver sección \ref{Sec:HolonomiaGlobal}). Como el grupo fundamental de $\Omega=\C\setminus\{0,1\}$ está generado por dos lazos $\gamma_{1}$ y $\gamma_{2}$, el grupo de holonomía global $G_{\alpha}$ está generado por dos elementos $f_{\alpha}$ y $g_{\alpha}$. Antes de continuar veamos quién es la fibra de este haz topológico.
\begin{Proposicion}
\label{Prop:FibraDelHaz}
Sea $(V,h,\Omega)$ el haz topológico de arriba, si $\Gamma=\mathbb{Z}\oplus j\mathbb{Z}$ entonces, $h^{-1}(a)\simeq\C/\Gamma$.
\end{Proposicion}
\begin{proof}
Si fijamos una foliación $\tilde{\Foli_{\alpha}^{4}}$ ,como el grupo fundamental de $\Omega$ está generado por dos elementos ${\gamma_{1},\gamma_{2}}$, entonces  el grupo de holonomía $G_{\alpha}$ está generado por dos biholomorfismos $f_{\alpha},g_{\alpha}\colon T_{a}\rightarrow T_{a}$. La foliación $\tilde{\mathcal{F}}_{\alpha}^{4}$ sólo tiene nueve puntos singulares y de manera local, alrededor de cada uno de estos puntos, la foliación está generada por el campo vectorial $3u\tfrac{\partial}{\partial u}-v\tfrac{\partial}{\partial v}$ (proposición \ref{Prop:2}). Así, la transformación de holonomía asociada a la separatriz local $\{v=0\}$ es $v\mapsto e^{\tfrac{2\pi i}{3}}v=jv$ y tiene orden tres. Si tomamos a $T_{a}$ como una transversal a la separatriz local $\{v=0\}$, entonces $f_{\alpha}^{3}=g_{\alpha}^{3}=Id$ y tanto $f_{\alpha}$ como $g_{\alpha}$ fijan un punto. Un toro que tenga un automorfismo con las propiedades anteriores es biholomorfo a $\C/\Gamma$ donde $\Gamma=\mathbb{Z}\oplus j\mathbb{Z} \}$ (ver apéndice \ref{ApendiceToros}).
\end{proof}

\section{El grupo de holonomía global de $\tilde{\Foli_{\alpha}^{4}}$.}

Para probar que una foliación $\Foli_{\alpha}^{4}$ tiene una primera integral racional vamos a usar agunas propiedades del grupo de holonomía global que se construyó en la sección anterior. En esta sección probaremos algunas propiedades de este grupo. 

\begin{Lema}
\label{Lema:GrupoHolonomia}
Sea $G_{\alpha}$ el grupo de holonomía global asociado al haz fibrado $(V,h,\Omega)$ y la foliación $\tilde{\Foli_{\alpha}^{4}}$. Entonces en una carta adecuada de $\C$, $G_{\alpha}$ está generado por $f_{\alpha}(z)=jz$ y $g_{\alpha}(z)=jz+A(\alpha)$.
\end{Lema}
\begin{proof}
Si fijamos coordenadas de una cubierta universal $P\colon \C\rightarrow T_{a}$ entonces en esta cubierta, $f_{\alpha}(z)=jz+a(\alpha)$ y $g_{\alpha}=jz+b(\alpha)$ (ver apéndice \ref{ApendiceToros}). Notemos que $z_{0}(\alpha):=\tfrac{a(\alpha)}{1-j}$ es un punto fijo de $f_{\alpha}$, por lo tanto, si $k_{\alpha}(z)=z+z_{0}(\alpha)$ entonces $k_{\alpha}^{-1}\circ f_{\alpha}\circ k_{\alpha}(z)=jz$. En estas nuevas coordenadas, $k_{\alpha}^{-1}\circ g_{\alpha}\circ k_{\alpha}(z)=jz+b(\alpha)-a(\alpha)=jz+A(\alpha)$. 
\end{proof}

A continuación veremos cómo se ve un subrgupo de biholomorfismos de $\C/\Gamma$ generado por dos elementos $f(z)=jz$ y $g(z)=jz+A$.

\begin{Proposicion}
\label{Prop:DescripcionDelGrupo}
Sea $G$ un subgrupo de biholomorfismos de $\C/\Gamma$ generado por $f(z)=jz$ y $g(z)=jz+A$, entonces $G=\{j^{i}z+A;\, i\in\{0,1,2\},\, d\in\mathbb{Z}\oplus j\mathbb{Z} \}$.
\end{Proposicion}
\begin{proof}
Si $G_{1}=\{j^{i}z+A;\, i\in\{0,1,2\},\, d\in\mathbb{Z}\oplus j\mathbb{Z} \}$, es claro que $G\subset G_{1}$. Para demostrar que $G_{1}\subset G$ basta ver que $z+A$ y $z+jA$ están en $G$ pues en este caso, si $d=m+jn$ tenemos que $\mu(z)=z+dA=z+(m+nj)A=z+mA+njA\in G$ y si a $\mu$ le anteponemos $f$ y $f^{2}$ obtenemos las transformaciones $jz+dA$ y $j^{2}z+dA$ respectivamente. Así pues, $g\circ f^{2}=j(j^{2}z)+A=z+A$ y $f\circ g\circ f=j(j^{2}z+A)=z+jA$
\end{proof}

Las siguientes equivalencias nos van a ayudar a probar la existencia de primeras integrales racionales.

\begin{Corolario}
\label{Coro:Equivalencias}
Sea $G$ un grupo como el de la proposición anterior, entonces son equivalentes:
\begin{enumerate}
\item $G$ es finito.
\item $G$ tiene órbita finita en $\C/\Gamma$.
\item Existe $n\in\mathbb{N}$ tal que $nA\in\Gamma$.
\end{enumerate}
\end{Corolario}
\begin{proof}
$1\Rightarrow 2$. Es claro.

\noindent $2\Rightarrow 3$ Sea $\mu(z)=z+A$, fijemos $z_{0}\in\C/\Gamma$ y consideremos al conjunto $\{z_{0},\mu(z_{0}),\mu^{2}(z_{0}),\ldots \}$. Como la órbita de $z_{0}$ es finita tienen que existir $m,n\in\mathbb{N},\, m<n$ tales que $\mu^{m}(z_{0})=\mu(z_{0})^{n}$ (mod $\Gamma$). Esto quiere decir que $mA=nA$ (mod $\Gamma$) y por lo tanto $(n-m)A\in\Gamma$.

\noindent $3\Rightarrow 1$. Para probar este úlitmo inciso y para uso posterior, vamos a calcular explícitamente el número de elementos de $G$. Sea $k\in\mathbb{N}$ el natural más pequeño que satisface $kA\in\Gamma$ y $\mu(z)=j^{i}z+dA\in G$, encontraremos una manera equivalente de escribir a $\mu$. Si $d=(p+jq)$ y dividimos a $p$ y $q$ entre $k$ obtenemos $p=ka +\tilde{p}$ y $q=kb+\tilde{q}$ con $0\leq\tilde{p},\tilde{q}<k$ y entonces podemos escribir a $\mu$ como $\mu(z)=j^{i}+k(a+jb)A+(\tilde{p}+j\tilde{q})A$; módulo $\Gamma$, $k(a+jb)A=0$, puesto que $kA\in\Gamma$ por lo tanto, $\mu(z)$ induce la misma transformación que $\tilde{\mu}(z)=j^{i}z+(\tilde{p}+j\tilde{q})A$ y además $0\leq\tilde{p},\tilde{q}<k$. Lo anterior quiere decir que toda transformación $\mu(z)=j^{i}z+dA$ la podemos escribir con una $d=m+jn$ tal que $0\leq m,n<k$. Con esto no es difícil convencerse de que $G$ tiene $3k^{2}$ elementos.
\end{proof}

Si $\tilde{L}$ es una hoja de la foliación $\tilde{F}_{\alpha},\, \alpha\neq\infty$, la construcción del grupo de holonomía global (ver sección \ref{Sec:HolonomiaGlobal}) nos asegura que la función $h\colon \tilde{L}\setminus h^{-1}(0,1,\infty)\rightarrow\C\setminus\{\, 0,1 \, \}$ es una aplicación cubriente. Para ver bajo qué condiciones la hoja $L=\Pi(\tilde{L})$ es algebraica, vamos a necesitar una expresión local de la función $h\colon \tilde{L}\rightarrow\overline{\C}$ alrededor de los puntos $p\in\tilde{L}\cap h^{-1}(0,1,\infty)$.

\begin{Lema}
\label{Lema:ExpresionLocalDeLaAplicacionCubriente}
Sea $\tilde{L}$ una hoja de la foliación $\tilde{F}_{\alpha}, \alpha\neq\infty$ que no pasa por ninguno de los nueve puntos silla $\tilde{q}_{i}(\alpha)$ de la foliación, entonces, alrededor de cada punto $p\in\tilde{L}\cap h^{-1}(0,1,\infty)$ existe una carta local $z$ alrededor de $p$ tal que en esta carta la función $h\colon \tilde{L}\rightarrow\overline{\C}$ adquiere la expresión $h(z)=1+z^{3}$.
\end{Lema}

\begin{proof}
Como hemos venido haciendo, basta probar la afirmación para $p\in h^{-1}(1)=D_{1}\cup\tilde{l}_{1}\cup\tilde{l}_{2}\cup\tilde{l}_{3}$ (ver observación \ref{Obs:FibrasSingulares}) y despúes usar el corolario \ref{Coro2Prop1} y el lema \ref{Lema:Jalando} para ver que la afirmación también se vale en cada punto del conjunto $\tilde{L}\cap h^{-1}(0,\infty)$. Como la hoja $\tilde{L}$ no pasa por ninguno de los nueve puntos silla $\tilde{q}_{i}\in\tilde{l}_{i}$ y las rectas $\tilde{l}_{i}$ son invariantes para la foliación $\tilde{\Fol{\alpha}}$, entonces $\tilde{L}\cap h^{-1}(1)\subset D_{1}$. Usando la carta $(x,u)$ de $M$ en la cual $\Pi(x,u)=(x,ux)$ tenemos que $h(x,u)=H_{\infty}\circ\Pi(x,u)=\tfrac{u^{3}x^{3}-1}{x^{3}-1}$ y así, $h(x,u)-1=\tfrac{x^{3}(u^{3}-1)}{x^{3}-1}$. En esta carta el divisor $D_{1}$ queda descrito por $\{\, x=0\, \}$ y como $\tilde{L}$ interseca de manera transversal al divisor $D_{1}$ (proposición \ref{Prop:Transversalidad}) podemos parametrizar a $\tilde{L}$ alrededor de $p\in\tilde{L}\cap h^{-1}(1)$ con una función $x\mapsto (x,u(x))$ donde $u\colon (\C,0)\rightarrow\C$ es holomorfa. Así, $h-1\colon \tilde{L}\rightarrow\overline{\C}$ adquiere la expresión $h(x,u(x))-1=\tfrac{x^{3}((u(x))^{3}-1)}{x^{3}-1}$ alrededor del punto $p$. Como $u(0)\notin\{\, 1,j,j^{2}\, \}$ pues estos valores de $u$ se corresponden con las intersecciones del divisor $D_{1}$ con las rectas $\tilde{l}_{i}$, concluimos que la función $h-1\colon \tilde{L}\rightarrow\overline{\C}$ tiene orden tres alrededor del punto $p\in\tilde{L}\cap h^{-1}(1)$ y por lo tanto existe una carta coordenada $z$ alrededor de $p\in\tilde{L}$ tal que, $h(z)=1+z^{3}$. 
\end{proof}

\begin{Proposicion}
\label{Prop:EquivalenciasIntegrabilidad}
Sea $G_{\alpha}=\langle jz,jz+A(\alpha)\rangle$ el grupo de holonomía global asociado a $\tilde{F}_{\alpha}$, entonces son equivalentes:
\begin{enumerate}
\item $\Foli_{\alpha}^{4}$ tiene una primera integral racional.
\item $\Foli_{\alpha}^{4}$ tiene una hoja algebraica que no está contenida en las líneas invariantes de $\mathcal{L}$.
\item Existe un natural $n\in\mathbb{N}$ tal que $nA(\alpha)\in\Gamma$.
\end{enumerate}
\end{Proposicion} 
\begin{proof}
$1\Rightarrow 2$. Es claro.

\noindent $2\Rightarrow 3$ Sea $L$ una hoja algebraica de $\Foli_{\alpha}^{4}$ que no está contenida en las líneas de $\mathcal{L}$. Por el teorema de Bézout tenemos que $L\cap H^{-1}(a)$ es un conjunto finito. Si $\tilde{L}$ es el transformado estricto de $L$ ($\mathit{i.e.}$ $\tilde{L}=\overline{\Pi^{-1}(L\setminus\mathcal{P})}$ donde $\mathcal{P}$ son los puntos que explota $\Pi^{-1}$), entonces $\tilde{L}\cap T_{a}$ es un conjunto finito pues $T_{a}=\Pi^{-1}(H^{-1}(a))$, pero $\tilde{L}\cap T_{a}$ es una órbita de $G_{\alpha}$ y por el corolario anterior, existe $n\in\mathbb{N}$ tal que $nA(\alpha)\in\Gamma$.

\noindent $3\Rightarrow 1$ Sea $\tilde{L}$ una hoja de $\tilde{\Foli}_{\alpha}$ tal que $\Pi(\tilde{L})$ no está contenida en las nueve rectas de la configuración y tampoco pasa por los nueve puntos silla $q_{i}(\alpha)$, veremos que podemos cubrir a $\tilde{L}$ con una cantidad finita de abiertos relativamente compactos y por lo tanto $\tilde{L}$ es compacta. Por el lema anterior, en cada punto $p_{i}\in\tilde{L}\cap h^{-1}(0,1,\infty)$ existe una bola $B_{i}$ tal que $h(B_{i})$ es abierto en $\overline{\C}$ y como $h\colon\tilde{L}\setminus{h^{-1}(0,1,\infty)}\rightarrow\C\setminus\{\, 0,1\, \}$ es una aplicación cubriente, para cada punto $x\in\tilde{L}\setminus h^{-1}(0,1,\infty)$ existe un abierto $U_{x}$ relativamente compacto tal que $h(U_{x})$ es abierto en $\overline{\C}$. Los conjuntos $h(B_{i})$ y $h(U_{x})$ forman una cubierta abierta de $\overline{\C}$ así que podemos extraer una subcubierta finita $\{\, V_{i} \, \}$ (donde $V_{i}=h(U_{x})$ ó $V_{i}=h(B_{k})$). Por hipótesis $G_{\alpha}$ tiene órbita finita en $\C/\Gamma$ así que existe un $m\in\mathbb{N}$ tal que $\#\tilde{L}\cap h^{-1}(b)\leq m$ para toda $b\in \C\setminus\{\, 0,1,\infty\,\}$ (ver observación \ref{Obs:Interseccion}). Por continuidad lo mismo sucede con las intersecciones de $h^{-1}(0),\, h^{-1}(1)$ y $h^{-1}(\infty)$ con $\tilde{L}$. Así, para un conjunto abierto $V_{i},\, V_{i}=h(U_{x})$, $h^{-1}(V_{i})$ está formado por un número finito de abiertos relativamente compactos y por lo tanto $h^{-1}(V_{i})$ también es relativamente compacto. Los conjuntos $h^{-1}(V_{i})$ junto con las bolas $B_{i}$ que son un número finito, forman una cubierta finita de $\tilde{L}$ por abiertos relativamente compactos. Una subvariedad compacta de una variedad proyectiva es algebraica (referencia...) y por lo tanto la hoja $L=\Pi(\tilde{L})$ de la foliación $\Foli_{\alpha}$ es algebraica. Como la foliación $\mathcal{F}_{\alpha}$ tiene un infinidad de hojas algebraicas, tiene una primera integral racional (teorema \ref{Teo:Darboux}).
\end{proof}
