En este apéndice veremos algunos resultados básicos sobre superficies de Riemann.

\begin{defn}
Sea $X$ un espacio topológico, una \emph{carta compleja} en $X$ es un homeomorfismo $\phi\colon U\rightarrow V$, donde $U\subset X$ es un subconjunto abierto de $X$ y $V\subset\C$ es un subconjunto abierto de $\C$. Diremos que dos cartas complejas $\phi_{1}\colon U_{1}\rightarrow V_{1},\phi_{2}\colon U_{2}\rightarrow V_{2}$ son \emph{compatibles} si $U_{1}\cap U_{2}=\emptyset$ ó $\phi_{2}\circ\phi^{-1}_{1}\colon (U_{1}\cap U_{2})\rightarrow (U_{1}\cap U_{2})$ es una función holomorfa.
\end{defn}

\begin{Ejemplo}
Si $X=\mathbb{S}^{2}$ entonces la proyección estereográfica desde el polo norte ó el polo sur $\phi_{1}\colon \mathbb{S}^{2}\setminus\{\, N\, \}\rightarrow \C, \phi_{1}\colon \mathbb{S}^{2}\setminus\{\, S\, \}\rightarrow \C$ son cartas complejas que además son compatibles pues $\phi_{1}(\mathbb{S}^{2}\setminus\{\, N\, \}\cap \mathbb{S}^{2}\setminus\{\, S\, \})=\phi_{1}(\mathbb{S}^{2}\setminus\{\, N,S\, \})=\C\setminus\{\, 0\, \}$ y $\phi_{2}\circ\phi^{-1}_{1}(z)=\tfrac{1}{z}$.
\end{Ejemplo}

\begin{defn}
Un \emph{atlas complejo} en un espacio topológico $X$ es una colección de cartas complejas compatibles $\mathcal{A}=\{\, \phi_{\alpha}\colon U_{\alpha}\rightarrow V_{\alpha}\, \}$ tal que $X=\cup_{\alpha}U_{\alpha}$. Diremos que dos atlases $\mathcal{A}$ y $\tilde{\mathcal{A}}$ \emph{son compatibles} si toda carta del atlas $\mathcal{A}$ es compatible con toda carta del atlas $\tilde{\mathcal{A}}$.
\end{defn}

De la definición anterior se sigue inmediatamente que la relación ser compatible induce una relación de equivalencia en el conjunto de atlases de un espacio topológico $X$.

\begin{defn}
Una \emph{estructura compleja} en un espacio topológico $X$ es una clase de equivalencia en el conjunto de atlases de $X$.
\end{defn}

\begin{defn}
Una \emph{superficie de Riemann} es un espacio topológico $X$, conexo, Hausdorff y segundo numerable junto con una estructura compleja. 
\end{defn}

\begin{Ejemplo}
El toro. Sean $\omega_{1}$ y $\omega_{2}$ dos números complejos linealmente independientes sobre $\mathbb{R}$. Al subgrupo aditivo de $\C$, $\Gamma:=\omega_{1}\mathbb{Z}\oplus\omega_{2}\mathbb{Z}$ lo llamaremos la retícula generada por $\omega_{1}$ y $\omega_{2}$. Vamos a dotar al grupo cociente $\C/\Gamma$ de una estructura compleja. Si a $\C/\Gamma$ le damos la topología cociente, entonces $\C/\Gamma$ es un espacio Hausdorff y segundo numerable y la proyección canónica $\Pi\colon\C\rightarrow\C/\Gamma$ que a cada punto asocia su clase de equivalencia es una función continua y abierta. Las cartas que van a dar una estructura compleja a $\C/\Gamma$ las obtenemos de la siguiente manera. Sea $V\subset\C$ un abierto tal que cualesquiera dos puntos en $V$ no son equivalentes bajo $\Gamma$. Entonces, $U:=\Pi(V)$ es un abierto y $\Pi\colon V\rightarrow U$ es un homeomorfismo y su inversa es una carta de $\C/\Gamma$. Si $\mathcal{A}$ es la colección de todas las cartas que se obtienen de esta manera entonces cualesquiera dos cartas $\phi_{i}\colon U_{i}\rightarrow V_{i}, i=1,2$ son compatibles. En efecto, si $V_{1}\cap V_{2}\neq\emptyset$ entonces el cambio de coordenadas es la identidad y si $V_{1}\cap V_{2}=\emptyset$ entonces el cambio de coordenadas es una traslación por algún elemento de $\Gamma$.

Notemos además que la función que a un punto $\lambda\omega_{1}+\mu\omega_{2}\in\C/\Gamma$ le asocia el punto $(e^{2\pi i\lambda},e^{2\pi i\mu})\in\mathbb{S}^{1}\times\mathbb{S}^{1}$ es un homeomorfismo entre $\C/\Gamma$ y $\mathbb{S}^{1}\times\mathbb{S}^{1}$.
\end{Ejemplo}

\section{Funciones entre superficies de Riemann.}

\begin{defn}
Sea $f\colon X\rightarrow Y$ una función entre dos superficies de Riemann. Diremos que $f$ es \emph{holomorfa} en $p\in X$ si existen cartas $\phi_{1}$ y $\phi_{2}$ alrededor de $p$ y $F(p)$ respectivamente tal que la composición $\phi_{2}\circ f\circ\phi^{-1}_{1}$ es holomorfa.
\end{defn}

La definición anterior no depende de las cartas que se usen para verificar si $f$ es holomorfa. El siguiente teorema muestra que una función holomorfa entre superficies de Riemann tiene un comportamiento local sencillo.

\begin{Teorema}
\label{Teo:FormaNormalLocal}
Sea $f\colon X\rightarrow Y$ una función holomorfa no constante en $p\in X$. Entonces existe un entero $k\geq 1$ cartas $\phi_{1}\colon U_{1}\rightarrow V_{1}, \phi_{2}\colon U_{2}\rightarrow V_{2}$ alrededor de $p$ y $f(p)$ respectivamente tal que $\phi_{2}\circ f \circ\phi^{-1}_{1}=z^{k}$.
\end{Teorema}
\begin{proof}
Dadas cualesquiera cartas $\psi$ y $\phi_{2}$ alrededor de $p$ y $f(p)$ respectivamente, al componer con una traslación (que es una función biholomorfa), siempre podemos suponer que $\psi(P)=\phi_{2}(f(p))=0$. Una vez hecha esta traslación tenemos que $f_{1}(0):=\phi_{2}\circ f \circ\psi(0)=0$ y por lo tanto existe un entero $k\geq 1$ y una función $g$ tal que $f_{1}(z)=z^{k}g(z)$, donde $g(0)\neq 0$. Por continuidad la función $g$ no se anula en una vecindad del cero y entonces existe una función holomorfa $h$ tal que $h^{k}=g$. Si llamamos $\eta(z)=zh(z)$ entonces $f_{1}(z)=(zh)^{k}=(\eta(z))^{k}$. Como $\eta'(0)=h(0)\neq 0$, la función $\eta$ es invertible alrededor del cero y en consecuencia la composición $\phi_{1}=\eta\circ\psi$ vuelve a ser una carta alrededor de $p$. En esta carta,
\begin{align*}
\phi_{2}\circ f\circ\phi_{1}^{-1}(z) &= (\phi_{2}\circ f\circ\psi^{-1})\circ\eta^{-1}(z)\\
\, &= f_{1}\circ\eta^{-1}(z)\\
\, &= z^{k}.
\end{align*}
\end{proof}

\begin{Corolario}
\label{Coro:FuncionAbierta}
Sean $X,Y$ superficies de Riemann y $f\colon X\rightarrow Y$ una función holomorfa no constante. Entonces $f$ es abierta.
\end{Corolario}
\begin{proof}
Por el teorema anterior para cualquier abierto  $U$ suficientemente pequeño de $X$, $f(U)$ es abierto en $Y$. 
\end{proof}

\begin{Teorema}
Sean $X,Y$ superficies de Riemann. Si $X$ es compacta y $f\colon X\rightarrow Y$ es una función holomorfa no constante, entonces $Y$ es compacta y $f$ es suprayectiva
\end{Teorema}
\begin{proof}
Por el corolario anterior, $f(X)$ es un subconjunto abierto de $Y$. Como $X$ es compacto, $f(X)$ es compacto y por lo tanto cerrado en $Y$. De la conexidad de $Y$ se sigue que $f(X)=Y$ y en consecuencia $Y$ es compacto y $f$ es suprayectiva.
\end{proof}

El entero $k\geq 1$ que aparece en el teorema \ref{Teo:FormaNormalLocal} no depende de las cartas pues este número coincide con el número de preimágenes que tiene un punto $y\in Y$ suficientemente cercano a $f(p)$, y si contamos con multiplicidades, el valor $f(p)$ también tiene $k$ preimágenes.

\begin{defn}
\label{Def:Multiplicidad}
Sea $f\colon X\rightarrow Y$ una función holomorfa no constante en $p\in X$. La \emph{multiplicidad} de $f$ en $p\in X$ es el único entero $k$ tal que en algunas coordenadas locales alrededor de $p$ la función $f$ se escribe como $z\mapsto z^{k}$. A este número lo denotaremos como mult$_{p}(f)$. A un punto $p\in X$ en el cual mult$_{p}(f)>1$ lo llamaremos \emph{punto de ramificación}.
\end{defn}

Notemos que los puntos de ramificación de una función son aquellos en los cuales, en una carta local, la derivada se anula.

\begin{Ejemplo}
\label{Ej:Proyeccion}
Sea $F:\C^{2}\rightarrow \C$ una función holomorfa tal que para todos los puntos $(x,y)$ tales que $F(x,y)=0$ se tiene que $\tfrac{\partial F}{\partial x}$ ó $\tfrac{\partial F}{\partial y}\neq 0$. Bajo estas condiciones, por el teorema de la función implícita el conjunto $X:=\{\, (x,y)\in\C^{2}; F(x,y)=0 \, \}$ es una superficie de Riemann. Si restringimos la proyección en el primer factor $\Pi\colon\C^{2}\rightarrow\C$ obtenemos una función holomorfa entre superficies de Riemann. Si $(\tfrac{\partial F}{\partial y})_{p}\neq 0$ entonces, de manera local, la proyección es una carta alrededor de $p$ y en consecuencia no hay puntos de ramificación. Si $(\tfrac{\partial F}{\partial y})_{p}= 0$, como $(\tfrac{\partial F}{\partial x})_{p}\neq 0$, existe una función $\phi$ tal que $F(\phi(y),y)=0$. Al aplicar $\Pi$ obtenemos la función $y\mapsto\phi(y)$, es decir $\phi$ es una expresión local de la función $\Pi\colon X\rightarrow \C$. Como $F(\phi(y),y)=0$ al derivar respecto a $y$ obtenemos $\tfrac{\partial F}{\partial x}\tfrac{d\phi}{dy}=0$ y como $(\tfrac{\partial F}{\partial x})_{p}\neq 0$ tenemos que $\tfrac{d\phi}{dy}=0$. Es decir, si $(\tfrac{\partial F}{\partial y})_{p}= 0$, $p$ es un punto de ramificación de la proyección en el primer factor.
\end{Ejemplo}

\begin{Teorema}
\label{Teo:GradoDeUnaFuncionEntreSuperficiesDeRiemann}
Sean $X,Y$ superficies de Riemann compactas y $f\colon X\rightarrow Y$ una función holomorfa no constante. Si definimos,
\begin{equation*}
d_{y}(f)=\sum_{p\in f^{-1}(y)}\mathrm{mult}_{p}(f).
\end{equation*}
Entonces $d_{y}(f)$ es constante y no depende de $y\in Y$.
\end{Teorema}
\begin{proof}
Veremos que la función $y\mapsto d_{y}(f)$ es localmente constante y de la conexidad de $Y$ se sigue que esta función es constante. Sea $y\in Y$, como $X$ es compacta, $f^{-1}(y)$ es finito. Si $f^{-1}(y)=\{\,x_{1},\ldots,x_{n} \, \}$ por el teorema \ref{Teo:FormaNormalLocal} alrededor de cada $x_{i}$ existen vecindades $U_{i}$ tal que en coordenadas adecuadas $f\colon U_{i}\rightarrow Y$ se escribe como $z\mapsto z^{k_{i}}$. Por el corolario \ref{Coro:FuncionAbierta} $f(U_{i})$ es un subconjunto abierto de $Y$ y en consecuencia $V:=\cap f(U_{i})$ es un subconjunto abierto. Por construcción, si $y_{0}\in V$, $|f^{-1}(y_{0})|=k_{1}+\cdots +k_{n}$. 
\end{proof}

El teorema anterior nos permite hacer la siguiente definición.

\begin{defn}
\label{Def:GradoDeUnaFuncionEntreSuperficiesDeRiemann}
Sea $f\colon X\rightarrow Y$ una función holomorfa no constante entre superficies de Riemann. El \emph{grado} de $f$, denotado por deg$(f)$, es el entero $d_{y}(f)$ para cualquier $y\in Y$.
\end{defn}

\section{El género de una supericie de Riemann compacta.}

Si a una función holomorfa $f\colon U\subset\C\rightarrow\C$ la pensamos como una función diferenciable $f\colon\mathbb{R}^{2}\rightarrow\mathbb{R}^{2}$ entonces el determinante de la matriz derivada de $f$ (pensada como transformación en el plano real) en un punto $z=(x,y)$ es igual a la norma de $f'(z)$. En consecuencia, toda función holomorfa preserva la orientación. Como el cambio de coordenadas en una superficie de Riemann es una función holomorfa, tenemos que todos los cambios de coordenadas preservan la orientación. Así, toda superficie de Riemann es orientable.

Si además nuestra superficie de Riemann es compacta, el teorema de clasificación de superficies (ver referencia...) nos asegura que nuestra superficie de Riemann es una esfera ó un n-toro. Si triangulamos a una de estas superficies $X$, su característica de Euler, $\chi(X):=$(\#caras)-(\#aristas)+(\#vértices) no depende de la triangulación y $\chi(X)$ es igual a $2-2g(X)$ donde $g(X)$ es el género de la superficie $X$. Como en una función holomorfa no constante $f\colon X\rightarrow Y$ entre superficies de Riemann compactas, deg$(f)$ es constante, una buena idea es levantar una triangulación $Y$ a una triangulación de $X$ para así poder calcular la característica de Euler de la superfice $X$ en términos de la característia de Euler de la superficie $Y$ y la función $f$.

\begin{Teorema}
\label{Teo:FormulaDeRiemannHurwitz}
Sea $f\colon X\rightarrow Y$ una función holomorfa no constante entre superficies de Riemann compactas. Entonces,
\begin{equation*}
\chi(X)=\mathrm{deg}(f)\chi(Y)-\sum_{p\in X}(\mathrm{mult}_{p}(f)-1).
\end{equation*}
\end{Teorema}
\begin{proof}
Primero observemos que la suma que aparece en el lado derecho de la fórmula es una suma sobre los puntos de ramificación de $f$ pues en los demás puntos mult$_{p}(f)=1$. En una superficie de Riemann compacta los puntos de ramificación son finitos pues estos se corresponden con los puntos donde la función $f$ adquiere la expresión $z\mapsto z^{k}, k>1$, es decir, los puntos de ramifiación son aquellos donde la derivada de $f$ se anula, y en una superficie de Riemann compacta estos son un número finito.

Para mayor claridad supongamos primero que $f$ no tiene puntos de ramificación, en este caso si tomamos una triangulación de $Y$ con $c$ caras, $a$ aristas y $v$ vértices, al levantar mediante $f$ esta triangulación obtenemos una triangulación de $X$ con $\tilde{c}=\mathrm{deg}(f)c$ caras, $\tilde{a}=\mathrm{deg}(f)a$ aristas y $\tilde{v}=\mathrm{deg}(v)$ vértices. En consecuencia $\chi(X)=\mathrm{deg}(f)\chi(Y)$.

Si $f$ tiene puntos de ramificación escogemos una triangulación de $Y$ en la cual la imagen de cada punto de ramificación es un vértice. Como las caras y aristas de esta triangulación no pasan por puntos de ramificación, al levantar esta triangulación obtenemos una triangulación de $X$ con $\tilde{c}=\mathrm{deg}(f)c$ caras y $\tilde{a}=\mathrm{deg}(f)a$ aristas. Para un vértice $y\in Y$, contando multiplicidades tiene deg$(f)$ preimágenes pero para contar correctamente el número de vértices que obtenemos al levantar esta triangulación hay que quitar las multiplicidades, así, tenemos que $\tilde{v}=\mathrm{deg}(f)v-\sum_{p\in X}\mathrm{mult}_{p}(f)-1$. Por lo tanto $\chi(X)=\tilde{c}-\tilde{a}+\tilde{v}=\mathrm{deg}(f)\chi(Y)-\sum_{p\in X}(\mathrm{mult}_{p}(f)-1)$.
\end{proof}

\begin{Ejemplo}
\label{Ej:HToros}
Consideremos la superficie de Riemann $X\subset\CP$ definida por los ceros de $H(x,y,z)=y^{3}-z^{3}-c(x^{3}-z^{3}), c\in\C\setminus\{\, 0,1 \, \}$. Si consideramos la proyección $\Pi$ en el eje $x$ obtenemos una función $f\colon X\rightarrow \overline{\C}$. Vamos a usar la fórmula de Riemann-Hurwitz y el ejemplo \ref{Ej:Proyeccion} para calcular la característica de Euler de $X$. En la carta afín $X$ queda descrita por $h=y^{3}-1-c(x^{3}-1)=0$. Si fijamos un valor $x_{0}$ entonces $y^{3}-1-c(x_{0}^{3}-1)=0$ de manera genérica tiene 3 soluciones, así, deg$(\Pi)=3$. Los puntos de ramificación son aquellos donde $(\tfrac{\partial h}{\partial y})_{p}=3y^{2}=0$, es decir los puntos de ramifiación son $((\tfrac{1-c}{c})^{1/3},0)$. Estos puntos sólo tienen una preimagen y por lo tanto la multiplicidad en cada uno de ellos es 3. Sustituyendo lo anterior en la fórmula de Riemann-Hurwitz obtenemos $\chi(X)=3\cdot 2-3\cdot 2=0$. Esto quiere decir que $X$ es un toro. 
\end{Ejemplo} 
