\label{sec:FamiliaDeGradoCuatro}
Como ya habíamos mencionado, la familia de grado 4 deja invariantes a todas las rectas de $\mathcal{C}$. Usando la invarianza de las rectas  $\{x=1\}, \{x=j\}, \{x=j^{2}\}, \{y=1\}, \{y=j\}, \{y=j^{2}\}, \{y=x\}, \{y=jx\}, \{y=j^{2}x\}$ obtenemos una ecuación diferencial de la forma:



\begin{align*}
\dot{x} & =(x^{3}-1)(a_{0} + a_{10}x + a_{01}y + a_{20}x^{2} + a_{11}xy + a_{02}y^{2}) \\
\dot{y} & =(y^{3}-1)(b_{0} + b_{10}x + b_{01}y + b_{20}x^{2} + b_{11}xy + b_{02}y^{2}).
\end{align*}



Como el origen es punto singular, $a_{0}=b_{0}=0$. Y como la foliación es de grado cuatro, la parte homogénea de grado cinco de $\dot{x}$ y la de $\dot{y}$ son de la forma $xg(x,y)$ y $yg(x,y)$ respectivamente. Así:


\begin{align*}
\dot{x}& =\cdots+a_{20}x^{5} + a_{11}x^{4}y + a_{02}x^{5}y^{2} =\cdots+x(a_{20}x^{4} + a_{11}x^{3}y + a_{02}x^{2}y^{2}) \\
\dot{y}& =\cdots+b_{20}x^{2}y^{3} + b_{11}xy^{4} + b_{02}y^{5} =\cdots+y(b_{20}x^{2}y^{2} + b_{11}xy^{3} + b_{02}y^{4}) .
\end{align*}

Por lo tanto, $a_{20}=a_{11}=b_{11}=b_{02}=0$ y $a_{02}=b_{20}=b$. Usando lo anterior, nuestra ecuación adquiere la forma:


\begin{align*} 
\dot{x}&=(x^{3}-1)(a_{10}x + a_{01}y + by^{2}) \\
\dot{y}&=(y^{3}-1)(b_{10}x + b_{01}y + bx^{2}).
\end{align*}

Ahora usaremos la invarianza de las tres rectas restantes.

Si $y=x$:

$$ 1 = \frac{dy}{dx} = \frac{b_{10}x + b_{01}x + bx^{2}}{a_{10}x + a_{01}x + bx^{2}}$$
$$\Rightarrow (a_{10} + a_{01})x = (b_{10} + b_{01})x$$
\begin{equation}
\label{temp1}
\Rightarrow a_{10}+a_{01}=b_{10}+b_{01}.
\end{equation}

De manera análoga, en $y=jx$:

$$j = \frac{dy}{dx} = \frac{(b_{10}+b_{01}j)x + bx^{2}}{(a_{10}+a_{01}j)x + bj^{2}x^{2}}$$
\begin{equation}
\label{temp2}
\Rightarrow a_{10}j + a_{01}j^{2} = b_{10} + b_{01}j.
\end{equation}

Por último, al evaluar en $y=j^{2}x$ obtenemos:

\begin{equation}
\label{temp3}
a_{10}j^{2} + a_{01}j = b_{10} + b_{01}j^{2}.
\end{equation}

Sumando (\ref{temp1}), (\ref{temp2}), (\ref{temp3}) y recordando que $1 + j + j^{2} = 0$:

$$(1 + j + j^{2})(a_{10} + a_{01}) = 3b_{10} + (1+j +j^{2})b_{01}$$
$$\Rightarrow b_{10}=0.$$

Y sustituyendo en (\ref{temp1}), $b_{01}=a_{10} + a_{01}$, e insertando esto en (\ref{temp2}):

$$ a_{10}j + a_{01}j^{2 }= (a_{10} + a_{01})j$$
$$ \Rightarrow a_{01}j^{2} = a_{01}j$$
$$ \Rightarrow a_{01} = 0.$$

Y de (\ref{temp1}) nuevamente:

$$a_{10} = b_{01} = a.$$

Usando todo lo anterior la ecuación se ve como:

\begin{align*}
\dot{x} &= (x^{3}-1)(ax + by^{2}) \\
\dot{y} &= (y^{3}-1)(ay + bx^{2}).
\end{align*}

Y al dividir por $a$ y hacer $\alpha = -\frac{b}{a}$:

\begin{equation}
%\tag{$*$}
\label{CampVect4}
\begin{aligned}
\dot{x} &= (x^{3}-1)(x - \alpha y^{2}) \\
\dot{y} &= (y^{3}-1)(y - \alpha x^{2}).
\end{aligned}
\end{equation}

O sí $a = 0$:

\begin{equation}
%\tag{\#}
\label{CampVect4Inf}
\begin{aligned}
\dot{x} &= (x^{3}-1)y^{2} \\
\dot{y} &= (y^{3}-1)x^{2}.
\end{aligned}
\end{equation}

A la foliación generada por la ecuación (\ref{CampVect4}) la denotaremos por $\Fol[4]{\alpha}$ y a la que es generada por (\ref{CampVect4Inf}) la denotaremos $\Fol[4]{\infty}$.\\

Terminaremos esta sección con un lema que nos facilitará muchos cálculos en el futuro:

\begin{Lema}
\label{Lema:Jalando}
Sea $S$ un automorfismo de $\CP$ tal que $S(\mathcal{C})=\mathcal{C}$, entonces $S^{*}(\Fol[4]{\alpha})=\Fol[4]{\beta}$ para alguna $\beta\in\overline\C$.
\end{Lema}

\begin{proof}
Como el grado de la foliación no depende de la carta, $S^{*}(\Fol[4]{\alpha})$ también es una foliación de grado cuatro que deja invariantes a las nueve rectas de la configuración y por lo tanto se puede escribir como aquella generada por los campos vectoriales (\ref{CampVect4}) ó (\ref{CampVect4Inf}).
\end{proof}








