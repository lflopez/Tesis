\documentclass[10pt]{book}
\usepackage[spanish]{babel}
\selectlanguage{spanish}
\usepackage[utf8]{inputenc}
\usepackage{amsmath}
\usepackage{amsfonts}
\usepackage{amsthm}
\usepackage[margin = 1.5in]{geometry}
\usepackage{amscd}
%amsthm
\newtheorem{Teorema}{Teorema}[section]
\newtheorem{Proposicion}{Proposición}[section]
\newtheorem{Lema}{Lema}[section]
\newtheorem{Corolario}{Corolario}[section]
\newtheorem*{TeoPrin}{TEOREMA PRINCIPAL}
\theoremstyle{definition}
\newtheorem{Ejemplo}{Ejemplo}[section]
\newtheorem{defn}{Definición}[section]
\newtheorem{Not}{Notación}
\newtheorem{Obs}{Observación}
%Macros de este documento
\newcommand{\C}{\ensuremath{\mathbb{C}}}
\newcommand{\CP}{\ensuremath{\C\mathbb{P}^2}}
\newcommand{\Fol}[2][\ ]{\ensuremath{\mathcal{F}^{#1}_{#2}}}
\newcommand{\Foli}{\mathcal{F}}
\newcommand{\Pro}{\ensuremath{\mathbb{P}}}


\begin{document}
\tableofcontents
\chapter{Preliminares}

En todo este trabajo hablaremos de foliaciones holomorfas del plano proyectivo complejo $\CP$. El propósito de este capítulo es recordar algunas definiciones y resultados importantes.

     \section{Ecuaciones diferenciales y foliaciones.}

Sea $U\subset\C^{2}$ un abierto y $F=(F_{1},F_{2})\colon U\rightarrow\C^{2}$ un campo vectorial holomorfo. La ecuación diferencial (autónoma) asociada a $F$ se define como:

\begin{equation}
\label{EcuacionDiferencial}
\begin{aligned}
\frac{dx}{dt} &=F_{1}(x,y) \\
\frac{dy}{dt} &=F_{2}(x,y),\ \ t\in\C.
\end{aligned}
\end{equation}

Denotaremos por $(\dot{x},\dot{y})$ a $(\tfrac{dx}{dt},\tfrac{dy}{dt})$.\\

Una solución de esta ecuación diferencial es una curva parametrizada $\varphi\colon D\subset\C\rightarrow\C^{2}$ que satisface $\frac{d\varphi(t)}{dt} = F(\varphi(t)),\ \forall t\in V\subset\C$.\\

Un resultado fundamental en la teoría de ecuaciones diferenciales es el teorema de existencia y unicidad de soluciones. A continuación enunciamos este resultado:

\begin{Teorema}
\label{Teo:ExistenciaUnicidad}
\cite{IlyaYako} Para cualquier ecuación diferencial (\ref{EcuacionDiferencial}) y todo punto $(x_{0},y_{0})\in U$ existe un abierto $V\subset U$ y un disco $D_{\epsilon}=\{\left|t-t_{0}\right|<\epsilon \}\subset\C$, tal que la solución con condición inicial $(x,y,t)\in V\times D_{\epsilon}$ existe y es única.

Más aún, la solución depende de manera holomorfa de la condición inicial, y si $F$ depende de manera holomorfa de algunos otros parámetros, la solución también depende de manera holomorfa de estos parámetros.
\end{Teorema}

El teorema anterior nos asegura que alrededor de cada punto $x\in U$, existe una función analítica $\varphi\colon V\times D_{\epsilon}\rightarrow U$ tal que si fijamos $x_{0}\in V$, $\varphi(x_{0},t)$ es la solución de la ecuación diferencial (\ref{EcuacionDiferencial}) con condición incial $\varphi(x_{0},0)=x_{0}$. A esta función se le suele llamar \emph{flujo de la ecuación diferencial} (\ref{EcuacionDiferencial}).

A pesar de contar con el teorema de existencia y unicidad para ecuaciones diferenciales holomorfas, en principio, un campo vectorial puede ser muy complicado y las soluciones de la ecuación diferencial que determina pueden ser imposibles de escribir explícitamente. Es por eso que con frecuencia buscamos cambios de coordenadas (continuos, diferenciables ó analíticos) que simplifiquen el campo vectorial.

\begin{defn}
\label{Def:EquivalenciaAnalitica}
Decimos que dos ecuaciones diferenciales determinadas, respectivamente, por dos campos vectoriales $F$ y $F'$ definidos en los conjuntos abiertos $U$ y $U'$ son \emph{analíticamente equivalentes} si existe un biholomorfismo $H\colon U \rightarrow U'$ que cumple la siguiente relación:
\begin{equation}
\label{EquivalenciaAnalitica}
\Big(\frac{\partial H}{\partial x}\Big)F(x)=F'(H(x)).
\end{equation}
\end{defn}

Lo anterior quiere decir que la diferencial del biholomorfismo $H$ lleva el vector $F(x)$, anclado en el punto $x$, en el vector $F'(H(x))$ anclado en el punto $H(x)$.\\

Es muy común que en los textos sobre foliaciones ó ecuaciones diferenciales en variedades analíticas de dimensión dos, se usen las palabras foliación, campo vectorial y 1-forma para referirse al mismo objeto. Esta costumbre está justificada por los teoremas (\ref{Teo:FoliacionesGeneradas}) y (\ref{Teo:CamposEquivalentes}) que básicamente nos dicen que cuando en una ecuación diferencial nos olvidamos del tiempo y la parametrización de las soluciones, nos quedamos con una foliación. No es el propósito de esta tesis probar estos resultados, pero es bueno entender bien lo que éstos dicen para comprender como es que se relacionan estos objetos geométricos aparentemente distintos.\\
\begin{defn}
\label{Def:PuntoSingular}
A un punto $x\in U$ tal que $F(x)=(0,0)$ se le conoce como \emph{punto singular}. A lo largo de este trabajo denotaremos por $\Sigma$ al conjunto de puntos singulares de $F$.
\end{defn}
 Otro hecho importante de la teoría de ecuaciones diferenciales es que fuera de los lugares con puntos singulares, todas las ecuaciones diferenciales tienen el mismo comportamiento local.

\begin{Teorema}
\label{Teo:Rectificacion}
Todo ecuación diferencial definida por un campo vectorial holomorfo $F$ es, en una vecindad de un punto no singular, analíticamente equivalente a la ecuación diferencial determinada por el campo vectorial constante $\tilde{F}(x)=(1,0)$.
\end{Teorema}
\begin{proof}
  El flujo del campo vectorial $\tilde{F}$ es $\tilde{\varphi}((x,y,t))=(x+t,y)$. Observemos que para toda $(x,y)$, $\tilde{\varphi}((x,y,-x))\in\Pi':=\{x=0\}$. Sea $\Pi$ una transversal al vector $F((x_{0},y_{0}))$ en el punto $(x_{0},y_{0})$. Si $h\colon\Pi'\rightarrow\Pi$ es un isomorfismo lineal y $\varphi((x,y,t))$ es el flujo del campo vectorial $F$, entonces la función $H=\varphi(h[\tilde{\varphi}((x,y,-x))],x)$ lleva las soluciones del campo vectorial $\tilde{F}$ en las soluciones del campo vectorial $F$. Además, $H$ manda a la transversal $\Pi'$ en la transversal $\Pi$ y el vector $(1,0)$ en el vector $F((x_{0},y_{0}))\neq 0$, esto nos permite concluir que $H$ es invertible.   
\end{proof}
Al teorema anterior se le conoce como el teorema de rectificación. Las soluciones de la ecuación diferencial definida por el campo $F'=(1,0)$ son $x=x_{0}+t,y=y_{0}$, y como afirma el teorema, si no hay puntos singulares, localmente podemos escribir a las soluciones de esta manera; esto motiva la siguiente definición:

\begin{defn}
\label{Def:FoliacionEstandar}
La \emph{foliación estándar} del disco $B:=\{(x,y)\in\C^{2} \mid \left|x\right|<1, \left|y\right|<1 \}$ es la partición de $B$ por conjuntos $L_{y}=\{ \left|x\right|<1 \}\times\{y \} $ que llamaremos \emph{hojas de la foliación}, $\mathit{i.e.}$:
\begin{equation}
B=\bigsqcup_{\left|y\right|<1}L_{y}.
\end{equation}
\end{defn}

En la mayor parte de este trabajo trabajaremos en $\CP$, es por eso que necesitamos la siguiente definición:

\begin{defn}
\label{Def:FoliacionNoSingular}
Una \emph{foliación no singular} $\mathcal{F}$ en una dos variedad analítica $M$, es una partición de $M$ en hojas $L_{\alpha}$ de tal forma que todo punto $x\in M$ tiene una vecindad $B'$ y un biholomorfismo $H\colon B'\rightarrow B $ que manda las hojas locales $L_{\alpha}\cap B'$ de $\mathcal{F}$ en las hojas de la foliación estándar. A la pareja $(H,B')$ la llamaremos una \emph{carta distinguida} de la foliación $\mathcal{F}$.
\end{defn}

Es decir, una foliación no singular en una dos variedad analítica $M$, es una partición de $M$ que localmente se ve como la foliación estándar. El teorema (\ref{Teo:Rectificacion}) nos asegura que las soluciones de una ecuación diferencal definida por un campo vectorial holomorfo sin puntos singulares, generan una foliación no singular.\\

Como en los puntos que no son singulares una ecuación diferencial tiene un comportamiento local sencillo, el verdadero interés es estudiar el comportamiento de una ecuación diferencial en una vecindad de un punto singular y, eventualmente, hacer un análisis global del comportamiento de una ecuación cuando este sea posible.

\begin{defn}
Una \emph{foliación singular holomorfa} en una dos variedad analítica $M$ es una foliación no singular en $M\setminus\Sigma$ donde $\Sigma$ es un conjunto de puntos aislados al cual llamaremos el \emph{conjunto singular de la foliación}.
\end{defn}

Todo campo vectorial $F$ en $M$, con conjunto singular $\Sigma$, define, mediante las soluciones de la ecuación diferencial que determina,  una foliación no singular $\mathcal{F}$ en $M\setminus\Sigma$. Pero en principio, $\Sigma$ podría ser un conjunto muy grande (por ejemplo, $\Sigma$ podría ser toda una curva analítica) como para definir una foliación singular en M. El siguiente resultado afirma que para foliaciones en dos variedades analiticas, siempre es posible encontrar un conjunto de puntos aislados $\Sigma'\subset\Sigma$ y una foliación no singular $\tilde{\mathcal{F}}$ de $M\setminus\Sigma'$ de tal forma que las hojas de $\tilde{\mathcal{F}}$ coincidan con las hojas de $\mathcal{F}$ en $M\setminus\Sigma$.

\begin{Teorema}
\label{Teo:ExtensionFoliaciones}
Sea $F$ un campo vectorial definido en una dos variedad analítica $M$. Si $\Sigma$ es el conjunto singular de $F$, entonces existe un conjunto de puntos aislados $\Sigma'\subset\Sigma$ y una foliación no singular holomorfa $\tilde{\mathcal{F}}$ de $M\setminus\Sigma'$ cuya restricción a $M\setminus\Sigma$ coincide con la foliación generada por el campo vectorial $F$.
\end{Teorema}

El teorema anterior básicamente dice que en las foliaciones singulares generadas por campos vectoriales holomorfos siempre podemos suponer que el conjunto singular $\Sigma$, es un conjunto de puntos aislados. El resultado recíproco también es cierto.

\begin{Teorema}
\label{Teo:FoliacionesGeneradas}
Sea $\Sigma\subset M$ un conjunto aislado de puntos de la dos variedad analítica $M$, $\mathcal{F}$ una foliación no singular holomorfa en $\Sigma\setminus M$ que no se puede extender a ningún subconjunto de $\Sigma$.

Entonces, en una vecindad $U$ de cada punto $a\in\Sigma$, la foliación $\mathcal{F}$ está generada por un campo vectorial holomorfo $F$ con conjunto singular $\Sigma\cap U$.
\end{Teorema} 

Así como definimos equivalencia analítica entre campos vectoriales, ahora damos la definición correspondiente para foliaciones.

\begin{defn}
Decimos que dos foliaciones $\mathcal{F}$ y $\mathcal{F}'$ definidas en las variedades $M$ y $M'$ son \emph{analíticamente equivalentes} si existe un biholomorfismo $H\colon M\rightarrow M'$ que manda las hojas de $\mathcal{F}$ en las hojas de $\mathcal{F}'$ y el conjunto singular $\Sigma$ de $\mathcal{F}$ en el conjunto singular $\Sigma'$ de $\tilde{\mathcal{F}}$.
\end{defn}

El siguiente teorema relaciona campos vectoriales equivalentes con foliaciones equivalentes y foliaciones equivalentes con campos vectoriales ``casi'' equivalentes.

\begin{Teorema}
\label{Teo:CamposEquivalentes}
Sean $\dot{x=F(x)}$ y $\dot{x}=F'(x)$ dos ecuaciones diferenciales definidas por los campos vectoriales $F$ y $F'$ respectivamente. Si las ecuaciones diferenciales son analíticamente equivalentes, entonces las foliaciones que generan son analíticamente equivalentes.

Recíprocamente, si las foliaciones $\mathcal{F}$ y $\mathcal{F}'$ que generan los campos vectoriales $F$ y $F'$ respectivamente, son analíticamente equivalentes, entonces existe una función holomorfa $\rho$ que no se anula fuera del conjunto singular de $\mathcal{F}$ y cumple:
\begin{equation}
\rho(x)\Big( \frac{\partial H}{\partial x}\Big)F(x) = F'(H(x)).
\end{equation} 
\end{Teorema}

El teorema anterior nos dice que, cuando hablamos de foliaciones, ya no importa tanto el vector tangente (y por lo tanto la parametrización de la solución) si no todo el subespacio uno dimensional que es tangente a la hoja de la foliación (la multiplicación por la función $\rho$ es lo que nos permite concluir esto). Una manera de codificar estos subespacios tangentes es usando 1-formas.\\

A todo campo vectorial $F=(F_{1},F_{2})$ le podemos asociar la 1-forma $F_{1}\, dy - F_{2}\, dx $. Observemos que nuestro campo vectorial anula a esta 1-forma, pero como la 1-forma es lineal, también anula a todo el subespacio generado por el vector $(F_{1},F_{2})$. Así, el kernel de la 1-forma está formado por todos los subespacios tangentes a las soluciones de la ecuación diferencial determinada por el campo vectorial $F$.\\

Una de las ventajas de usar 1-formas es que si tenemos una función holomorfa $H$ entre dos variedades analíticas $M$ y $N$, podemos ``jalar'' cualquier 1-forma $\omega$ en $N$ a una 1-forma $H^{*}(\omega)$ en $M$. Esto lo podemos lograr usando la diferencial de $H$ para empujar vectores tangentes a $M$ en vectores tangentes a $N$ y después evaluarlos en $\omega$, $\mathit{i.e.}$:

\begin{equation}
H^{*}(\omega)(v):=\omega\Big(\Big(\frac{\partial H}{\partial x}\Big)v\Big),\ v\in TM.
\end{equation}

\section{Holonomía local.}

Una herramienta muy útil al estudiar una ecuación diferencial es el análisis del comportamiento de la dinámica transversal de sus soluciones. Esta dinámica la obtenemos estudiando el grupo de holonomía asociado a una hoja determinada de la foliación. A continuación veremos cómo asociar este grupo a una ecuación diferencial.\\

Una transversal a una hoja $L$ de una foliación $\mathcal{F}$ de $M$ en el punto $a$, es la imagen de una función holomorfa $\tau\colon (\C,0)\rightarrow (M,a)$ que corta de manera transversal a $L$.\\

Si en un punto $a\in L$ tomamos una carta distinguida $(H=(h_{1},h_{2}),B')$ del punto $a$ tal que $H(L)=L_{0}=\{(x,0)\}\subset B$ entonces, para todo punto $b\in L\cap B'$, si consideramos las transversales locales a $L$ en los puntos $a$ y $b$ definidas por $\tau:=H^{-1}(\,\{\,(h_{1}(a),y); \left|y\right|<1 \,\}\,)$ y $\tau':=H^{-1}(\,\{\,(h_{1}(b),y); \left|y\right|<1 \,\}\,)$, podemos definir una transformación entre las transversales $\tau$ y $\tau'$.

En efecto, si en la hoja $L$ tomamos un curva $\gamma\colon [0,1]\rightarrow L$ que una a los puntos $a$ y $b$ entonces, la curva imagen $\tilde{\gamma}:=H(\gamma)$ que une a los puntos $(h_{1}(a),0)$ y $(h_{1}(b,0)$ puede ser levantada a curvas $\tilde{\gamma_{\alpha}}$ en las hojas $L_{\alpha}$ de la foliación estándar para así, mediante el biholomorfismo $H^{-1}$, obtener curvas $\gamma_{\alpha}$ que unan a las transversales $\tau$ y $\tau'$.\\

Así, para un punto $x\in\tau$, existe una curva $\gamma_{\alpha}\colon[0,1]\rightarrow L_{\alpha}$ tal que $\gamma_{\alpha}(0)\in\tau$ y $\gamma_{\alpha}(1)\in\tau'$ y por lo tanto, podemos definir la  \emph{transformación de correspondencia} entre $\tau$  y $\tau'$ asociada a $\gamma$ como $\Delta^{\gamma}_{\tau,\tau'}(x)=\gamma_{\alpha}(1)$. Como las soluciones de una ecuación diferencial dependen de manera holomorfa de la condición inicial, la transformación de correspondencia es una función holomorfa.\\

Si tomamos un tercer punto $c\in\gamma([0,1])$ y una tercera transversal $\tau''$ a $L$ por el punto $c$, entonces se satisface la identidad:

\begin{equation}
\label{IdentidadCorrespondencia}
\Delta^{\gamma}_{\tau,\tau''}=\Delta^{\gamma}_{\tau',\tau''}\circ\Delta^{\gamma}_{\tau,\tau'},
\end{equation}

\noindent además, la transformación de correspondencia sólo depende de la clase de homotopía de $\gamma$. Estas dos observaciones nos van a permitir definir una transformación de correspondencia entre dos transversales a una hoja $L$ que sean unidas por un camino $\gamma$.\\  

Si tenemos una curva $\gamma\colon [0,1]\rightarrow L$ y dos transversales $\tau,\tau'$ a $L$ en los puntos $\gamma(0)$ y $\gamma(1)$ respectivamente, podemos cubrir a $\gamma([0,1])$ con un número finito de abiertos  distinguidos $U_{j}$ de tal forma que en cada $U_{j}$ la foliación es equivalente a la foliación estándar. En cada uno de estos abiertos $U_{j}$ podemos poner transversales auxiliares $\tau_{j}$ y obtener un mapeo de correspondencia entre transversales consecutivas. Si componemos todos estos mapeos, al final obtenemos una transformación $\Delta_{\gamma}$ entre las transversales $\tau$ y $\tau'$.\\

Si $\gamma^{-1}$ es la curva $\gamma$ recorrida en sentido opuesto tenemos que $\Delta_{\gamma^{-1}}\circ\Delta_{\gamma}=Id$ y entonces, la transformación $\Delta_{\gamma}$ es un biholomorfismo.\\

Como una transversal no es más que una imagen biholomorfa de una vecindad $U\subset\C$ podemos tomar cartas de cada una de ellas y  así, si los puntos $a,a'\in L$ por los cuales pasan las transversales $\tau$ y $\tau'$ son distintos, reparametrizando las transversales, siempre es posible encontrar coordenadas de tal forma que el mapeo de correspondencia asociado a un camino que una ambos puntos, se vea como la identidad, es por eso que vamos a considerar caminos cerrados y usaremos una sola transversal $\tau$.

\begin{defn}
Sea $a\in L$, $\tau$ una transversal a $L$ por el punto $a$ y $\gamma\in\pi_{1}(L,a)$ un camino cerrado. La \emph{transformación de holonomía} $\Delta_{\gamma}\colon (\tau,a)\rightarrow(\tau,a)$ es la transformación de correspondencia a lo largo del camino cerrado $\gamma$. Tomando una carta de $\tau$ alrededor del punto $a$, podemos pensar a la transformación de holonomía como un biholomorfismo de una vecindad $U\subset\C$ del origen en ella misma, es decir, $\Delta_{\gamma}\in \mathrm{Diff}(\C,0)$.
\end{defn} 

Con las construcciones anteriores, por cada elemento $\gamma$ del grupo fundamental de la hoja $L$ obtenemos un biholomorfismo $\Delta_{\gamma}\in\mathrm{Diff}(\C,0)$. Al grupo formado por todos estos biholomorfismos lo llamaremos grupo de holonomía de la foliación $\mathcal{F}$ a lo largo de la hoja $L$.

\begin{Ejemplo}
\label{Ej:HolonomiaLineal}
La ecuación diferencial lineal
\begin{equation}
\begin{aligned}
\dot{x} &= \lambda_{1}x\\
\dot{y} &= \lambda_{2}y
\end{aligned}
\end{equation}
tiene por solución a 
\begin{equation}
\begin{aligned}
x(t) &= c_{1}e^{\lambda_{1}t}\\
y(t) &= c_{2}e^{\lambda_{2}t},\ \  c_{1},c_{2}\in\C.
\end{aligned}
\end{equation}
Vamos a calcular el grupo de holonomía asociado a la hoja $L=\{y=0\}\setminus\{0\}$ y la transversal $\{x=1\}$.

Un generador del grupo fundamental de $L$ es el lazo $\gamma=\{\left|x\right| = 1\}$. Si consideramos el segmento de recta $\{t\tfrac{2\pi i}{\lambda_{1}} \mid t\in[0,1]\}\subset\C$, su imagen bajo la solución con condición inicial (1,0) coincide con el lazo $\gamma$, y su imagen bajo cualquier otra solución que tenga condición inicial en la recta $\{x=1\}$ termina de nuevo en esta transversal pues, $x(\tfrac{2\pi i}{\lambda_{1}})=c_{1}e^{2\pi i}=c_{1}$ y si la condición inicial está en la transversal, entonces $c_{1}=1$ y así, $x(\tfrac{2\pi i}{\lambda_{1}})=1$. Por lo tanto, el mapeo de holonomía es: 

\begin{equation}
y \mapsto e^{2\pi i \frac{\lambda_{2}}{\lambda_{1}}}y.
\end{equation}   
\end{Ejemplo}

\marginpar{referencia}

La transformación de holonomía se puede definir en cualquier transversal a una hoja de la foliación, podemos cambiar de transversal usando, otra vez, a las hojas de la foliación y así obtenemos un biholomorfismo entre las transversales. Por lo tanto, los grupos de holonomía asociados a ambas transversales son conjugados el uno del otro. 

\section{Holonomía global.}

La transformación de holonomía que se definió en el sección anterior es un concepto local, la transformación de holonomía fija un punto (la intersección de la transversal $\tau$ con la hoja  $L$ a la cual se le está calculando el grupo de holonomía) y la transformación sólo está definida en una vecindad de ese punto.\\

Sería agradable poder, en algunos casos, definir esta transformación de manera global, es decir, en toda la transversal $\tau$. Para definir la transformación de holonomía (local), usábamos las hojas de la foliación para movernos de una transveral $\tau$ a otra transversal $\tau'$, pero para poder hacer esto, necesitábamos que la transversal $\tau$ a la hoja $L$ siguiera cortando de manera transversal a hojas suficientemente cercanas a $L$. Por lo tanto, si queremos definir una transformación de holonomía global vamos a necesitar una curva analítica que sea transversal a todas las hojas de la foliación.\\

Además, vamos a necesitar que, cuando nos movamos de un punto $x$ a un punto $y$ en una misma hoja $L$, la transversal a $L$ en el punto $x$ sea analíticamente equivalente a la tansversal a $L$ en el punto $y$. Es por eso que hacemos la siguiente definición.

\begin{defn}
\label{Def:FoliacionTransversal}
Sea $E$ una dos variedad analítica y $\Pi\colon E\rightarrow B$ un haz fibrado con fibra $F$. Diremos que una foliación $\mathcal{F}$ de $E$ es \emph{transversal a las fibras de $(E,\Pi)$} si:
\begin{enumerate}
\item Para todo $p\in E$, la hoja $L_{p}$ de $\mathcal{F}$ corta de manera transversal a $F_{\Pi(p)}$.

\item Si $L$ es una hoja de $\mathcal{F}$ entonces $\Pi\colon L\rightarrow B$ es una aplicación cubriente.
\end{enumerate}
\end{defn}

Si tenemos que la foliación $\mathcal{F}$ es transversal a las fibras de $(E,\Pi)$, por cada elemento $[\gamma]$ del grupo fundamental $\pi_{1}(B,b)$ del espacio base $B$, podemos asociar un biholomorfismo de la fibra $\varphi_{[\gamma]}\colon F\rightarrow F$ de la siguiente manera.\\

Sea $\gamma\colon [0,1]\rightarrow B$ es una lazo tal que $\gamma(0)=\gamma(1)=b$ y sea $y\in\Pi^{-1}(b)=F$. Si $L_{y}$ es la hoja de $\mathcal{F}$ que pasa por el punto $y$, entonces, como $\Pi\colon L \rightarrow B $ es una aplicación cubriente, podemos levantar a $\gamma$ a una única curva $\tilde{\gamma}\colon [0,1] \rightarrow L_{y}$ tal que, $\tilde{\gamma}(0)=y$ y $\Pi\circ\tilde{\gamma}=\gamma$. Esto quiere decir que $\Pi(\tilde{\gamma}(1))=\gamma(1)=b$ y por lo tanto $\tilde{\gamma}(1)\in\Pi^{-1}(b)=F$.\\

Así, para un punto $y\in\Pi^{-1}(b)$ podemos definir $\varphi_{\gamma}(y)=\tilde{\gamma}(1)$. Como el punto final de $\tilde{\gamma}$ sólo depende de la clase de homotopía de $\gamma$, la asociación anterior asigna a cada elemento $[\gamma]\in\pi_{1}(B,b)$ una transformación $\varphi_{[\gamma]}\colon F\rightarrow F$.\\

Gracias a que las soluciones de una ecuación diferencial dependen de manera analítica de las condiciones iniciales (Teorema \ref{Teo:ExistenciaUnicidad}), $\varphi_{[\gamma]}$ es una transformación analítica y es un biholomorfismo ya que $\varphi_{[\gamma^{-1}]}$ es una inversa analítica de $\varphi_{\gamma}$.\\

\begin{defn}
\label{Def:HolonomiaGlobal}
Sea $\mathcal{F}$ una foliación transversal a las fibras de $(E,\Pi,B)$. Si $[\gamma]\in\pi_{1}(B,b)$ y $F=\Pi^{-1}(b)$, llamaremos \emph{transformación de holonomía global} asociada a $[\gamma]$, a la transformación $\varphi_{[\gamma]}\colon F\rightarrow F$ construida arriba. 
\end{defn}

Si tenemos una separatriz $L$ de la foliación $\mathcal{F}$, y $p\in\overline{L}$ es un punto singular de $\mathcal{F}$, podemos obtener la transformación de holonomía local asociada a $L$, al restringir, a una vecindad $V\subset F$ suficientemente pequeña de $L\cap F$, la transformación de holonomía global asociada a un lazo $\gamma$ que rodee a $\Pi(p)$. Esto se debe a que el lazo $\gamma$ se levanta a $L$ como un lazo $\tilde{\gamma}$ que rodea al punto singular $p$ en la separatriz $L$ (localmente, la separatriz es un disco sin un punto). Y la transformación de holonomía local, también puede construirse levantando un lazo de $L$ a curvas en hojas de $\mathcal{F}$ suficientemente cercanas a $L$. Es precisamente la imposibilidad de levantar lazos de $L$ a cualquier hoja de $\mathcal{F}$ lo que hace a la holonomía de la sección anterior, una transformación local.\\

Lo anterior nos permite heredar algunas propiedades de la transformación de holonomía local a la transformación de holonomía global. Por ejemplo, si la transformación de holonomía local  $f$ tiene orden finito ($\mathit{i.e.},\, f^{n}=Id\,$ para algún $n\in\mathbb{N}$), entonces la transformación de holonomía global que coincide en un abierto con $f$, también tiene orden finito y los órdenes de ambas coinciden.\\

\marginpar{prueba caso particular}

Más adelante necesitaremos el siguiente resultado,

\begin{Lema}
\label{Lema:HazTopologico}
Sea $V$ una dos variedad analitica y $H\colon V \rightarrow\Omega\subset\C$ una función analítica sin puntos críticos tal que existe una foliación no singular $\mathcal{F}$ que corta de manera transversal a todas las curvas de nivel de $H$ y, para cualesquiera $a,b\in \Omega$, $H^{-1}(a)$ es compacta y topológicamente equivalente a $H^{-1}(b)$. Entonces, $(V,H,\Omega)$ es un haz topológico.
\end{Lema}
\begin{proof}
Para exhibir a las trivializaciones locales alrededor de un punto $a\in \Omega$ vamos a usar el flujo de el campo vectorial que genera a la foliación $\mathcal{F}$ en una vecindad de cada punto $z\in H^{-1}(a)$.\\
Alrededor de este punto $z$, existe una vecindad $\tilde{W_{z}}\subset V$, un disco $D_{\epsilon}$ y una función $g\colon \tilde{W_{z}}\times D_{\epsilon}$ tal que para $z_{0}\in W_{z}:=\tilde{W_{z}\cap H^{-1}(a)}$, $g(z_{0},t)$ es la hoja de la foliación $\mathcal{F}$ que pasa por $z_{0}$.\\
Sea $U_{z}=H(\tilde{W_{z}})$ y consideremos la función $H\circ g\colon W_{z}\times D_{\epsilon}\subset H^{-1}(a)\times D_{\epsilon}\rightarrow U_{z}$. Como  para toda $p\in\Omega$, $H^{-1}(p)$ corta de manera transversal a todas las hojas de la foliación $\mathcal{F}$ tenemos que,

$$\frac{\partial (H\circ g)}{\partial t}=\nabla H\frac{\partial g}{\partial t}\neq 0.$$

Así, dado $b\in U_{z}$ existe una función analítica $\varphi_{z}(b,\_)\colon W_{z}\subset H^{-1}(a)\rightarrow H^{-1}(b)$ (haciendo un abuso de notación) tal que alrededor de $z\in W_{z}$, $H\circ g(x,\varphi_{z}(b,x))=b$.\\

Si para cada $z\in H^{-1}(a)$ hacemos la construcción anterior, como $H^{-1}(a)$ es compacto, podemos encontrar una cubierta finita $W_{1},\ldots,W_{n}$ de $H^{-1}(a)$ tal que si $b\in U:=\bigcap_{i=1}^{n} H(W_{i})$, entonces existe una función analítica e inyectiva $\varphi(b,\_)\colon H^{-1}(a)\rightarrow H^{-1}(b)$. En efecto, para $x\in H^{-1}(a)$ se tiene que $x\in W_{k}$ para alguna $k$ y así podemos definir $\varphi(b,x)=\varphi_{k}(b,x)$. La función anterior está bien definida pues si $x\in W_{i}\cap W_{k},\, i\neq k$ el teorema de existencia unicidad nos asegura que las funciones $\varphi_{i}(b,\_)$ y $\varphi_{k}(b,\_)$ coinciden en el conjunto $W_{i}\cap W_{k}$ y por lo mismo, tenemos que $\varphi(b,\_)$ es inyectiva. Como $H^{-1}(a)$ es compacta, la función analítica $\varphi(b\_)$ también es suprayectiva y es por lo tanto, un biholomorfismo.\\


Como todas las curvas de nivel son homeomorfas a una superficie $F$, para cada $a\in\Omega$ hemos encontrado una vecindad $U\subset\Omega$ y un biholomorfismo $\varphi\colon U\times F\rightarrow H^{-1}(U)$ que manda a los puntos $(b,x)\in U\times F$ en el conjunto $H^{-1}(b)$. Esto quiere decir que $\varphi^{-1}$ es una trivialización local de $(V,H,\Omega)$.\\
\end{proof}

\noindent Por último, observemos que si fijamos $x_{0}\in H^{-1}(a)$ y $L_{x_{0}}$ es la hoja de $\mathcal{F}$ que pasa por $x_{0}$, entonces $\varphi(t,x_{0})\subset L_{x_{0}}\cap H^{-1}(U)$. Esto quiere decir que,
$$\varphi^{-1}(L_{x_{0}}\cap H^{-1}(U))=\bigsqcup_{x\in L_{x_{0}}\cap H^{-1}(U)}U\times \{x\}.$$ Como de manera local tenemos que $H=\Pi_{1}\circ \varphi^{-1}$ entonces, si $L$ es una hoja de la foliación $\mathcal{F}$, $H\colon L\rightarrow \Omega$ es una aplicación cubriente.

\section{El plano proyectico complejo $\C\mathbb{P}^{2}$.}

El espacio proyectivo complejo $\C\mathbb{P}^{n}$ es el conjunto de todos los subespacios uno dimensionales de $\C^{n+1}\setminus\{0\}$. Si en este conjunto consideramos la relación de equivalencia $x\sim y$ sí y sólo sí $x=\lambda y, \lambda\in\C\setminus\{0\}$, entonces $\C\mathbb{P}^{n} = \C^{n+1}\setminus\{0\} / \sim$.\\

En el caso $n=2$ obtenemos el plano proyectivo complejo, y en el caso $n=1$ obtenemos la recta proyectiva compleja que es nada más y nada menos que una esfera.\\

Denotaremos por $[x_{0}:y_{0}:z_{0}]$ a la clase de $(x_{0},y_{0},z_{0})$ $\mathit{i.e.}$ $[x_{0}:y_{0}:z_{0}]:=\{(x,y,z)\in\C^{3}\setminus\{0\} \mid \exists\lambda\in\C,(x_{0},y_{0},z_{0})=\lambda (x,y,z)\}$. A esta manera de denotar los puntos de $\CP$ se le conoce como coordendas homogéneas.\\

Con la topología cociente, $\CP$ tiene una estructura natural de  dos variedad analítica dada por las siguientes cartas: los abiertos que usaremos son $U_{x}=\{[x:y:z] \mid x\neq 0\},U_{y}=\{[x:y:z] \mid y\neq 0\},U_{z}=\{[x:y:z] \mid z\neq 0\}$ y los homemorfismos son:

$$
\begin{matrix}
\psi_{x} \colon & U_{x} &  \rightarrow & \C^{2}\\
& [x:y:z] &  \mapsto &  (\frac{y}{x},\frac{z}{x})
\end{matrix}
$$

$$
\begin{matrix}
\psi_{y} \colon & U_{y} &  \rightarrow & \C^{2}\\
& [x:y:z] &  \mapsto &  (\frac{x}{y},\frac{z}{y})
\end{matrix}
$$

$$
\begin{matrix}
\psi_{z} \colon & U_{z} &  \rightarrow & \C^{2}\\
& [x:y:z] &  \mapsto &  (\frac{x}{z},\frac{y}{z})
\end{matrix}
$$\\

Podemos pensar al mapeo $\psi_{z}$ como si a cada punto $[x:y:z]$ de $U_{z}\subset\CP$ lo mandáramos al representante $(\tfrac{x}{z},\tfrac{y}{z},1)$; a esta carta sólo le falta cubrir a las clases $[x:y:0]$. Si nos quedamos solamente con las primeras dos coordenadas $[x:y]$, estos puntos forman un $\C\mathbb{P}^{1}$, que como ya hemos mencionado es una esfera.\\

Así, podemos pensar a $\CP$ como un plano complejo $\C^{2}$ al cual, le hemos pegado una recta proyectiva, la recta al infinito.\\ 
 
La mayoría del tiempo trabajaremos en las coordenadas $\psi_{z}(U_{z})=\C^{2}$, pero a veces necesitaremos ir a las otras dos cartas para ver el comportamiento de algunas cosas que pasan en la recta al infinito.\\

Los cambios de coordenadas de $\psi_{x}(U_{x})$ a $\psi_{z}(U_{z})$ y de $\psi_{y}(U_{y})$ a $\psi_{z}(U_{z})$ están dados por:

\begin{equation} 
%\tag{$\dagger$}
\label{CambiosDeCoordenadas}
\begin{aligned}
\psi_{z} \circ \psi^{-1}_{x}(u,v) = (\frac{1}{u},\frac{v}{u})\\
\psi_{z} \circ \psi^{-1}_{y}(u,v) = (\frac{u}{v},\frac{1}{v}).
\end{aligned}
\end{equation}


\section{El teorema de linealización de Poinacaré.}
\label{sec:LinenalizacionDePoincare}
Como mencionamos anteriormente, un problema fundamental de la teoría de ecuaciones diferenciales es simplificar, mediante un cambio de coordenadas, un campo vectorial $F$. Si este campo vectorial tiene parte lineal, una pregunta natural es cuándo este campo vectorial es analíticamente equivalente a su parte lineal. El teorema de linealización de Poincaré nos dice bajo qué condiciones un campo vectorial es analíticamente equivalente a su parte lineal.\\

Sea $F(x)=Ax+V_{2}(x)+\cdots+V_{m}(x)+\cdots$ donde $A=(\tfrac{\partial F}{\partial x})(0)$ es la parte lineal del campo vectorial en el origen y los $V_{i}$ son campos vectoriales homogéneos de grado $i$. Una manera de atacar el problema de linealización es intentar eliminar el término $V_{2}$ mediante un biholomorfismo $H_{2}$. Una vez logrado ésto procedemos a eliminar el término $V_{3}$ con un biholomorfismo $H_{3}$ y así sucesivamente.\\

Supongamos que tenemos un campo vectorial  $F=(F_{1},F_{2})$ del siguiente estilo:

\begin{equation}
\label{CampoVectInicial}
\begin{aligned}
F_{1} &= \lambda_{1}x +ax^{\alpha}y^{\beta}+O(||(x,y)||^{\alpha+\beta})\\
F_{2} &= \lambda_{2}y + O(||(x,y)||^{\alpha+\beta}).
\end{aligned}
\end{equation}

\noindent El objetivo es encontrar un biholomorfimso $H\colon (\C^{2},0)\rightarrow(\C^{2},0)$ que conjugue este campo vectorial con un campo vectorial $\tilde{F}=(\tilde{F}_{1},\tilde{F}_{2})$ de la forma:

\begin{equation}
\label{CampoVectFinal}
\begin{aligned}
\tilde{F}_{1} &= \lambda_{1}x +O(||(x,y)||^{\alpha+\beta})\\
\tilde{F}_{2} &= \lambda_{2}y + O(||(x,y)||^{\alpha+\beta}).
\end{aligned}
\end{equation}

\noindent Para lograr esto, el biholomorfismo $H$ y los campos vectoriales tienen que statisfacer la igualdad:
\begin{equation}
\label{NuevaEquivalenciaAnalitica}
\Big(\frac{\partial H}{\partial x}\Big)F(x)=\tilde{F}(H(x)).
\end{equation}


\noindent Si hacemos $H(x,y)=(x,y)+c(x^{\alpha}y^{\beta},0)$ el lado izquierdo de la ecuación (\ref{NuevaEquivalenciaAnalitica}) es:

\begin{equation*}
(Id+c\begin{pmatrix} \alpha x^{\alpha -1}y^{\beta} & \beta x^{\alpha}y^{\beta -1} \\ 0 & 0 \end{pmatrix})\begin{pmatrix}\lambda_{1}x +ax^{\alpha}y^{\beta}+O(||(x,y)||^{\alpha+\beta})\\ \lambda_{2}y \end{pmatrix}.
\end{equation*} 

\noindent La primera entrada de este vector es,

\begin{equation*}
\lambda_{1}x +ax^{\alpha}y^{\beta}+cx^{\alpha}y^{\beta}(\alpha\lambda_{1} +\beta\lambda_{2}) +O(||(x,y)||^{\alpha+\beta}).
\end{equation*} 
%\begin{pmatrix}\lambda_{1}(x+p(x,y)) + O(||(x,y)||^{\alpha+\beta}) \\ \lambda_{2}y\end{pmatrix}.

\noindent La primera coordenada del campo vectorial $\tilde{F}(H(x,y))=\tilde{F}((x,y)+c(x^{\alpha}y^{\beta},0))$ es,

\begin{equation*}
\lambda_{1}(x+cx^{\alpha}y^{\beta}) + O(||(x,y)||^{\alpha+\beta}).
\end{equation*}

\noindent Si queremos que el bihlomorfismo $H$ conjugue a los campos vectoriales $F$ y $\tilde{F}$, los monomios de grado uno y de grado $\alpha + \beta$ de ambos campos vectoriales deben ser iguales. Así,

\begin{equation}
a+c(\alpha\lambda_{1}+\beta\lambda_{2})=\lambda_{1}c.
\end{equation}

En caso de que $\lambda_{1}-\alpha\lambda_{1}+\beta\lambda_{2}\neq 0$ podemos despejar a $c$ y obtener:

%\begin{equation*}
%\lambda_{1}x +(a+c(\alpha\lambda_{1} +\beta\lambda_{2}))x^{\alpha}y^{\beta} +O(||(x,y)||^{\alpha+\beta +1}) = \lambda_{1}x + \lambda_{1}cx^{\alpha}y^{\beta}+O(||(x,y)||^{\alpha+\beta +1}).
%\end{equation*}

\begin{equation}
c = \frac{a}{\lambda_{1} -(\alpha\lambda_{1} + \beta\lambda_{2})}.
\end{equation}

Con nuestra elección de $H$, es claro que la segunda coordenada de ambos vectores de la ecuación (\ref{NuevaEquivalenciaAnalitica}). Por lo tanto, si $\lambda_{1}-\alpha\lambda_{1}+\beta\lambda_{2}\neq 0$, existe un biholomorfismo $H$ que conjuga a los campos vectoriales $F$ y $\tilde{F}$.
\begin{defn}
\label{Def:Resonancias}
Diremos que la pareja $\lambda=(\lambda_{1},\lambda_{2})\in\C^{2}$ es \emph{resonante} si existen números naturales $\alpha_{1},\alpha_{2}$, $\alpha_{1}+\alpha_{2}>1$ tales que:
\begin{equation}
\label{Denominadores}
\lambda_{j}=\alpha_{1}\lambda_{1}+\alpha_{2}\lambda_{2}.
\end{equation}
\end{defn}

En el cáclulo anterior si remplazamos el campo vectorial $F$ por un campo vectorial de la forma,

\begin{equation}
\begin{aligned}
F_{1} &= \lambda_{1}x +V_{\alpha+\beta}(x,y)+O(||(x,y)||^{\alpha+\beta})\\
F_{2} &= \lambda_{2}y + O(||(x,y)||^{\alpha+\beta}),
\end{aligned}
\end{equation}

\noindent donde $V_{\alpha+\beta}(x,y)$ es un polinomio homogéneo de grado $\alpha+\beta$ en el cual aparece el monomio $ax^{\alpha}y^{\beta}$, no es difícil convencerse de que $H(x,y)=(x,y)+(cx^{\alpha}y^{\beta},0)$ conjuga a este campo vectorial con:
 
\begin{equation}
\begin{aligned}
\tilde{F}_{1} &= \lambda_{1}x + V_{\alpha+\beta}-ax^{\alpha}y^{\beta}+O(||(x,y)||^{\alpha+\beta})\\
\tilde{F}_{2} &= \lambda_{2}y + O(||(x,y)||^{\alpha+\beta}).
\end{aligned}
\end{equation}

Una vez eliminado el monomio $ax^{\alpha}y^{\beta}$, si la pareja $(\lambda_{1},\lambda_{2})$ es no resonante, podemos eliminar los demás monomios de grado $\alpha+\beta$ en la primera coordenada. Si en la segunda coordenada también hay monomios de grado $\alpha+\beta$, con un procedimiento similar podemos eliminarlos. \\ 
Si los valores propios de $A$ son no resonantes entonces podemos eliminar cualquier término no lineal de $F$ mediante un biholomorfismo pero, si queremos eliminar todos los términos no lineales de un jalón, tendremos que componer una infinidad de biholomorfismos y el resultado no siempre será un biholomorfismo sino simplemente una serie de potencias formal. Esto se debe a que los números $\lambda_{j}-\alpha_{1}\lambda_{1}-\alpha_{2}\lambda_{2}$ pueden ser muy pequeños y en consecuencia, al dividir por ellos, podemos afectar la convergecia de la serie de potencias que linealiza (de manera formal) al campo vectorial. Es por eso que si queremos que la serie de potencias converja, tenemos que pedir algo adicional a los valores propios de $A$.

\begin{defn}
\label{Def:DominioPoincare}
Sean $\lambda_{1},\lambda_{2}\in\C$, si la cerradura convexa del conjunto $\{ \lambda_{1},\lambda_{2}\}\subset\C$ no contiene al origen, diremos que $\lambda=(\lambda_{1},\lambda_{2})$ está en el \emph{dominio de Poincaré}. En caso contrario, diremos que $\lambda$ está en el \emph{dominio de Siegel}.
\end{defn}

Si pensamos a los número complejos $\lambda_{1},\lambda_{2}$ como vectores en $\mathbb{R}^{2}$, estar en el dominio de Siegel se traduce a que $\lambda_{1}$ y $\lambda_{2}$ sean linealmente dependientes y de sentidos opuestos.\\

Con las dos definiciones anteriores podemos enunciar el teorema de linealización de Poincaré.
\marginpar{referencia}
\begin{Teorema}
Si los valores propios de la parte lineal del campo vectorial $F$ son no resonantes y están en el dominio de Poincaré, entonces $F$ es analíticamente equivalente a su parte lineal.
\end{Teorema}

\begin{Ejemplo}
\label{Ej:LinealizacionRadial}
Si los valores propios de la parte lineal de un campo vectorial $F$ son iguales y distintos de cero, entonces el campo vectorial $F$ es analíticamente equivalente a su parte lineal. Esto se debe a que si $\lambda_{1}=\lambda_{2}$ entonces la ecuación (\ref{Denominadores}) sólo tiene solución con $\alpha_{1}=1$ y $\alpha_{2}=0$ y en consecuencia la pareja $(\lambda_{1},\lambda_{2})$ es no resonante. Además como los valores propios coinciden, su cerradura convexa es un único punto $\lambda_{1}\neq 0$. 
\end{Ejemplo}

\section{Explosión de singularidades.}  

En la sección anterior enunciamos bajo qué condiciones un campo vectorial, con parte lineal, es analíticamente equivalente a su parte lineal pero, ¿y si nuestro campo vectorial no tiene parte lineal? Cuando esto sucede, una técnica muy utilizada es la explosión de singularidades. A continuación describimos este proceso.\\

Intuitivamente, si un campo vectorial no tiene parte lineal en el origen, el campo vectorial aplasta todos los subespacios 1-dimensionales que salen del origen. Es por eso que vamos a intentar ``separar'' a estos subespacios 1-dimensionales.\\

Consideremos la proyección canónica de $p\colon\C^{2}\setminus\{0\}\rightarrow\C\mathbb{P}^{1}$. Esta proyección asocia a cada punto $(x,y)\in\C^{2}\setminus\{0\}$ su clase de equivalencia $[x:y]\in\C\mathbb{P}^{1}$, o lo que es lo mismo, $p$ envía a todo un subespacio 1-dimensional menos el origen en un solo punto de $\C\mathbb{P}^{1}$.\\

La gráfica de esta función $graf(p)$ es un subconjunto del producto $\C^{2}\times\C\mathbb{P}^{1}$ y está determinada por los puntos de la forma $((x,y),[x:y])$. En esta gráfica, todos los subespacios 1-dimensionales de $\C^{2}$ ya están ``separados'' pues a subespacios 1-dimensional distintos de $\C^{2}$ les corresponden puntos distintos en $\C\mathbb{P}^{1}$ .\\

La proyección $\Pi\colon\C^{2}\times\C\mathbb{P}^{1}\rightarrow\C^{2}$ es una función biyectiva entre $graf(p)$ y $\C^{2}\setminus\{0\}$ y además la imagen inversa del origen es \emph{el divisor excepcional} $E:=\Pi^{-1}(0)=\{0\}\times\C\mathbb{P}^{1}\simeq\C\mathbb{P}^{1}$. Si con una transformación queremos alterar la parte lineal de un campo vectorial, no podemos utilizar transformaciones analíticas invertibles, ya que éstas preservan la parte lineal del campo vectorial.

Observemos que si $M:=graf(p)\cup E$, entonces, $\Pi\colon M\rightarrow\C^{2}$ es una función biyectiva entre $M\setminus E$ y $\C^{2}\setminus\{0\}$, la imagen inversa del origen es toda una curva y además, en $M$ ya están ``separados'' los subespacios 1-dimensionales de $\C^{2}$.\\

Por todo lo anterior, la pareja $(M,\Pi)$ es un buen candidato para desingularizar un campo vectorial. Pero todo el proceso de desingularización lo debemos llevar acabo de manera analítica, es por eso que debemos darle a $M$ una estructura de variedad analítica y mostrar que con esa estructura, $\Pi$ es una función analítica.\\   

Si usamos la carta $u=\tfrac{w}{z}$ de $\C\mathbb{P}^{1}$, $M$ queda descrita por los puntos que satisfacen $u=\tfrac{y}{x}\Rightarrow y=ux$ (esta última ecuación también inculye a los puntos de $E$).

En estas coordenadas la función $\varphi((x,y),[z:w])=(x,\tfrac{w}{z})=(x,u)$ restringida a $M$ es una carta de $M$. Como en los puntos de $M$ se cumple que $y=ux$, la función inversa de esta carta es $\varphi^{-1}(x,u)=((x,ux),u)=((x,ux),[z:w])$.\\

De manera análoga, en la otra carta de $\C\mathbb{P}^{1}$, $v=\tfrac{x}{y}$, $M$ queda descrita por $x=vy$ y la función $\phi((x,y),[z:w])=(y,\tfrac{w}{x})=(y,v)$ es otra carta de $M$ cuya inversa es $\phi^{-1}(y,v)=((vy,y),v)$\\

Como las cartas $((x,y),u)$ y $((x,y),v)$ cubren a todo $\C^{2}\times\C\mathbb{P}^{1}$, las cartas $(x,u)$ y $(y,v)$ cubren a $M$ y el cambio de coordenadas de $(x,u)$ a $(y,v)$ está dado por:

\begin{equation}
\label{CambiosCoordenadasExplosion}
\phi\circ\varphi^{-1}(x,u)=(ux,\tfrac{1}{u}).
\end{equation}

Las dos cartas anteriores hacen de $M$ una dos variedad analítica y en la carta $(x,u)$ la proyección $\Pi\colon M\rightarrow\C^{2}$ adquiere la forma $\Pi(x,u)=(x,ux)$ ya que en esta carta $y=ux$. Mientras que en la otra carta, $\Pi(y,v)=(vy,y)$.\\

Lo anterior prueba que, con la estructura analítica que le hemos dado a $M$, $\Pi$ es una función analítica y entonces podemos usarla para jalar 1-formas en $\C^{2}$ (y en consecuencia foliaciones y campos vectoriales) a $M$.\\

Si tenemos una 1-forma $\omega$ con punto singular aislado en el origen, esta 1-forma define una foliación no singular $\mathcal{F}$ en $\C^{2}\setminus\{0\}$. Así, $\Pi^{*}(\omega)$ define una foliación no singular $\Pi^{*}(\mathcal{F})$ en $M\setminus E$. Pero gracias al Teorema (\ref{Teo:ExtensionFoliaciones}) podemos extender a $\Pi^{*}(\mathcal{F})$ a todo $M$ como una foliación singular con puntos singulares aislados en el divisor excepcional $E$. Al proceso anterior se le conoce como \emph{explosión de la 1-forma} $\omega$ ó \emph{explosión de la foliación} $\mathcal{F}$. A esta técnica de explosión de singularidades también se le conoce como \emph{desingularización}.

\begin{Ejemplo}
\label{Ej:BlowUpLineal}
Haremos el blow-up de la 1-forma $\omega=x\, dy-\lambda y\, dx,\lambda\in\C$ en la carta $(x,u)$.\\

$\Pi^{*}(\omega)=x\, d(ux)-\lambda ux\, dx=x(u\, dx +x\, du)-\lambda ux\, dx=x(x\, du-(\lambda-1)u\, dx)$.\\

$\Pi^{*}(\omega)$ define una foliación no singular en $M\setminus E$ y como en esta carta $E=\{x=0\}$, la función $x$ no se anula en $M\setminus E$. Así, podemos multiplicar a la 1-forma $\Pi^{*}(\omega)$ por $x$ para obtener una nueva 1-forma $\omega_{1}=x\, du-(\lambda-1)u\, dx$ que define la misma foliación que $\Pi^{*}(\omega)$ en $M\setminus E$ pero tiene singularidades aisladas en el divisor excepcional $E$.
\end{Ejemplo}

%%La 1-forma que explotamos en el ejemplo anterior se corresponde con el campo vectorial:

%\begin{equation}
%\begin{aligned}
%\dot{x}&=x\\
%\dot{y}&=\lambda y.
%\end{aligned}
%\end{equation}

%Y el campo vectorial que obtenemos en la carta $(u,x)$ al explotar es:

%\begin{equation}
%\begin{aligned}
%\dot{x}&=x\\
%\dot{u}&=(\lambda-1) u.
%\end{aligned}
%\end{equation}

Un caso particular del ejemplo anterior que utilizaremos más adelante es cuando $\lambda=1$. En este caso $\Pi^{*}(\omega)=x^{2}du$, pero en $M\setminus E$ la función $x^{2}$ tampoco se anula y por lo tanto podemos multiplicar por ella para así obtener la 1-forma $\omega_{1}=du$.

Esta 1-forma no tiene puntos singulares y las hojas de la foliación que determina quedan descritas por $u=cte$. Usando la otra carta $(y,v),\ v=\tfrac{x}{y}$ podemos ver que, en la parte de $M$ que la carta $(x,u)$ no nos permite ver, tampoco hay puntos singulares. Es decir, al explotar la foliación generada por $\omega=x\, dy-y\, dx$ obtenemos una foliación en $M$ sin puntos singulares.\\

En este caso particular, $\lambda=1$, el campo vectorial que se corresponde con la 1-forma $\omega=x\, dy-y\, dx$ es el campo vectorial radial:

\begin{equation}
\begin{aligned}
\dot{x}&=x\\
\dot{y}&=y.
\end{aligned}
\end{equation}

Una observación muy importante es que todo el proceso de desingularización puede llevarse acabo de manera local (sólo hay que restringir toda la construcción a una vecindad del origen $(\C^{2},0)$). Si tenemos una foliación $\mathcal{F}$ con un número finito de singularidades $\Sigma$ en una dos variedad analítica $M$, la observación anterior nos permite hacer una explosión local en cada uno de los puntos singulares para así obtener una nueva variedad $M'$ y una función holomorfa $\Pi\colon M' \rightarrow M$ que satisface las siguientes propiedades:

\begin{enumerate}
\item Si $p\in\Sigma$ entonces $\Pi^{-1}(p):=E_{p}\simeq\C\mathbb{P}^{1}$.
\item $\Pi$ es un biholomorfimso entre $M'\setminus \bigcup_{p\in\Sigma} E_{p}$ y $M\setminus\Sigma$.
\end{enumerate}

La variedad $M'$ se puede construir explotando un punto $p_{1}\in\Sigma$, obteniendo así una variedad $M_{1}$, un divisor $E_{p_{1}}$ y un mapeo $\Pi_{1}\colon M_{1}\rightarrow M$. Como $\Pi_{1}$ es un biholomorfismo entre $M_{1}\setminus E_{p_{1}}$ y $M\setminus \{p_{1}\}$, alguna vecindad de otro punto $p_{2}\in\Sigma\setminus \{p_{1}\}$ se mapea de manera biholomorfa a la nueva variedad $M_{1}$. Así, podemos aplicar el procedimiento anterior al punto $\Pi^{-1}_{1}(p_{2})$ para obtener otra variedad $M_{2}$, otro divisor $E_{p_{2}}$ y otro mapeo $\Pi_{2}\colon M_{2}\rightarrow M_{1}$ que es un biholomorfismo entre $M_{2}\setminus E_{p_{2}}$ y $M_{1}\setminus \{\Pi^{-1}_{1}(p_{2})\}$.\\ 

Si repetimos este procedimiento con todos los puntos restantes de $\Sigma$, como $\Sigma$ es un conjunto finito, al final obtendremos una variedad $M_{n}$, un divisor $E_{p_{n}}$ y un mapeo $\Pi_{n}\colon M_{n}\rightarrow M_{n-1}$ que es un biholomorfismo entre $M_{n}\setminus E_{p_{n}}$ y $M_{n-1}\setminus \{\Pi^{-1}_{n-1}(p_{n})\}$. Si llamamos $M'$ a $M_{n}$ y $\Pi$ a $\Pi_{1}\circ\cdots\circ\Pi_{n}$, la variedad $M'$ y el mapeo $\Pi$ satisfacen las propiedades deseadas.   
 
\section{El grado de una foliación dicrítica en $\C\mathbb{P}^{2}$.}

Dada una foliación holomorfa del plano proyectivo complejo $\CP$ se tiene que, como consecuencia del teorema de Chow (el cual afirma que todo subconjunto analítico de una variedad proyectiva es algebraico, ver \cite{Mumford}) , ésta es generada, en cualquier carta afín, por un campo vectorial polinomial \cite[p.~477]{IlyaYako}.\\

Si en la carta afín $(x,y)$, la foliación está generada por el campo vectorial:

\begin{equation}
\label{EcuacionEnCP2}
\begin{aligned}
\dot{x} &=p_{1}(x,y)+\cdots+p_{n}(x,y)\\
\dot{y} &=q_{1}(x,y)+\cdots+q_{n}(x,y),
\end{aligned}
\end{equation}

\noindent donde los $p_{k},q_{k}$ son polinomios homogéneos de grado $k$, entonces, para obtener un campo vectorial que genere a la foliación en una vecindad de la recta al infinito, consideramos el cambio de coordenadas $x=\tfrac{1}{u}$ y $y=\tfrac{v}{u}$. En estas coordenadas el campo vectorial se expresa como:

\begin{equation}
\begin{aligned}
\dot{u} &=\frac{1}{u^{n+1}}p_{n}(1,v)+\cdots+\frac{1}{u^{2}}p_{1}(1,v)\\
\dot{v} &=\frac{1}{u_{n+2}}(vp_{n}(1,v)-q_{n}(1,v))+\cdots+\frac{1}{u^{3}}(vp_{1}(1,v)-q_{1}(1,v)).
\end{aligned}
\end{equation}

Si llamamos $h_{k+1}=yp_{k}(x,y)-xq_{k}(x,y)$ podemos escribir la ecuación anterior como:

\begin{equation}
\label{EcuacionEnCP2Infinito}
\begin{aligned}
\dot{u} &=\frac{1}{u^{n+1}}p_{n}(1,v)+\cdots+\frac{1}{u^{2}}p_{1}(1,v)\\
\dot{v} &=\frac{1}{u_{n+2}}h_{n+1}(1,v)+\cdots+\frac{1}{u^{3}}h_{2}(1,v).
\end{aligned}
\end{equation}

En esta carta, la recta al infinito queda descrita por $\{u=0\}$ y entonces, antes de obtener la expresión final del campo vectorial cerca del infinito, podemos multiplicar por una potencia de $u$ adecuada para eliminar los polos. Así, tenemos dos casos distintos:

\begin{enumerate}

\item Si $h_{n+1}\neq 0$, entonces diremos que la foliación es \emph{no dicrítica} y así, podemos multiplicar por $u^{n+2}$ para obtener el campo vectorial:
\begin{equation}
\label{EcuacionNoDicritica}
\begin{aligned}
\dot{u} &=up_{n}(1,v)+\cdots+u^{n}p_{1}(1,v)\\
\dot{v} &=h_{n+1}(1,v)+uh_{n}(1,v)+\cdots+u^{n-1}h_{2}(1,v).
\end{aligned}
\end{equation}

Observemos que en este caso, al hacer el cambio de coordenadas, el grado de los polinomios que definen la foliación en la nueva carta es uno más que el grado de los polinomios que definen la foliación en la carta inicial. Además, la recta al infinito $\{u=0\}$ es invariante y tiene singularidades en los puntos $(0,v_{j})$ donde $v_{j}$ es una raíz del polinomio $h_{n+1}(1,v)$.

\item Si $h_{n+1}\equiv 0$, diremos que la foliación es \emph{dicrítica} y entonces basta multiplicar por $u^{n+1}$ para obtener:

\begin{equation}
\label{EcuacionDicriticaInfinito}
\begin{aligned}
\dot{u} &=p_{n}(1,v)+\cdots+u^{n-1}p_{1}(1,v)\\
\dot{v} &=h_{n}(1,v)+\cdots+u^{n-2}h_{2}(1,v).
\end{aligned}
\end{equation}

En este caso volvemos a obtener un campo vectorial polinomial del mismo grado que el campo vectorial original pero ahora la recta al infinito ya no es invariante. En efecto, la foliación corta de manera transversal a la recta al infinito salvo en las raíces del polinomio $p_{n}(1,v)$, en estos puntos tenemos tangencias que se vuelven puntos singulares si $p_{n}(1,v)$ y $h_{n}(1,v)$ tienen raíces en común.
\end{enumerate}

En el caso dícritico, el polinomio $h_{n+1}$ se anula y esto se traduce a que

\begin{equation}
\label{CondicionDicritica}
yp_{n}(x,y)=xq_{n}(x,y).
\end{equation}

Si evaluamos esta expresión en los puntos $(1,y)$ obtenemos $yp_{n}(1,y)=q_{n}(1,y)$ y entonces, el polinomio $p_{n}(1,y)$ es de grado estrictamente menor a $n$ o lo que es lo mismo, $p_{n}(x,y)$ no tiene monomios de la forma $ay^{n}$. Un razonamiento similar nos permite concluir que $q_{n}(x,y)$ no tiene monomios de la forma $bx^{n}$.

Como el polinomio $p_{n}$ es homogéneo de grado $n$ y no hay monomios de la forma $ay^{n}$, todos los monomios de $p_{n}$ tienen una potencia de $x$ y por la misma razón, todos los monomios del polinomio $q_{n}$ tienen una potencia de $y$. Así, podemos escribir a $p_{n}$ y $q_{n}$ como

\begin{equation}
\begin{aligned}
p_{n}(x,y) &=xf(x,y)\\
q_{n}(x,y) &=yg(x,y).
\end{aligned}
\end{equation}

 \noindent Además, los polinomios $f$ y $g$ son homogéneos de grado $n-1$. Si insertamos estas dos últimas igualdades en (\ref{CondicionDicritica}) obtenemos:

\begin{equation}
\label{A}
xyf(x,y)=xyg(x,y)
\end{equation}

\noindent Entonces, al cancelar el factor $xy$ de ambos lados de la igualdad (\ref{A}) se tiene que,

\begin{equation}
f(x,y)=g(x,y).
\end{equation}

Todo lo anterior quiere decir que, una foliación dicrítica siempre la podemos escribir de la forma:

\begin{equation}
\label{EcuacionDicritica}
\begin{aligned}
\dot{x} &=p_{1}(x,y)+\cdots+p_{d}(x,y)+xg(x,y)\\
\dot{y} &=q_{1}(x,y)+\cdots+q_{d}(x,y)+yg(x,y),
\end{aligned}
\end{equation}

\noindent con $g(x,y)$ un polinomio homogéneo de grado $d$.\\

En el caso de una foliación no dicrítica, vimos que el grado de los polinomios que definen a la foliación no son invariantes bajo cambios de coordenadas. Es por eso que si queremos asociar un grado a una foliación de $\C\mathbb{P}^{2}$ debemos encontrar otra manera de mirar a una foliación de $\C\mathbb{P}^{2}$.\\

Si en $\C^{3}$ consideramos una 1-forma $\Omega=A\, dx+B\, dy+C\, dz$, donde $A,B,C\in\C[x,y,z]$ son polinomios homogéneos de grado $d+1$, el conjunto $\{\Omega=0\}$ define una distribución de planos en $\C^{3}$. Podemos pensar que $\Omega$ asocia a un punto $(x_{0},y_{0},z_{0})$ el kernel de la transformación lineal $A(x_{0},y_{0},z_{0})x+B(x_{0},y_{0},z_{0})y+C(x_{0},y_{0},z_{0})z$.\\

Si queremos que $\Omega$, al proyectar en la carta afín de $\C\mathbb{P}^{2},\ z=1$ siga asociando a cada punto $[x_{0}:y_{0}:z_{0}]$ ya no un plano sino una recta, el plano original debe contener a la dirección determinada por $[x_{0}:y_{0}:z_{0}]$. Por ejemplo, si $\Omega$ asocia al punto $(0,0,1)$ el plano $x+z=0$, al proyectar en el plano $z=1$, el plano $x+z=0$ se proyecta en la recta $x=-1$ y esta recta ni siquiera pasa por el punto $(0,0)$ que es el representante de la clase $[0:0:1]$ en el plano $z=1$.\\

Una manera de evitar lo anterior es pedir que:

\begin{equation}
\label{EulerFieldCondition}
xA(x,y,z)+yB(x,y,z)+zC(x,y,z)=0 \,\, \forall x,y,z\in\C^{3}.
\end{equation}

\noindent Lo anterior es equivalente a pedir que la distribución de planos que define $\{\Omega=0\}$ contenga al campo vectorial radial:

\begin{equation}
\label{RadialVectorField}
V=x\frac{\partial}{\partial x}+y\frac{\partial}{\partial y}+z\frac{\partial}{\partial z}.
\end{equation}

\noindent Así, cualquier 1-forma $\Omega=A\, dx+B\, dy+C\, dz$ con coeficientes polinomiales homogéneos de grado $d+1$ que satisface la condición (\ref{EulerFieldCondition}), define una foliación de $\C\mathbb{P}^{2}$ que en la carta afín $z=1$ adquiere la forma:

\begin{equation}
\label{FormaAfin}
\omega=A(x,y,1)\, dx+B(x,y,1)\, dy.
\end{equation}

\noindent Observemos que el coeficiente $C$ desaparece por que si $z=1$, entonces $dz=0$.\\

Recíprocamente, si en una carta afín de $\C\mathbb{P}^{2}$ una foliación está generada por la 1-forma $\omega=p(x,y)\, dx+q(x,y)\, dy$ con $p,q\in\C[x,y]$ polinomios de grado $d$, entonces podemos levantar $\omega$ a una 1-forma $\Omega$ de $\C^{3}$ con coeficientes polinomiales homogéneos que satisface la identidad (\ref{EulerFieldCondition}). En efecto, si escogemos los coeficientes $A$ y $B$ como:

\begin{align}
A(x,y,z) &=z^{d+1}p(\tfrac{x}{z},\tfrac{y}{z})\\
B(x,y,z) &=z^{d+1}q(\tfrac{x}{z},\tfrac{y}{z}).
\end{align}

\noindent Entonces, la identidad (\ref{EulerFieldCondition}) obliga a que:

\begin{equation}
C(x,y,z) = z^{-1}(xA(x,y,z)+yB(x,y,z)).
\end{equation}

\noindent Es decir, para obtener a los polinomios homogéneos $A$ y $B$ de los polinomios $p$
y $q$ respectivamente, homogeneizamos a los polinomios $p$ y $q$ y después los multiplicamos por $z$.\\

Una observación muy importante que nos va a permitir definir el grado de una foliación en $\C\mathbb{P}^{2}$ es que, el grado de una 1-forma polinomial $\Omega$ de $\C^{3}$ es invariante bajo transformaciones del grupo general lineal $GL(3,\C)$ y en consecuencia es invariante bajo el grupo de transformaciones proyectivas $PGL(3,\C)$ de $\C\mathbb{P}^{2}$.  

\begin{defn}
\label{GradoDeUnaFoliacion}
Sea $\mathcal{F}$ una foliación de $\C\mathbb{P}^{2}$. Si en coordenadas homogéneas esta foliación está generada por la 1-forma $\Omega=A\, dx+B\, dy+ C\, dz$ con coeficientes polinomiales homogéneos de grado $d+1$ diremos que la foliación $\mathcal{F}$ tiene \emph{grado} $d$. 
\end{defn} 

Un resultado que será muy importante, es que toda foliación de grado $d$ tiene, contando multiplicidades, $d^{2}+d+1$ puntos singulares. Es por eso que apesar de que la foliación esté generada, en coordenadas homogéneas, por una 1-forma de grado $d+1$ diremos que la foliación tiene grado $d$. Cabe mencionar que algunos autores no hacen esta convención y definen el grado de una foliación como el grado de la 1-forma que genera a la foliación en coordenadas homogéneas.\\

A continuación veremos algunas propiedades que cumplen los polinomios $A,B,C$ de la 1-forma $\Omega=A\, dx+B\, dy+C\, dz$ que nos van a ayudar a probar que el número de puntos singulares que tiene la foliación generada por $\{ \Omega=0\}$ es $d^{2}+d+1$.\\

Si tenemos una foliación dada por la 1-forma $\Omega=A\, dx+B\, dy+C\, dz$ entonces podemos escribir a $A$ y a $B$ de la siguiente manera:
\begin{equation}
\label{CoeficientesRespectoAz}
\begin{aligned}
A &=a_{d+1}+a_{d}z+\cdots+a_{1}z^{d}+a_{0}z^{d+1}\\
B &=b_{d+1}+b_{d}z+\cdots+b_{1}z^{d}+b_{0}z^{d+1},
\end{aligned}
\end{equation}

\noindent donde los $a_{k},b_{k}\in\C[x,y]$ son polinomios homogéneos de grado $k$. Entonces tenemos que:

\begin{equation}
xA+yB = (xa_{d+1}+yb_{d+1})+(xa_{d}+yb_{d})z+\cdots+(xa_{1}+yb_{1})z^{d}+(xa_{0}+yb_{0})z^{d+1}.
\end{equation}

\noindent De la indentidad (\ref{EulerFieldCondition}) tenemos que $-zC=xA+yB$. Es decir que $z$ divide al polinomio $xA+yB$ y en consecuencia tenemos que:

\begin{equation}
\label{CondicionImportante}
 xa_{d+1}+yb_{d+1}\equiv 0.
\end{equation}

\noindent Razonando de manera análoga a como hicimos con la ecuación (\ref{CondicionDicritica}) y suponiendo que alguno de los polinomios $a_{d+1},b_{d+1}$ no es idénticamente cero, podemos concluir que:

\begin{equation} 
\begin{aligned}
\label{RadialHomogeneo}
a_{d+1} &= -yg(x,y)\\
b_{d+1} &= xg(x,y),
\end{aligned}
\end{equation}

\noindent con $g(x,y)$ un polinomio homogéneo de grado $d$.\\ %Además, de la ecuación (\ref{EulerFieldCondition}) se sigue que:

%\begin{equation}
%C=(xa_{d}+yb_{d})+(xa_{d-1}+yb_{d-1})z\cdots+(xa_{1}+yb_{1})z^{d-1}+(xa_{0}+yb_{0})z^{d}.
%\end{equation}

Si miramos a la foliación definida por $\{ \Omega=0 \}$ en la carta $z=1$ obtenemos la 1-forma $\omega=A(x,y,1)\, dx+B(x,y,1)\, dy$, y si usamos las ecuaciones (\ref{CoeficientesRespectoAz}) y (\ref{RadialHomogeneo}) obtenemos que $\omega$ adquiere la forma:

\begin{equation}
\omega=(a_{0}+a_{1}\cdots+a_{d}-yg)\, dx+(b_{0}+b_{1}+\cdots+b_{d}+xg)\, dy.
\end{equation} 

\noindent Esta 1-forma genera la misma foliación que el campo vectorial:

\begin{equation}
\begin{aligned}
\dot{x} &= b_{0}+b_{1}+\cdots+b_{d}+xg\\
\dot{y} &= -a_{0}-a_{1}+\cdots-a_{d}+yg.
\end{aligned}
\end{equation}

Podemos resumir lo anterior en el siguiente lema:

\begin{Lema}
Sea $\mathcal{F}$ una foliación de grado $d$ en $\C\mathbb{P}^{2}$. Entonces en cualquier carta afín $\mathcal{F}$ está generada por un campo vectorial de la forma:
\begin{equation}
\begin{aligned}
\dot{x} &= b_{0}+b_{1}+\cdots+b_{d}+xg\\
\dot{y} &= a_{0}+a_{1}+\cdots+a_{d}+yg.
\end{aligned}
\end{equation}
Donde $a_{k},b_{k}\in\C[x,y]$ son polinomios homgéneos de grado $k$ y $g\in\C[x,y]$ es un polinomio homogéneo de grado $d$ o g(x,y) es el polinomio cero.
\end{Lema}

Es fácil convencerse, usando la expresión (\ref{EcuacionEnCP2Infinito}), que el caso no dicrítico se da cuando $g(x,y)\equiv 0$ y el caso dicrítico se da cuando $g(x,y)$ no se anula idénticamente.\\

Como mencionamos anteriormente, podemos pensar que toda foliación en una dos variedad analítica $M$ sólo tiene singularidades aisladas, y en caso de que $M$ sea compacta, sólo hay un número finito de ellas. Así, como $\C\mathbb{P}^{2}$ es compacto, cualquier foliación de él tiene un número finito de singularidades y esto nos permite que siempre podamos tomar una carta afín en la cual la recta al infinito no tenga singularidades. Vamos a usar esta observación en la prueba del siguiente teorema:

\begin{Teorema}
Sea $\mathcal{F}$ una foliación de grado $d$. Entonces, contando multiplicidades, $\mathcal{F}$ tiene $d^{2}+d+1$ puntos singulares.
\end{Teorema}

\begin{proof}
Si escogemos una carta afín en la cual la recta al infinito no tenga puntos singulares, en esta carta afín el campo vectorial que genera a la foliación adquiere la forma:
\begin{equation*}
\begin{aligned}
\dot{x} &= b_{0}+b_{1}+\cdots+b_{d}+xg\\
\dot{y} &= a_{0}+a_{1}+\cdots+a_{d}+yg.
\end{aligned}
\end{equation*}

Las dos curvas algebraicas que definen esta ecuación son de grado $d+1$ así que, por el Teorema de Bézout, estas dos curvas se intersecan, contando multiplicidades,  en $d^{2}+2d+1$ puntos singulares. Cualquiera de estos puntos de intersección que esté en nuestra carta afín va a ser un punto singular de la foliación, pero los puntos de intersección de estas dos curvas en la recta al infinito, gracias a la manera en que escogimos nuestras coordenadas, no son puntos singulares de la foliación. Por lo tanto, para probar el teorema basta demostrar que las dos curvas algebraicas que definen la ecuación se intersecan $d$ veces en la recta al infinito.\\

Para ver en que puntos la curva $\{A=a_{0}+\cdots+a_{d}+yg=0\}$ interseca a la recta al infintio tenemos que homogeneizar al polinomio $A$ y después evaluar en $z=0$. Si homogeneizamos al polinomio $A$ obtenemos el polinomio $a_{0}z^{d}+\cdots+a_{d}z+yg$, y al evaluar en $z=0$ vemos que esta curva corta a la recta al infinito en el punto $[1:0:0]$ y en los puntos determinados por las raíces del polinomio $g$. De manera análoga la curva $\{B=b_{0}+b_{1}+\cdots+b_{d}+xg\}$ corta a la recta al infinito en los puntos $[0:1:0]$ y en los puntos determinados por las raíces de $g$. Como $g$ es un polinomio de grado $d$, las dos curvas se interesecan $d$ veces en la recta al infinito.

\end{proof}


\chapter{Introducción}

  \section{El problema de Poincaré}

    El problema de Poincaré para foliaciones de $\CP$ consiste en acotar, dada una foliación, el grado de una posible solución algebraica en términos del grado de la foliación.
 
\begin{Ejemplo}
\label{Ej:Intro}
Consideremos la siguiente familia de ecuaciones diferenciales en $\CP$ a un parámetro $\lambda\in\mathbb{R}$:
\begin{align*}
\dot{x} & = x \\
\dot{y} & = \lambda y.
\end{align*}
O lo que es lo mismo, la foliación generada por los ceros de la 1-forma $x\,dy - \lambda y\,dx$.
\end{Ejemplo}

\noindent En el ejemplo anterior si $\lambda$ es un número irracional, la foliación definida por la ecuación diferencial sólo tiene tres hojas algebraicas invariantes, los dos ejes coordenados y la recta al infinito. En efecto, si existiera una hoja algebraica de grado $n$, el teorema de Bézout nos asegura que esta hoja interseca a la recta $x=1$ en $n$ puntos, pero la holonomía asociada a la separatriz $y=0$ y la transversal $x=1$ es $z\mapsto e^{2\pi i\lambda}z$,  y como $\lambda$ es irracional, la órbita de cada punto es infinita, pero por construcción de la holonomía, la órbita de un punto $z$ está contenida en la intersección de la hoja de la foliación $\mathcal{L}_z$ que pasa por $z$ con la transversal $x=1$. 
\\
Por otro lado, si $\lambda = \tfrac{p}{q}$, entonces $H = \tfrac{y^q}{x^p}$ es una primera integral racional ya que $dH = q\tfrac{y^{q-1}}{x^p}\,dy - p\tfrac{y^q}{x^{p+1}}\,dx = \tfrac{y^{q-1}}{x^{p+1}}(qx\,dy -py\,dx)$ y así, todas las hojas $y^q-cx^p=0$ son algebraicas.
\\

La ecuación del ejemplo anterior es lineal y eso hace que el grado de la foliación que genera sea uno. Sin embargo, es fácil ver que no importa qué número natural $k$ demos, siempre hay algún parámetro $\lambda$ para el cual se cumplen las siguientes dos propiedades:

\begin{enumerate}
\item La foliación correspondiente a ese parámetro $\Fol{\lambda}$ tiene una primera integral racional y por lo tanto todas sus hojas son algebraicas.

\item El grado de las hojas de la foliación es mayor que $k$.
\end{enumerate}  

En efecto, basta tomar $\lambda = \tfrac{1}{n}$ con $n>k$ pues así las hojas de la foliación corresponden a $y^n - cx = 0$.
\\

El ejemplo anterior muestra que si no ponemos algunas condiciones más restrictivas a la foliación, no podemos, en general, acotar el grado de las soluciones algebraicas.\\

Si suponemos que la foliación $\mathcal{F}$ tiene una hoja algebraica $C=\{ f=0\}, \, f\in\C[x,y]$ y que además la hoja algebraica $C$ es suave o sólo se autointerseca de manera transversal se tiene el siguiente resultado:\\

\begin{Teorema}
\label{Teo:CotaHojaSuave}
Sea $\mathcal{F}$ una foliación de grado $d$ y $C$ una hoja algebraica de grado $m$ de la foliación $\mathcal{F}$.

Si la curva $C$ es suave o a lo más tiene autointersecciones transversales, entonces $m\leq d+2$.
\end{Teorema}

En lugar de pedir condiciones a la hoja algebraica, podemos suponer que la foliación no tiene singularidades dicríticas.

\begin{Teorema}
\label{Teo:CotaSingularidadesNoDicriticas}
Si una foliación $\mathcal{F}$ de grado $d$ no tiene puntos singulares dicríticos, entonces cualquier hoja algebraica de la foliación es de grado $m\leq d+2$.
\end{Teorema}

Para ver una demostración de estos teoremas véase \cite[pp.~479,480]{IlyaYako}.

\section{Escasez de foliaciones con hojas algebraicas.}

En los dos resultados anteriores es necesario suponer la existencia de una hoja algebraica, pero de manera genérica, una foliación de grado $d$ no tiene hojas algebraicas. Para precisar qué quiere decir que algo suceda de manera genérica haremos algunas observaciones.\\

El conjunto $\mathbb{L}_{f}$ de polinomios homogéneos de grado $d+1$ es un espacio vectorial y por lo tanto, el conjunto de 1-formas en $\C^{3}$ con coeficientes polinomiales homogéneos de grado $d+1$ es un espacio vectorial. Observemos que si $\omega, \, \omega'$ son 1-formas homogéneas de grado $d+1$ que satisfacen la condición (\ref{EulerFieldCondition}) entonces $\omega + \omega'$ y $\lambda\omega,\, \lambda\in\C$ también satisfacen la condición (\ref{EulerFieldCondition}). Así, el conjunto $\mathbb{L}_{\Foli,d}$ de 1-formas  con coeficientes polinomiales homogéneos  de grado $d+1$ que satisfacen la condición (\ref{EulerFieldCondition}) es un espacio vectorial.\\

Si los coeficientes de $\Omega\in\mathbb{L}_{\Foli,d}$ tienen un factor $g$ en común, entonces el grado de la foliación que determina $\{\Omega =0\}$ es $d-deg(g)$. Así, si una foliación $\Foli$ de grado $n\leq d$ es generada por la 1-forma $\Omega'$ entonces, si multiplicamos a $\Omega'$ por un polinomio homogéneo $g$ de grado $d-n$, la 1-forma $\Omega=g\Omega'$ genera a la misma foliación $\Foli$ y $\Omega\in\mathbb{L}_{\Foli,d}$. Es decir, en el espacio vectorial $\mathbb{L}_{\Foli,d}$ están todas las foliaciones de grado menor o igual a $d$.\\

Como $\mathbb{L}_{\Foli,d}$ es un espacio vectorial, podemos introducir en él la medida de Lebesgue usual y entonces podemos afirmar que, salvo en un conjunto de medida cero, cualquier foliación de grado $d$ no tiene hojas algebraicas.\\

Para dar una idea de la demostración de este hecho consideremos $f\in\C[x,y,z]$ un polinomio homogéneo  de grado $m$  y una foliación $\mathcal{F}$ de grado $d$ generada por la 1-forma $\Omega$. Si $C=\{f=0\}$ es una curva invariante de $\mathcal{F}$ entonces en los puntos $p\in C$, $df$ y $\Omega$ determinan el mismo subespacio lineal y en consecuencia $(\Omega\wedge df)_{p}=0$. Esto quiere decir que la 2-forma $\Omega\wedge df$ se anula en toda la curva $C=\{f=0\}$ y por lo tanto,

\begin{equation}
\label{EcuacionAlgebraica} 
\Omega\wedge df=f\Phi.
\end{equation}
\marginpar{que pasa en el caso afin}
\noindent Notemos que la 2-forma $\Phi$ es polinomial y homogénea de grado $d$ (esto se debe a que los coeficientes de $\Omega$ son de grado $d+1$ y $df$ tiene grado $m-1$). El conjunto $\mathbb{L}_{\Phi,d}$ de 2-formas homogéneas también es un espacio vectorial complejo y si un par $(f,\Phi)\in\mathbb{L}_{f,m}\times\mathbb{L}_{\Phi,d}$ satisface la ecuación (\ref{EcuacionAlgebraica}) entonces $C=\{f=0\}$ es una curva invariante de la foliación determinada por $\{\Omega=0\}$.\\

Con todo lo anterior, la ecuación (\ref{EcuacionAlgebraica}) define una variedad algebraica $Q$ en el espacio $\mathbb{L}_{\Omega,d}\times\mathbb{L}_{f,m}\times\mathbb{L}_{\Phi,d}$. En efecto, la operación $\wedge$ sólo involucra sumas y productos, así $\Omega\wedge df-f\Phi=0$ determina tres polinomios (para especificar una 2-forma en $\C^{3}$ necesitamos tres coeficientes) en el espacio afín $\mathbb{L}_{\Omega,d}\times\mathbb{L}_{f,m}\times\mathbb{L}_{\Phi,d}$. Por lo tanto, la proyección de la variedad algebraica $Q$ en el espacio $\mathbb{L}_{\Omega,d}$ es el conjunto de todas las foliaciones de $\CP$ de grado $r\leq d$ que tienen una hoja algebraica de grado $n\leq m$. Como las 1-formas $\omega$ y $\lambda\omega,\ \lambda\neq 0$ generan la misma foliación podemos proyectivizar a $\Pi(Q)$. Veremos que la proyectivización de $\Pi(Q)$ es una variedad algebraica en $\mathbb{P}(\mathbb{L}_{\Omega,d})$. Para eso requeriremos algunos resultados y definiciones. \\


\begin{defn}
\label{Def:ProyeccionProyectiva}
Sean $[x_{0},\ldots,x_{n}]$ coordenadas homogéneas de $\C\mathbb{P}^{n}$. Si $L^{k}\subset\C\mathbb{P}^{n}$ es un subespacio lineal definido por los ceros en común de las $n-k$ formas lineales $y_{i}=\sum_{j=0}^{n} a_{ij}x_{j}$, $0\leq i\leq n-k$ definimos la proyección de $\mathbb{CP}^{n}$ con centro en $L^{k}$,
\begin{equation*}
p_{L^{k}}\colon \mathbb{CP}^{n}\setminus L^{k}\rightarrow\mathbb{CP}^{n-k-1},
\end{equation*}
como
\begin{equation*}
[b_{0},\ldots,b_{n}]\rightarrow[\sum_{j=0}^{n} a_{oj}b_{j},\ldots,\sum_{j=0}^{n} a_{(n-k)j}b_{j}].
\end{equation*}
\end{defn}

A continuación vamos a proyectivizar muchos espacios vectorial complejos, es por eso que si $V$ es un espacio vectorial complejo denotaremos por $\mathbb{P}(V)$ al espacio obtenido al proyectivizar $V$.

\begin{Ejemplo}
\label{Ej:ProyeccionCanonica}
Si $[x_{0},\ldots,x_{n}]$ y $[y_{0},\ldots,y_{m}]$ son coordenadas homogéneas de $\mathbb{P}(\C^{n+1})$ y $\mathbb{P}(\C^{m+1})$ respectivamente, un caso particular de la definición anterior y que usaremos más adelante es cuando en $\mathbb{P}(\C^{n+1}\times\C^{m+1})$ proyectamos de manera paralela a $L=\{ [0,\ldots,0,y_{0},\ldots,y_{m}] \}\simeq\mathbb{P}(\C^{m+1})$. Este subespacio lineal queda descrito por $x_{0}=\cdots=x_{n}=0$ y así, la proyección adquiere la expresión sencilla:
\begin{equation*}
[x_{0},\ldots,x_{n},y_{0},\ldots,y_{m}]\rightarrow [x_{0},\ldots,x_{n}].
\end{equation*}
Por lo tanto, podemos pensar a la proyección de $\mathbb{P}(\C^{n+1}\times\C^{m+1})$ con centro en $\{0\}\times\mathbb{P}(\C^{m+1})$ como la proyección de $\mathbb{P}(\C^{n+1}\times\C^{m+1})$ en $\mathbb{P}(\C^{n+1})\times\{0\}\simeq\mathbb{P}(\C^{n+1})$.  
\end{Ejemplo}

\begin{Teorema}
\label{Teo:CompacidadProyectiva}
Si $X\subset\C^{n}\times\mathbb{CP}^{p}$ es una variedad algebraica, entonces la proyección de $X$ en el primer factor es una variedad algebraica en $\C^{n}$.
\end{Teorema}

El resultado anterior es muy sorprendente pues no es difícil convencerse de que si $X\subset\C^{n}\times\C^{m}$ es una variedad algebraica, la proyección de $X$ en cualquiera de los dos factores no siempre es una variedad algebraica (por ejemplo, el cero no está en la proyección de $\{xy-1=0\}$ en cualquiera de los dos ejes). El siguiente lema, conocido como el lema de normalización de Noether, es una consecuencia del teorema anterior.

\begin{Lema}
\label{Lema:NormalizacionDeNoether}
Si $X\subset\C\mathbb{P}^{n}$ es una variedad algebraica y $L^{k}\subset\mathbb{CP}^{n  }$ es un subespacio que no interseca a $X$, entonces la proyección con centro en $L^{k}$ de $X$ en $\mathbb{CP}^{n-k-1}$ es una variedad algebraica. 
\end{Lema}

Para ver una prueba de los dos resultados anteriores consúltese \cite{Mumford}. Con estos dos resultados podemos probar el siguiente lema,

\begin{Lema}
\label{Lema:ConjuntoAlgebraico}
Para todo $d\geq 2$ y $m\geq 1$ las foliaciones de grado $r\leq d$ con una hoja algebraica de grado $n\leq m$ constituyen una variedad algebraica en el espacio $\Pro(\mathbb{L}_{\Omega,d})$.
\end{Lema}

\begin{proof}
Como vimos anteriormente la ecuación (\ref{EcuacionAlgebraica}) determina una variedad algebraica en $\mathbb{L}_{\Omega,d}\times\mathbb{L}_{f,m}\times\mathbb{L}_{\Phi,d}$. Notemos que si $f$ satisface la ecuación (\ref{EcuacionAlgebraica}) entonces $\lambda f$ también satisface la ecuación y por lo tanto la variedad algebraica está en $\mathbb{L}_{\Omega,d}\times\Pro(\mathbb{L}_{f,m)}\times\mathbb{L}_{\Phi,d}$.

 Si llamamos $Q$ a esta variedad, por el teorema \ref{Teo:CompacidadProyectiva}, la proyección de $Q$ en $\mathbb{L}_{\Omega,d}\times\mathbb{L}_{\Phi,d}$ es una variedad algebraica $Q'$. Observemos que si $(\Omega,\Phi)$ satisfacen la ecuación (\ref{EcuacionAlgebraica}), $(\lambda\Omega,\lambda\Phi)$ también lo hacen, así podemos proyectivizar a $Q'$ para obtener una variedad algebraica $Q''$ en $\Pro(\mathbb{L}_{\Omega,d}\times\mathbb{L}_{\Phi,d})$.

El subespacio $\{0\}\times\Pro(\mathbb{L}_{\Phi,d})$ de $\Pro(\mathbb{L}_{\Omega,d}\times\mathbb{L}_{\Phi,d})$ (véase ejemplo \ref{Ej:ProyeccionCanonica}) no interseca a $Q''$. En efecto, si $\Omega=0$ entonces para que la ecuación (\ref{EcuacionAlgebraica}) se satisfaga $f=0$ ó $\Phi=0$ y el punto $(0,0)\in\mathbb{L}_{\Omega,d}\times\mathbb{L}_{\Phi,d}$ no está en $\Pro(\mathbb{L}_{\Omega,d}\times\mathbb{L}_{\Phi,d})$. Por lo tanto, si hacemos la proyección con centro en $\{0\}\times\Pro(\mathbb{L}_{\Phi,d})$ obtenemos una variedad algebraica $Q'''$ en $\Pro(\mathbb{L}_{\Omega,d})$. Esta variedad algebraica se corresponde con las foliaciones de grado $r\leq d$ con una hoja algebraica de grado $n\leq m$.  
\end{proof}

Una variedad algebraica tiene medida cero a menos que la variedad algebraica sea todo el espacio y entonces tiene medida infinita. Por lo tanto, para demostrar que  el conjunto de las foliaciones de grado $d$ con una hoja algebraica de grado $m$ tiene medida cero, basta exhibir una foliación de grado $d$ que no tenga hojas algebraicas.\\

Un ejemplo, debido a Jouanolou, de una foliación de grado $d$ sin hojas algebraicas es la foliación generada en la carta afín $(x,y)$ por la 1-forma,

$$(x^{d}-yy^{d})\, dx-(1-xy^{d})\, dy.$$

Para ver una demostración de este hecho consúltese \cite{IlyaYako}.


\section{El teorema de Darboux.} 

A pesar de que una foliación de $\CP$ casi nunca tiene hojas algebraicas, si tomamos una ecuación diferencial polinomial en $\C^{2}$ y la foliación que genera tiene más de cierto número de soluciones algebraicas, entonces todas las soluciones son algebraicas.\\

Si $F=(F_{1},F_{2})$ es un campo vectorial polinomial de grado $r$ y $C=\{f=0\}$ es una curva algebraica de grado $m$ que es solución la ecuación diferencial determinada por el campo $F$ entonces, para todo punto $p\in C$ tenemos que,

\begin{align}
df(F) &=\frac{\partial f}{\partial x}F_{1}+\frac{\partial f}{\partial y}F_{2}=0.
\end{align}

\noindent Por lo tanto, $f$ divide a la función $df(F)$ y en consecuencia $df(F)=fg$. Como el grado de $f$ es $m$ y el grado de $F$ es $r$, el grado de $g$ es a lo más $r-1$. Además, el espacio de los polinomios de grado $n\leq r-1$ en dos variables tiene dimensión $\frac{1}{2}r(r+1)$. Estas dos observaciones nos permiten probar el siguiente teorema.

\begin{Teorema}
\label{Teo:AnteDarboux}
Si una foliación esta definida por un campo vectorial $F$ de grado $r$ y tiene $n\geq \frac{1}{2}r(r+1)+1$ soluciones algebraicas $C_{k}=\{f_{k}=0\}, k=1,\ldots,n$ entonces, la foliación tiene una primera integral multivaluada $H=f_{1}^{\lambda_{1}}\cdots f_{n}^{\lambda_{n}}$.
\end{Teorema}

\begin{proof}
Para cada polinomio $f_{i}$ que define una curva invariante $C_{i}=\{f_{i}=0\}$ tenemos que $df_{i}(F)=f_{i}g_{i}$ con el grado de $g_{i}$, $deg(g_{i})\leq r-1$. Por lo tanto, si existen $n\geq \frac{1}{2}r(r+1)+1$ curvas invariantes, entonces los polinomios $g_{i}, i=1,\ldots,n$ son linealmente dependientes en el espacio de polinomios de grado a lo más $r-1$ y así, podemos encontrar una combinación lineal de ellos tal que $\lambda_{1}g_{1}+\cdots \lambda_{n}g_{n}=0$ con al menos un $\lambda_{i}\neq 0$. Si $H=f_{1}^{\lambda_{1}}\cdots f_{n}^{\lambda_{n}}$ entonces tenemos que,
\begin {align*}
dH=\lambda_{1}\frac{H}{f_{1}}df_{1}+\cdots+\lambda_{n}\frac{H}{f_{n}}df_{n}=H\sum_{i=1}^{n}\lambda_{i}\frac{df_{i}}{f_{i}}.
\end{align*}

\noindent Por lo tanto,
\begin{align*}
dH(F)=H\sum_{i=1}^{n}\lambda_{i}\frac{df_{i}(F)}{f_{i}}=H\sum_{i=1}^{n}\lambda_{i}\frac{f_{i}g_{i}}{f_{i}}=H\sum_{i=1}^{n}\lambda_{i}g_{i}=0.
\end{align*}
Esto quiere decir que $H$ es una primera integral de la foliación generada por el campo vectorial $F$.
\end{proof}

Si hay una solución algebraica más podemos encontrar una primera integral racional.

\begin{Teorema}
\label{Teo:Darboux}
Si campo vectorial polinomial $F$ de grado $r$ tiene $n=\frac{1}{2}r(r+1)+2$ soluciones algebraicas, entonces tiene una primera integral racional.
\end{Teorema}
\begin{proof}
Si las curvas algebraicas están definidas por los ceros de $f_{1},\ldots,f_{n}$ entonces, para algunos $\lambda_{i}$, las funciones $H=f_{1}^{\lambda_{1}}\cdots f_{n-1}^{\lambda{n-1}}$ y $H'=f_{2}^{\lambda_{2}}\cdots f_{n}^{n}$ son primeras integrales multivaluadas del campo vectorial $F$. Así, las 1-formas racionales,
\begin{align*}
\omega=\frac{dH}{H}=\sum_{n=1}^{n-1}\lambda_{i}^{n-1}\frac{df_{i}}{f_{i}}\,\,\,\, \mathrm{y}\\
\omega'=\frac{dH'}{H'}=\sum_{n=2}^{n}\lambda_{i}^{n-1}\frac{df_{i}}{f_{i}},
\end{align*}
\noindent generan la misma foliación y por lo tanto, son proporcionales. Esto quiere decir que existe una función racional $h$ tal que $\omega =h\omega'$. Así, $0=d\omega=dh\wedge \omega + hd\omega' =dh\wedge\omega$ ya que $\omega$ y $\omega'$ son exactas. Como $dh\wedge\omega =0$, $h$ es una primera integral racional de $F$.
\end{proof}

\section{Teorema principal.}

Como vimos en las secciones anteriores, casi no hay foliaciones de $\CP$ con hojas algebraicas. Las foliaciones de grado a lo más $d$ con hojas algebraicas de grado a lo más $m$ forman un subconjunto algebraico en el espacio de foliaciones de grado a lo más $d$. Es por eso que es importante encontrar subconjuntos (familias de foliaciones) de esta variedad algebraica.
\begin{defn}
\label{Def:TipoAnFijo}
Sea $(\Fol{s})_{s\in S}$ una familia de foliaciones de \CP\, donde $S$ es una variedad holomorfa tal que los coeficientes de la ecuación que definen a cada foliación de la familia, en una carta afín fija, dependen de manera holomorfa de $s\in S$. Decimos que la familia tiene singularidades de \emph{tipo analítico fijo} si:

\begin{enumerate}

\item Las singularidades de $\mathcal{F}_s$, $s\in S$, se pueden escribir como $\mathit{sing}(\mathcal{F}_s) = \{p_1(s),\ldots,p_k(s) \}$, donde las funciones $s\in S\mapsto p_j(s)$ son holomorfas.

\item Para cada $j\in\{1,\ldots,k\}$ y $s_1,s_2\in S$, existen una vecindadades $U_1,U_2$ de $p_j(s_1)$ y $p_j(s_2)$ respectivamente, de tal forma que las foliaciones $\mathcal{F}_{s_1},\mathcal{F}_{s_2}$  son analíticamente equivalentes en estas vecindades.
%Es lo mismo decir que son analiticamente equivalentes
\end{enumerate}
\end{defn}

Si además se cumple que para toda $s\in S$, las singularidades de $\mathcal{F}_\lambda$ tienen dos valores propios distintos de cero, diremos que la familia tiene \emph{singularidades no degeneradas}.
\\

Uno de los propósitos de esta tesis es probar el siguiente resultado:

\begin{TeoPrin}
\label{Teo:Prin}
Para $d = 2,3,4$, existen familias de foliaciones de \CP, digamos $(\Fol[d]{\alpha})_{\alpha\in\overline{\C}}$, de grado $d$ que cumplen:

\begin{enumerate}

\item Existe un subconjunto finito de parámetros $A^d\subset\overline{\C}$ tal que la familia restringida, $(\Fol[d]{\alpha})_{\alpha\in\overline{\C}\setminus A^d}$ tiene singularidades no degeneradas de tipo analítico fijo.

\item Existe un subconjunto denso y numerable $E\subset\overline{\C}$, tal que para cualquier $\alpha\in E$, la foliación \Fol[d]{\alpha}\ tiene una primera integral racional $F_{\alpha} = \tfrac{P_{\alpha}}{Q_{\alpha}}$ de grado $d_{\alpha}$ y se satisface que para cualquier $k>0$, el conjunto $\{\alpha\in E ; d_{\alpha}\leq k\}$ es finito. Esto nos dice que para todo natural $k$, hay una infinidad de parámetros $\alpha\in E$ cuya foliación correspondiente $\Fol[d]{\alpha}$ tiene primera integral racional de grado mayor que $k$ y es por lo tanto un contraejemplo al problema de Poincaré.

\end{enumerate}
\end{TeoPrin}







\chapter{La familia de grado 4.}
En este capítulo estudiaremos una familia monoparamétrica $\alpha\in\C$ de foliaciones de grado cuatro generada por las ecuaciones diferenciales:
\begin{equation}
\label{EcuacionGeneral}
\begin{aligned}
\dot{x} &= (x^{3}-1)(x-\alpha y^{2})\\
\dot{x} &= (y^{3}-1)(y-\alpha x^{2}).
\end{aligned}
\end{equation}

Cada foliación de esta familia tiene como hojas invariantes a nueve rectas que se intersecan en doce puntos. Estas nueve rectas y doce puntos cumplen ciertas propiedades y veremos que si una foliación de grado cuatro deja invariantes a nueve rectas y doce puntos que cumplen las mismas propiedades, entonces en coordenadas adecuadas la foliación está generada por la ecuación (\ref{EcuacionGeneral}) para algún $\alpha\in\C$.

 
	\section{La configuración.}

     	    De aquí en adelante, denotaremos por $j=e^{\frac{2\pi i}{3}}$.
\\


El ejemplo de grado 4 es una familia de ecuaciones que dejan invariante a 9 rectas, a saber: $\{x=1\}, \{x=j\}, \{x=j^{2}\}, \{y=1\}, \{y=j\}, \{y=j^{2}\}, \{y=x\}, \{y=jx\}, \{y=j^{2}x\}$.
\\

Estas 9 rectas se intersectan en 12 puntos los cuales son $(1,1),\ (1,j),\ (1,j^{2}),\ (j,1),\ (j,j),$\\ $(j,j^{2}),\ (j^{2},1),\ (j^{2},j),\ (j^{2},j^{2}),\ [0:0:1],\ [0:1:0]$ y $[1:0:0]$.
\\

Denotemos por $\mathcal{L}$ al conjunto de las 9 rectas, por $\mathcal{P}$ al conjunto de los 12 puntos y por $\mathcal{C=(L,P)}$ a la configuración de las 9 rectas y los 12 puntos.
\\

Observemos que la configuración cumple las siguientes tres propiedades:

\begin{enumerate}

\item Cada recta tiene 4 puntos de la configuración.
\item Por cada punto de la configuración pasan 3 rectas de la configuración
\item Si 3 puntos de $\mathcal{P}$ no están en una recta de la configuración, entonces no están alineados.

\end{enumerate} 

A continuación probaremos que, módulo transformaciones proyectivas, ésta es la única configuración de 9 rectas y 12 puntos que satisfacen los tres propiedades anteriores. Esta proposición y sus dos corolarios serán de gran utilidad en el futuro, pues para muchas cuentas bastará hacer un cálculo en un lugar particular de $\CP$ y después usar alguna transformación proyectiva para argumentar que el cálculo es válido en otros lugares de $\CP$.
\\

Sea $\mathcal{C'=(L',P')}$ una configuración de 9 rectas y 12 puntos que satisfacen las tres propiedades de arriba. Es importante recordar que los 12 puntos son las intersecciones de las nueve rectas.

\begin{Lema}
\label{Lema:Conf}
$\mathcal{P'}$ puede ser dividido en cuatro conjuntos, $\mathcal{P}_{1}$, $\mathcal{P}_{2}$, $\mathcal{P}_{3}$, $\mathcal{P}_{4}$, tales que:

\begin{enumerate}

\item Cada $\mathcal{P}_{i}$ tiene tres elementos de $\mathcal{P'}$.
\item $\mathcal{P}_{i} \cap \mathcal{P}_{k} = \emptyset$ si $i \neq j$
\item Los tres puntos en cada $\mathcal{P}_{i}$ no son colineales.

\end{enumerate}

\end{Lema}

\begin{proof}

  Sea $p_{1} \in \mathcal{P'}$, entonces, por la propiedad 2 de la configuración, hay tres rectas $\mathit{l}_{1},\ \mathit{l}_{2}, \mathit{l}_{3} \in \mathcal{L'}$, que pasan por $p_{1}$. Por la propiedad 1, cada una de estas rectas tiene tres puntos de $\mathcal{P'}$ distintos de $p_{1}$. Por lo tanto, de los doce puntos de $\mathcal{P'}$, nueve están unidos a $p_{1}$ con una recta de $\mathcal{L'}$. Así, la recta que une a los dos puntos restantes, $p_{2},\ p_{3} \in \mathcal{P'}$ no es una recta de la configuración y entonces, por la propiedad 3, $p_{1},\ p_{2},\ p_{3}$, no son colineales. Llamemos $\mathcal{P}_{1} = \{p_{1},\ p_{2},\ p_{3} \}$.
\\

Ahora, si $\mathit{l}_{4},\ \mathit{l}_{5}, \mathit{l}_{6} \in \mathcal{L'}$ son las tres rectas que pasan por $p_{2}$, entonces, por el teorema de Bézout, $\{ \mathit{l}_{1},\ \mathit{l}_{2}, \mathit{l}_{3} \}$ intersecta a  $\{ \mathit{l}_{4},\ \mathit{l}_{5}, \mathit{l}_{6} \}$ en nueve puntos de $\mathcal{P'}$, y como ninguno de ellos puede ser $p_{1}$ ó $p_{3}$, concluímos que deben ser los mismos nueve que descartamos al elegir a $p_{2}$ y a $p_{3}$. Es decir, si a $p_{2}$ le asociamos otros dos puntos de $\mathcal{P'}$ como a $p_{1}$, los puntos que le corresponden son $p_{1}$ y $p_{3}$.
\\

Por lo tanto, la construcción anterior parte a $\mathcal{P'}$ en cuatro conjuntos con las propiedades deseadas.

\end{proof}

En el caso particular en que $\mathcal{P}'=\mathcal{P}$ los conjuntos $\mathcal{P}_{i}$ son: $\mathcal{P}_{1}=\{ (0,0),[1:0:0][0:1:0]\}, \mathcal{P}_{2}=\{(1,1),(j,j^{2}),(j^{2},j)\}, \mathcal{P}_{3}=\{(1,j,),(j,1),(j^{2},j^{2})\}$ y $\mathcal{P}_{4}=\{(1,j^{2}),(j,j),(j^{2},1)\}$.\\


\begin{Proposicion}
\label{Prop:1}
Sean $\mathcal{C'},\ \mathcal{P}_{i} = \{p_{1},\ p_{2},\ p_{3} \}$ como en el lema \ref{Lema:Conf}. Entonces existe un automorfismo $T$ de \CP tal que $T(\mathcal{C'})=C$ $(\mathit{i.e.}\ T(\mathcal{L'}) = \mathcal{L},\ T(\mathcal{P'}) = \mathcal{P})$ y además $ T(\mathcal{P}_{i}) = \{ [1:0:0],[0:1:0];[0:0:1] \}$

\end{Proposicion}

\begin{proof}

Como los puntos de $\mathcal{P}_{i}$ no son colineales, podemos encontrar coordenadas de tal forma que: $p_{1} = [0:0:1]\ p_{2} = [0:1:0]\ p_{3} = [1:0:0]$. En estas coordenadas y en la carta afín $z=1$, las retas por $p_{2} = [0:1:0]$ son de la forma $x=a_{1},\ x=a_{2},\ x=a_{3}$ y las que pasan por $p_{3} = [1:0:0]$ son de la forma $y=b_{1},\ y=b_{2},\ y=b_{3}$. Así, las tres rectas restantes (las que pasan por $p_{1} = [0:0:1]$) tienen la forma $ y = \alpha x,\ y = \beta x,\ y = \gamma x$ y los nueve puntos restantes de $\mathcal{P'}$ son $(a_{1},b_{1})\ (a_{1},b_{2})\ (a_{1},b_{3})\ (a_{2},b_{1})\ (a_{2},b_{2})\ (a_{2},b_{3})\ (a_{3},b_{1})\ (a_{3},b_{2})\ (a_{3},b_{3})\ $
\\

Reordenando los índices, podemos suponer que $(a_{i}, b_{i}) \in \{ y = \alpha x \},\ i=1,2,3$. $\mathit{i.e.}\ \frac{b_{i}}{a_{i}} = \alpha \ i=1,2,3$.
\\

Supongamos que $(a_{1},b_{2})$ está en la recta $\{ y = \beta x\}$, veamos que otros puntos de $\mathcal{P'}$ están en $\{ y = \beta x\}$. Notemos primero que estos puntos no pueden tener en su primera entrada a $a_{1}$ y en la segunda entrada no pueden tener a $b_{2}$, como además, $(a_{3},b_{3})$ ya está en $\{ y = \alpha x \}$, las únicas posibilidades son: $(a_{2}, b_{3}),\ (a_{2}, b_{1})$ y $(a_{3}, b_{1})$. Pero $(a_{2}, b_{3})$ y $(a_{2}, b_{1})$ no pueden estar ambos en $\{ y = \beta x\}$, así que forzosamente, $(a_{3},b_{1}) \in \{ y = \beta x\}$. De manera análoga, $(a_{2},b_{1})$ y $(a_{3},b_{1})$ no pueden estar ambos en $\{ y = \beta x\}$ y entonces $(a_{2},b_{3}) \in \{ y = \beta x\}$.
\\

Por lo tanto, $(a_{1},b_{1}),\ (a_{2},b_{2}),\ (a_{3},b_{3})\in \{ y = \alpha x\}$.
\\
$(a_{1},b_{2}),\ (a_{2},b_{3}),\ (a_{3},b_{1})\in \{ y = \beta x\}$.
\\
$(a_{1},b_{3}),\ (a_{2},b_{1}),\ (a_{3},b_{2})\in \{ y = \gamma x\}$.
\\

Y entonces, tenemos las siguintes relaciones:

$$\frac{b_{1}}{a_{1}}=\frac{b_{2}}{a_{2}}=\frac{b_{3}}{a_{3}}=\alpha,\ \ \frac{b_{2}}{a_{1}}=\frac{b_{3}}{a_{2}}=\frac{b_{1}}{a_{3}}=\beta,\ \ \frac{b_{3}}{a_{1}}=\frac{b_{1}}{a_{2}}=\frac{b_{2}}{a_{3}}=\gamma.$$

Lo anterior implica que $\alpha^{3} = \beta^{3} = \gamma^{3}$ y así, $\frac{\beta}{\alpha}$ y $\frac{\gamma}{\alpha}$ son raíces cúbicas distintas de la unidad, digamos $j$ y $j^{2}$ respectivamente.
\\

Por último, veamos que $(a_{1},a_{2},a_{3}) = a_{1}(1,j,j^{2})$ y $(b_{1},b_{2},b_{3}) = \alpha a_{1}(1,j,j^{2})$:

$$a_{1}j = a_{1}\frac{\beta}{\alpha}=a_{1}\frac{\frac{b_{2}}{a_{1}}}{\frac{b_{2}}{a_{2}}} = a_{2},\ \ a_{1}j^{2} = a_{1}\frac{\gamma}{\alpha} = a_{1}\frac{\frac{b_{3}}{a_{1}}}{\frac{b_{3}}{a_{3}}} = a_{3}.$$

Y como $\frac{b_{i}}{a_{i}} = \alpha \ i=1,2,3$

$$b_{1} = \alpha a_{1},\ \ b_{2} = \alpha a_{2} = \alpha a_{1}j,\ \ b_{3} = \alpha a_{3} = \alpha a_{1}j^{2}.$$ 

Y así, finalmente, $T([x:y:x]) = [a_{1}^{-1}x : (\alpha a_{1})^{-1}y : z ]$ cumple las condiciones de la proposición.

\end{proof}

De la proposición anterior se siguen los siguientes dos corolarios:

\begin{Corolario}
\label{Coro1Prop1}
  Dados dos conjuntos $\mathcal{P}_{i} \neq \mathcal{P}_{k}$, existe un automorfismo $S$ de \CP tal que $S(\mathcal{C'}) = \mathcal{C'}$ y $S(\mathcal{P}_{i}) = \mathcal{P}_{k}$.

\end{Corolario}

\begin{proof}

Por la proposición 1, existen dos automorfismos $R,T$ de $\mathbb{CP^{\mathrm{2}}}$ tales que, $R(\mathcal{C'}) = \mathcal{C} = T(\mathcal{C'})$ y $ R(\mathcal{P}_{i}) = \{ [1:0:0],[0:1:0],[0:0:1] \} = T(\mathcal{P}_{k})$. Entonces, el automorfismo buscado es: $S = T^{-1} \circ R$.

\end{proof}

\begin{Corolario}
\label{Coro2Prop1}
Dados $p_{i}, p_{j} \in \mathcal{P'}$ existe un automorfismo $S$ de \CP tal que $S(\mathcal{C'}) = \mathcal{C'}$ y $S(p_{i}) = p_{j}$. 

\end{Corolario}

\begin{proof}

Primero, observemos que en la transformación que encontramos en la proposición 1, podemos escoger las imágenes de los puntos de $P_{i}$ como queramos. Entonces, basta que en la proposición 1 tomemos $R, T $ de tal manera que $R(p_{i}) = T(p_{j}) = [1:0:0]$ y entonces $S = T^{-1} \circ R$ es el automorfismo que necesitamos.

\end{proof}




	\section{La familia de grado 4.}

     	    \label{sec:FamiliaDeGradoCuatro}
Como ya habíamos mencionado, la familia de grado 4 deja invariantes a todas las rectas de $\mathcal{C}$. Usando la invarianza de las rectas  $\{x=1\}, \{x=j\}, \{x=j^{2}\}, \{y=1\}, \{y=j\}, \{y=j^{2}\}, \{y=x\}, \{y=jx\}, \{y=j^{2}x\}$ obtenemos una ecuación diferencial de la forma:



\begin{align*}
\dot{x} & =(x^{3}-1)(a_{0} + a_{10}x + a_{01}y + a_{20}x^{2} + a_{11}xy + a_{02}y^{2}) \\
\dot{y} & =(y^{3}-1)(b_{0} + b_{10}x + b_{01}y + b_{20}x^{2} + b_{11}xy + b_{02}y^{2}).
\end{align*}



Como el origen es punto singular, $a_{0}=b_{0}=0$. Y como la foliación es de grado cuatro, la parte homogénea de grado cinco de $\dot{x}$ y la de $\dot{y}$ son de la forma $xg(x,y)$ y $yg(x,y)$ respectivamente. Así:


\begin{align*}
\dot{x}& =\cdots+a_{20}x^{5} + a_{11}x^{4}y + a_{02}x^{5}y^{2} =\cdots+x(a_{20}x^{4} + a_{11}x^{3}y + a_{02}x^{2}y^{2}) \\
\dot{y}& =\cdots+b_{20}x^{2}y^{3} + b_{11}xy^{4} + b_{02}y^{5} =\cdots+y(b_{20}x^{2}y^{2} + b_{11}xy^{3} + b_{02}y^{4}) .
\end{align*}

Por lo tanto, $a_{20}=a_{11}=b_{11}=b_{02}=0$ y $a_{02}=b_{20}=b$. Usando lo anterior, nuestra ecuación adquiere la forma:


\begin{align*} 
\dot{x}&=(x^{3}-1)(a_{10}x + a_{01}y + by^{2}) \\
\dot{y}&=(y^{3}-1)(b_{10}x + b_{01}y + bx^{2}).
\end{align*}

Ahora usaremos la invarianza de las tres rectas restantes.

Si $y=x$:

$$ 1 = \frac{dy}{dx} = \frac{b_{10}x + b_{01}x + bx^{2}}{a_{10}x + a_{01}x + bx^{2}}$$
$$\Rightarrow (a_{10} + a_{01})x = (b_{10} + b_{01})x$$
\begin{equation}
\label{temp1}
\Rightarrow a_{10}+a_{01}=b_{10}+b_{01}.
\end{equation}

De manera análoga, en $y=jx$:

$$j = \frac{dy}{dx} = \frac{(b_{10}+b_{01}j)x + bx^{2}}{(a_{10}+a_{01}j)x + bj^{2}x^{2}}$$
\begin{equation}
\label{temp2}
\Rightarrow a_{10}j + a_{01}j^{2} = b_{10} + b_{01}j.
\end{equation}

Por último, al evaluar en $y=j^{2}x$ obtenemos:

\begin{equation}
\label{temp3}
a_{10}j^{2} + a_{01}j = b_{10} + b_{01}j^{2}.
\end{equation}

Sumando (\ref{temp1}), (\ref{temp2}), (\ref{temp3}) y recordando que $1 + j + j^{2} = 0$:

$$(1 + j + j^{2})(a_{10} + a_{01}) = 3b_{10} + (1+j +j^{2})b_{01}$$
$$\Rightarrow b_{10}=0.$$

Y sustituyendo en (\ref{temp1}), $b_{01}=a_{10} + a_{01}$, e insertando esto en (\ref{temp2}):

$$ a_{10}j + a_{01}j^{2 }= (a_{10} + a_{01})j$$
$$ \Rightarrow a_{01}j^{2} = a_{01}j$$
$$ \Rightarrow a_{01} = 0.$$

Y de (\ref{temp1}) nuevamente:

$$a_{10} = b_{01} = a.$$

Usando todo lo anterior la ecuación se ve como:

\begin{align*}
\dot{x} &= (x^{3}-1)(ax + by^{2}) \\
\dot{y} &= (y^{3}-1)(ay + bx^{2}).
\end{align*}

Y al dividir por $a$ y hacer $\alpha = -\frac{b}{a}$:

\begin{equation}
%\tag{$*$}
\label{CampVect4}
\begin{aligned}
\dot{x} &= (x^{3}-1)(x - \alpha y^{2}) \\
\dot{y} &= (y^{3}-1)(y - \alpha x^{2}).
\end{aligned}
\end{equation}

O sí $a = 0$:

\begin{equation}
%\tag{\#}
\label{CampVect4Inf}
\begin{aligned}
\dot{x} &= (x^{3}-1)y^{2} \\
\dot{y} &= (y^{3}-1)x^{2}.
\end{aligned}
\end{equation}

A la foliación generada por la ecuación (\ref{CampVect4}) la denotaremos por $\Fol[4]{\alpha}$ y a la que es generada por (\ref{CampVect4Inf}) la denotaremos $\Fol[4]{\infty}$.\\

Terminaremos esta sección con un lema que nos facilitará muchos cálculos en el futuro:

\begin{Lema}
\label{Lema:Jalando}
Sea $S$ un automorfismo de $\CP$ tal que $S(\mathcal{C})=\mathcal{C}$, entonces $S^{*}(\Fol[4]{\alpha})=\Fol[4]{\beta}$ para alguna $\beta\in\overline\C$.
\end{Lema}

\begin{proof}
Como el grado de la foliación no depende de la carta, $S^{*}(\Fol[4]{\alpha})$ también es una foliación de grado cuatro que deja invariantes a las nueve rectas de la configuración y por lo tanto se puede escribir como aquella generada por los campos vectoriales (\ref{CampVect4}) ó (\ref{CampVect4Inf}).
\end{proof}










  	\section{Los puntos singulares de $\Fol[4]{\alpha}$.}

     	    En esta sección estudiaremos los puntos de singulares $\Fol[4]{\alpha}$. Por la construcción, cada ecuación de la familia es de grado cuatro y por lo tanto tiene 21 puntos singulares contando multiplicidades. Veremos que si $\alpha\notin \{1,j,j^{2},\infty \}$ entonces los 21 puntos son distintos y entonces son no degenerados mientras que si  $\alpha\in \{1,j,j^{2},\infty \}$ sólo los 12 puntos de $\mathcal{P}$ son puntos singulares, de esos 12 puntos, 9 son no degenerados y los tres restantes están en uno de los subconjuntos $\mathcal{P}_{i}$ del lema 1.
\\

Primero, obervemos que los 12 puntos de $\mathcal{P}$ son puntos singulares de $\Fol[4]{\alpha}$, ya que por cada uno de ellos pasan 3 rectas invariantes de la foliación. Como cada recta de $\mathcal{L}$ tiene 4 puntos de $\mathcal{P}$, cada recta tiene al menos 4 puntos singulares, pero como la foliación es de grado cuatro, cada recta invariante tiene, contando multiplicidades, 5 puntos singulares.\\


\begin{table}[h]
\begin{center}
  \begin{tabular}{c|c|c|c|c}
    \ & $(1,\alpha)$ & $(j,\alpha j^{2})$ & $(j^{2},\alpha j)$ & \  \\ \hline 
    $y=j^{2}$ & $(1,j^{2})$ & $(j,j^{2})$ & $(j^{2},j^{2})$ & $(\alpha j,j^{2})$ \\ \hline 
    $y=j$ & $(1,j)$ & $(j,j)$ & $(j^{2},j)$ & $(\alpha j^{2},j)$ \\ \hline 
    $y=1$ & $(1,1)$ & $(j,1)$ & $(j^{2},1)$ & $(\alpha ,1)$ \\ \hline 
    \ & $x=1$ & $x=j$ & $x=j^{2}$ & \ \\
  \end{tabular}
  \caption{Algunos puntos singulares de $\Fol[4]{\alpha}$.}
  \label{Tab:PuntSingGrad4}
\end{center}  
\end{table}


La tabla anterior muestra 4 puntos singulares de 6 rectas de la configuración. En las rectas $y=$cte, el quinto punto singular es $[1 : 0 : 0]$ y en las que $x=$cte, el quinto punto es $[0 : 1 : 0]$. En total, llevamos 17 puntos singulares.
\\

En las tres rectas restantes, el quinto punto singular es $(\frac{1}{\alpha},\frac{1}{\alpha})\in\{ y=x \}$, $(\frac{j}{\alpha}, \frac{j^{2}}{\alpha})\in\{ y=jx \}$ y $(\frac{j^{2}}{\alpha}, \frac{j}{\alpha})\in\{ y=j^{2}x \}$. Si a estos 20 puntos le añadimos el origen, que también es punto singular, tenemos los 21 puntos singulares de $\Fol[4]{\alpha}$.\\

\textbf{Los 21 puntos singulares en el caso $\alpha\notin\{\, 1,j,j^{2},\infty \, \}$.}

Observemos que si $\alpha\notin\{1,j,j^{2},\infty \}$, los 21 puntos singulares son distintos y por lo tanto no degenerados. Así, la parte lineal del campo vectorial en estos puntos tiene dos valores propios $(\lambda_{1},\lambda_{1})$ distintos de cero. Como veremos a continuación, en este caso, los 21 puntos singulares son linealizables, analizaremos primero los 12 puntos de la configuración y luego los 9 puntos singulares restantes.\\

Por cada punto de $\mathcal{P}$ pasan tres rectas invariantes de $\Fol[4]{\alpha}$, entonces, en estos puntos tenemos que $\lambda_{1} = \lambda_{2}$. Por lo tanto, en estos puntos singulares no hay resonancias (ver sección \ref{sec:LinenalizacionDePoincare}) y los valores propios están en el dominio de Poincaré. Entonces, por el teorema de linealización de Poincaré, la ecuación es linealizable en una vecindad de cada uno de estos doce puntos. Lo anterior quiere decir que en una vecindad del punto singular y utlizando coordenadas adecuadas, la ecuación se lee como $ \lambda (u \frac{\partial}{\partial u} + v \frac{\partial}{\partial v})$ y entonces, $\frac{v}{u}$ es una primera integral meromorfa de la ecuación en una vecindad de cada punto de $\mathcal{P}$.
\\

Denotemos por $q_{i}(\alpha) \ i=1,...,9$ a los otros nueve puntos singulares. La parte lineal del campo que representa a $\Fol[4]{\alpha}$  en el punto singular $(1,\alpha )$ es:
\\

$$\begin{pmatrix}

3(1-\alpha^{3}) & 0 \\
-2\alpha(\alpha^{3}-1) & \alpha^{3} -1

\end{pmatrix}$$
\\

\noindent Así, en este punto, $\lambda_{1} = -3\lambda_{2}$. Haciendo un cálculo similar, podemos ver que en los otros ocho puntos singulares los valores propios de la parte lineal cumplen esta misma relación, pero también podemos usar una transformación proyectiva que lleve la recta de la configuración donde se encuentra el punto singular en cuestión a la recta $x=1$ y que además fije a la configuración  ya que el lema \ref{Lema:Jalando} nos asegura que la nueva foliación también pertenece a la familia de grado cuatro.
\\

Por ejemplo, si $q_{i}(\alpha)$ está en la recta $\{ x = j \}$ ó $\{ x = j^{2} \}$ las transformaciones proyectivas $(x,y) \rightarrow (j^{2}x,y)$ y $(x,y) \rightarrow (jx,y)$ respectivamente cumplen lo deseado. Si $q_{i}(\alpha)$ está en una recta de la configuración que pasa por $(0,0)$, a las transformaciones anteriores les anteponemos una transformación como las del corolario 2, que intercambie $(0,0)$ con $[0:1:0]$ y fije a la configuración (cualquier transformación que intercambie $(0,0)$ con $[0:1:0]$ lleva las rectas por el origen en rectas $\{y=cte\}$), y si $q_{i}(\alpha)$ está en una recta de la configuración $\{ y = cte \}$, hacemos lo mismo pero ahora intercambiando a los puntos $[1:0:0]$ y $[0:1:0]$.
\\

Como los valores propios en cada uno de estos nueve puntos cumplen $\lambda_{1} = -3\lambda_{2}$, los valores propios están el dominio de Siegel y por ende, no podemos usar el teorema de linealización de Poincaré. Sin embargo, la ecuación también es linealizable en una vecindad de estos puntos. Para ver esto primero haremos unas definiciones que utilizaremos a lo largo de este capítulo:

\begin{Not}
\label{Notimp}
\begin{enumerate}

\*

\item Llamemos M a la variedad que obtenemos de explotar y resolver los 12 puntos singulares de $\mathcal{P}$ y denotemos por $\Pi\colon M \rightarrow \CP$ al mapeo que resuelve las singularidades.

\item $\tilde{\Fol{\alpha}}$ será la foliación en $M$ inducida por $\Fol[4]{\alpha}$,  $\mathit{i.e.} \ \Pi^{*}(\Fol[4]{\alpha}) = \tilde{\Fol{\alpha}}$.

\item $D_{i}$ va a ser el divisor asociado a $p_{i} \in \mathcal{P} \ i=1,...,12. \ D_{i} = \Pi^{-1}(p_{i})$.

\item Para cada $l_{i} \in \mathcal{L}$ denotaremos por $\tilde{l}_{i} = \overline{\Pi^{-1}(l_{i} \setminus \{p_{i1}, p_{i2}, p_{i3}, p_{i4} \})}$, donde $p_{ik} \ k=1,...,4$ son los cuatro puntos de $\mathcal{P}$ que están en $l_{i}$.

\end{enumerate}
\end{Not}
Ahora sí, como los 12 puntos singulares de $\mathcal{P}$ son radiales, al explotar no obtenemos nuevos puntos singulares en los divisores, así, $\tilde{\Fol{\alpha}}$ sólo tiene un punto singular en $\tilde{l}_{i}$, a saber $\Pi^{-1}(q_{i}(\alpha)) := \tilde{q}_{i}(\alpha)$. Por lo tanto, $\tilde{l}_{i}\setminus q_{i}(\alpha)$, es una hoja de $\tilde{\Fol{\alpha}}$ que es biholomorfa a $\C$ y entonces la holonomía de esta hoja es la identidad y por un teorema de Mattei-Moussu\footnote{Sea $X$ un campo vectorial con dos valores propios $\lambda_{1},\lambda_{2}\neq 0$ tales que $\tfrac{\lambda_{1}}{\lambda_{2}}\notin\mathbb{C}\setminus\mathbb{R}^{+}$. Entonces $X$ es linealizable si y sólo si el grupo de holonomía de una hoja de la foliación determinada por el campo vectorial $X$ es linealizable.} ver \cite[teorema, 2 p.~482]{Mattei-Moussu} $\tilde{\Foli}_{\alpha}$ es linealizable en una vecindad de $q_{i}(\alpha)$.
\\

Como $\Pi$ es un biholomorfismo en una vecindad de $q_{i}(\alpha)$, $\Fol[4]{\alpha}$ también es linealizable en una vecindad de este punto y en coordenadas adecuadas se ve como $3u\frac{\partial}{\partial u} - v\frac{\partial}{\partial v}$ y por lo tanto, $v^{3}u$ es una primera integral en una vecindad de $q_{i}(\alpha)$.
\\

Podemos resumir todo lo anterior en la siguiente proposición:

\begin{Proposicion}
\label{Prop:2}

Si $\alpha \notin \{1,j,j^{2},\infty \}$ entonces los 21 puntos singulares de $\Fol[4]{\alpha}$  son no degenerados. Los 12 puntos de $\mathcal{P}$ son radiales con primera integral meromorfa local $\frac{v}{u} = cte$. Los otros 9 puntos singulares son de tipo silla y tienen una primera integral holomorfa local de la forma $v^{3}u = cte$.

\end{Proposicion}




   
     	    
     	    En el otro caso, cuando $\alpha\in\{1,j,j^{2},\infty \}$ los 9 puntos singulares que no están en $\mathcal{P}$ degeneran en 3 puntos.
\\

Para $\alpha=1$:


$$(1,\alpha),(\alpha,1),(\frac{1}{\alpha},\frac{1}{\alpha}) \rightarrow (1,1).$$

$$(j,\alpha j^{2}),(\alpha j,j^{2}),(\frac{j}{\alpha},\frac{j^{2}}{\alpha}) \rightarrow (j,j^{2}).$$

$$(j^{2},\alpha j),(\alpha j^{2},j),(\frac{j^{2}}{\alpha},\frac{j}{\alpha}) \rightarrow (j^{2},j).$$


Para $\alpha=j$:


$$(1,\alpha),(\alpha j^{2},j),(\frac{j}{\alpha},\frac{j^{2}}{\alpha}) \rightarrow (1,j).$$

$$(\alpha,1),(j,\alpha j^{2}),(\frac{j^{2}}{\alpha},\frac{j}{\alpha}) \rightarrow (j,1).$$

$$(\alpha j,j^{2}),(j^{2},\alpha j),(\frac{1}{\alpha},\frac{1}{\alpha}) \rightarrow (j^{2},j^{}).$$


Para $\alpha=j^{2}$:


$$(1,\alpha),(\alpha j,j^{2}),(\frac{j^{2}}{\alpha},\frac{j}{\alpha}) \rightarrow (1,j^{2}).$$

$$(\alpha j^{2},j),(j,\alpha j^{2}),(\frac{1}{\alpha},\frac{1}{\alpha}) \rightarrow (j,j).$$

$$(\alpha ,1),(j^{2},\alpha j),(\frac{j}{\alpha},\frac{j^{2}}{\alpha}) \rightarrow (j^{2},j^{}).$$


Para $\alpha=\infty$:


$$(1,\alpha), (j,\alpha j^{2}),(j^{2}, \alpha j) \rightarrow [0:1:0].$$

$$(\alpha,1),(\alpha j^{2},j),(\alpha j, j^{2}) \rightarrow [1:0:0].$$

$$(\frac{1}{\alpha},\frac{1}{\alpha}),(\frac{j}{\alpha},\frac{j^{2}}{\alpha}),(\frac{j^{2}}{\alpha},\frac{j}{\alpha}) \rightarrow (0,0).$$
\\

Por lo tanto, si $\alpha\in\{1,j,j^{2},\infty \}$, $\Fol[4]{\alpha}$  tiene 12 puntos singulares, a saber, los 12 puntos de $\mathcal{P}$. Y el análogo a la proposición \ref{Prop:2} es la siguiente proposición:
\\

\begin{Proposicion}
\label{Prop:3}
Si $\alpha\in\{1,j,j^2,\infty\}$ las singularidades de $\Fol[4]{\alpha}$ son los 12 puntos de $\mathcal{P}$, 9 de ellos son de tipo radial y los otros 3 están contenidos en alguno de los conjuntos $\mathcal{P}_i$ de la proposición 1. Además, la foliación tiene una primera integral racional $H_\alpha=\tfrac{P_\alpha}{Q_\alpha}$ donde $P_\alpha$ y $Q_\alpha$ son producto de 3 líneas de $\mathcal{L}$. Estas líneas las podemos escoger de la siguiente manera, si $p_1,p_2,p_3\in\mathit{P}_i$ son las singularidades de $\Fol[4]{\alpha}$ que no son de tipo radial, sean $\mathit{l}_1, \mathit{l}_2, \mathit{l}_3,$ las rectas de $\mathcal{L}$ que pasan por alguno de estos 3 puntos y $\mathit{l}_4, \mathit{l}_5, \mathit{l}_6,$ rectas de $\mathcal{L}$ que pasan por algún otro de esos 3 puntos, entonces $H_\alpha=\tfrac{\mathit{l}_1\mathit{l}_2\mathit{l}_3}{\mathit{l}_4\mathit{l}_5\mathit{l}_6}$.
\end{Proposicion}

\begin{proof}
La única parte de la proposición que falta demostrar es que $\Fol[4]{\alpha}$ tiene una primera integral racional. Primero observemos que si $S$ es una transformación proyectiva como las del corolario \ref{Coro1Prop1}, por el lema \ref{Lema:Jalando}, $S^*(\Fol[4]{\alpha})=\Fol[4]{\beta}$ y además $\beta\in\{1,j,j^2,\infty\}$ pues en caso contrario $S^*(\Fol[4]{\alpha})$ tendría 21 puntos singulares mientras que $\Fol[4]{\alpha}$ tiene solamente 12 puntos singulares.
\\

%Aqui o antes se podría hablar mas de la observacion anterior
Por la observación anterior, basta encontrar una primera integral para $\Fol[4]{\infty}$ y después jalar esta integral a las demás foliaciones.
\\

Sea $H_\infty = \tfrac{y^3-1}{x^3-1} = \tfrac{(y-1)(y-j)(y-j^2)}{(x-1)(x-j)(x-j^2)} = \tfrac{P}{Q}$. Entonces, $\tfrac{dH}{H} = \tfrac{dP}{P} - \tfrac{dQ}{Q} = \tfrac{3y^2dy}{y^3-1} - \tfrac{3x^2dx}{x^3-1} = \tfrac{3}{(y^3-1)(x^3-1)}(x^3-1)y^2dy - (y^3-1)x^2dx$. Por lo tanto, $H_\infty$ es una primera integral de $\Fol[4]{\infty}$ y así, $H_\infty - 1$ y $\tfrac{1}{H_\infty} - 1$ también son primeras integrales de $\Fol[4]{\infty}$:

$$H_\infty = \frac{(y-1)(y-j)(y-j^2)}{(x-1)(x-j)(x-j^2)}$$
$$H_\infty - 1 = \frac{(y-x)(y-jx)(y-j^2x)}{(x-1)(x-j)(x-j^2)}$$
$$\frac{1}{H_\infty} - 1 = \frac{(y-x)(y-jx)(y-j^2x)}{(y-1)(y-j)(y-j^2)}$$

En la primera de las integrales los puntos que se escogieron son $[1:0:0]$ y $[0:1:0]$, en la segunda se escogieron $(0,0)$ y $[0:1:0]$ y en la última $(0,0)$  $[1:0:0]$.
%Al final (osea aqui) puedes explicar rapidamente como le harias en el caso de que alfa sea 1,j o j^2
\end{proof}

Con la proposición \ref{Prop:2} podemos probar que la familia de grado cuatro cumple el primer inciso del teorema principal %\ref{Teo:Prin}
.\\

En efecto, el conjunto $A^{4}$ que le quitaremos a $\overline\C$ para que la familia tenga singularidades no degeneradas de tipo analítico fijo es $A^{4} = \{1,j,j^{2},\infty\}$. La tabla \ref{Tab:PuntSingGrad4} muestra que las singularidades de $\Fol[4]{\alpha}$ se pueden escribir como funciones holomorfas $p_{j}\colon \overline\C\setminus A^{4}\rightarrow\CP$ y las primeras integrales que se obtuvieron en la proposición \ref{Prop:2} muestran que los puntos singulares $p_{j}(s),p_{j}(t)$ son localmente analíticamente equivalentes. 


        \section{La transversalidad de las distintas foliaciones.}

            %Para probar la segunda parte del teorema principal usaremos una curva algebraica que sea transversal a todas las hojas de todas las foliaciones de la familia. Mediante esta tranversal obtendremos un grupo de holonomía. Algunas propiedades de este grupo nos ayudarán a probar la segunda parte del teorema. 

La siguiente porposición nos brinda información de cómo se intersecan hojas de distintas foliaciones de M que son inducidas por $(\Fol[4]{\alpha})_{\alpha\in\overline\C}$ y $\Pi\colon M\rightarrow\CP$, pero antes, recordemos parte de la notación que se usó antes de la proposición \ref{Prop:2}.

\begin{enumerate}


\item Llamemos M a la variedad que obtenemos de explotar y resolver los 12 puntos singulares de $\mathcal{P}$ y denotemos por $\Pi\colon M \rightarrow \CP$ al mapeo que resuelve las singularidades.

\item $\tilde{\Fol{\alpha}}$ será la foliación en $M$ inducida por $\Fol[4]{\alpha}$,  $\mathit{i.e.} \ \Pi^{*}(\Fol[4]{\alpha}) = \tilde{\Fol{\alpha}}$.

\item $D_{i}$ va a ser el divisor asociado a $p_{i} \in \mathcal{P} \ i=1,...,12. \ D_{i} = \Pi^{-1}(p_{i})$.

\item Para cada $l_{j} \in \mathcal{L}$, en el conjunto de las nueve rectas que deja invariante toda foliación de la familia,  denotaremos por $\tilde{l}_{i}$ al transformado estricto bajo $\Pi$ de la recta $l_{i}$. Es decir, $\tilde{l}_{i}= \overline{\Pi^{-1}(l_{i} \setminus \{p_{i1}, p_{i2}, p_{i3}, p_{i4} \})}$, donde $p_{ik} \ k=1,...,4$ son los cuatro puntos de $\mathcal{P}$ que están en $l_{i}$.

\end{enumerate}


\begin{Proposicion}
\label{Prop:Transversalidad}
Si $\alpha\neq\beta$ entonces las foliaciones $\tilde{\Fol{\alpha}}$ y $\tilde{\Fol{\beta}}$ se intersecan transversalmente afuera del conjunto $\tilde{\mathcal{L}} = \tilde{l}_{1}\cup\cdots\cup\tilde{l}_{9}$.
\end{Proposicion}

\begin{proof}
Para demostrar la proposición vamos a usar las expresiones (\ref{CampVect4}) y (\ref{CampVect4Inf}) para mostrar la transversalidad en aquellos lugares donde la transformación $\Pi\colon M\rightarrow \CP$ lo permita. En los lugares donde no podamos hacer esto (los divisores que obtenemos al explotar los doce puntos) vamos a usar las cartas de la explosión para mostrar que las foliaciones se cortan de manera transversal. Supongamos que $\alpha , \beta\neq\infty$, primero probaremos que las foliaciones $\tilde{\Fol{\alpha}},\tilde{\Fol{\beta}}$ son transversales afuera del conjunto $\tilde{\mathcal{L}}\cup D_{1}\cup\cdots\cup D_{12}$. Como $\Pi$ es un biholomorfismo fuera de $D_{1}\cup\cdots\cup D_{12}$, basta demostrar que $\Fol[4]{\alpha},\Fol[4]{\beta}$ son transversales afuera del conjunto $l_{1}\cup\ldots\cup l_{9}$. Si usamos los campos vectoriales que generan a las foliaciones y calculamos su determinante obtenemos:

\begin{equation*}
P_{\alpha}Q_{\beta}-P_{\beta}Q_{\alpha}=(\beta - \alpha)(x^{3}-1)(y^{3}-1)(y^{3}-x^{3}).
\end{equation*}

Todavía nos hace falta probar que son transversales en la recta al infinito. Usando  el corolario \ref{Coro1Prop1} podemos encontrar un automorfismo $S$ de $\CP$ que mande a los puntos $[1:0:0],[0:1:0]$ a otros dos puntos de $\mathcal{P}$ pero que estén en la parte finita y por lo tanto la recta al infinito ahora está en la parte finita. Como las transformaciones que fijan a la configuración cumplen $S^{*}(\Fol[4]{\alpha})=\Fol[4]{\gamma}$, el cálculo anterior muestra que las foliaciones también son transversales en la recta al infinito.\\

Resta ver que $\tilde{\Fol{\alpha}},\tilde{\Fol{\beta}}$ también son transversales en $D_{1}\cup\cdots\cup D_{12}\setminus\tilde{\mathcal{L}}$. Como por cada punto $p_{i}\in\mathcal{P}$ tenemos un divisor $D_{i}$, si usamos el corolario \ref{Coro2Prop1} sólo tenemos que ver que $\tilde{\Fol{\alpha}},\tilde{\Fol{\beta}}$ son transversales en uno de los doce divisores, usemos  $D_{1}=\Pi^{-1}(p_{1}=(0,0))$. Si explotamos el origen usando la carta $(u,x)$ en la cual  $\Pi(u,x)=(x,ux)$, las tres rectas de $\mathcal{L}$ que pasan por el origen ($y=x,y=jx,y=j^{2}x$) se transforman en las rectas $u=1,u=j,u=j^{2}$, el divisor $D_{1}$ queda descrito por $x=0$ y el campo vectorial que genera a $\tilde{\Fol{\alpha}}$ en esta carta es:

\begin{align*}
\dot{u}&= \alpha(u^{3}-1) + xh_{1}(u,x)\\
\dot{x}&= 1 + xh_{2}(u,x).
\end{align*}  

\noindent donde $h_{1}$ y $h_{2}$ son polinomios. Por lo tanto en el divisor $D_{1}$ las pendientes de $\tilde{\Fol{\alpha}},\tilde{\Fol{\beta}}$ son $\tfrac{du}{dx}=\alpha(u^{3}-1)$ y $\tfrac{du}{dx}=\beta(u^{3}-1)$ respectivamente; esto quiere decir que las foliaciones son transversales en $D_{1}\setminus\tilde{\mathcal{L}}$ (si usamos la otra carta, en la cual $\Pi(y,v)=(vy,y)$, obtenemos un resultado análogo).

\noindent Cuando $\beta=\infty$ procedemos de manera análoga, en los lugares donde $\Pi$ es un biholomorfismo tenemos:
\begin{equation*}
P_{\alpha}Q_{\infty}-P_{\infty}Q_{\alpha}=(x^{3}-1)(y^{3}-1)(x^{3}-y^{3}),
\end{equation*}
\noindent este determinante sólo se anula en las nueve rectas que dejan invariantes ambas foliaciones y por lo tanto las foliaciones son transversales fuera de los 12 divisores y las nueve rectas invariantes. Al explotar el punto $(0,0)$ y usar la misma carta $(u,x)$ que antes, la foliación $\tilde{\Foli}_{\infty}$ esta generada por el campo vectorial,
\begin{align*}
\dot{u}=&u^{3}-1\\
\dot{x}=&-(x^{3}-1)u^{2}x.
\end{align*}
\noindent Esta última expresión muestra que el punto $(0,0)$ es un punto no dicrítico pues el divisor $D_{1}$ es invariante para la foliación $\tilde{\Foli}_{\infty}$ mientras que para la foliación $\tilde{\Foli}_{\alpha}$, el punto $(0,0)$ es dicrítico y por lo tanto las foliaciones son transversales. 
\end{proof}


La foliación $\Fol{\infty}$ tiene por primera integral a $H_{\infty}=\tfrac{y^{3}-1}{x^{3}-1}$ (proposición \ref{Prop:3}), si usamos coordenadas homogéneas, las hojas de la foliación quedan descritas por $L_{c}=\{\tfrac{y^{3}-z^{3}}{x^{3}-z^{3}}=c\}$. Para $c=0$, $L_{c}=\{(y-z)(y-jz)(y-j^{2}z)=0\}$ y entonces el punto $[1:0:0]$ es un punto singular. Si $c=\infty$, $L_{c}=\{(x-z)(x-jz)(x-j^{2}z)=0\}$ y así, $[0:1:0]$ es un punto singular. Para $c=1$, $L_{c}=\{(y-x)(y-jx)(y-j^{2}x)=0\}$ y entonces $[0:1:0]$ es otro punto singular. Para todas las demás $c\in\C\setminus\{\, 0,1\, \}$, $L_{c}=\{y^{3}-z^{3}-c(x^{3}-z^{3})=0\}$ y calculando el gradiente correspondiente, es fácil ver que estas curvas son suaves y como todas están dadas por un polinomio de grado tres entonces tienen género uno (ver \cite{FischerGerd}).\\

Observemos que los doce puntos singulares radiales están contenidos en las curvas de nivel singulares de $H_{\infty}$, por lo tanto, si consideramos la función $h=H_{\infty}\circ\Pi\colon M\rightarrow\overline{\C}$ y tomamos un valor regular $c\in\overline{\C}\setminus(0,1,\infty)$ de la función $H_{\infty}$, entonces $h^{-1}(c)$ es topológicamente equivalente a un toro.

\begin{Obs}
\label{Obs:FibrasSingulares}
El conjunto $h^{-1}(1)$ está formando por el divisor $D_{1}$ que resulta de explotar el punto $(0,0)$ y el transformado estricto bajo $\Pi$ de las rectas $\{\, (y-x)(y-jx)(y-j^{2}x)=0\, \}$. En efecto, $H_{\infty}^{-1}(1)=\{\, (y-x)(y-jx)(y-j^{2}x)=0\, \}$ y como se vio al final de la proposición \ref{Prop:Transversalidad} el divisor $D_{1}$ es invariante para la foliación $\tilde{\Fol{\infty}}$ y este divisor se proyecta bajo $\Pi$ al punto $(0,0)\in H_{\infty}^{-1}(1)$. Los otros puntos singulares que explota la transformación $\Pi$ y que están en las rectas $\{\, (y-x)(y-jx)(y-j^{2}x)=0\, \}$ son puntos singulares radiales (ver proposición \ref{Prop:3}) y por lo tanto al explotar estos puntos, las hojas de la foliación $\tilde{\Fol{\infty}}$ cortan de manera transversal a los divisores correspondientes y así, los divisores asociados a estos puntos singulares no están contenidos en $h^{-1}(1)$. De manera análoga $h^{-1}(0)$ y $h^{-1}(\infty)$, (cada uno de ellos) está formado por otro divisor junto con el transformado estricto de otras tres rectas que deja invariante cualquier foliación de la familia. Es decir, $h^{-1}(0,1,\infty)=D_{1}\cup D_{2}\cup D_{3}\cup_{i=1}^{9}\tilde{l_{i}}$, donde $D_{1},D_{2},D_{3}$ son los divisores que se obtienen al explotar los puntos (en coordenadas homogéneas) $[0:0:1][0:1:0][1:0:0]$ respectivamente y $\tilde{l}_{i}$ es el transformado estricto de alguna de las nueve rectas que deja invariante cualquier foliación $\Fol{\alpha}$ de la familia.
\end{Obs}

 Si $V=M\setminus h^{-1}(0,1,\infty)$ y $\Omega=\C\setminus\{0,1\}$ entonces $h\colon V\rightarrow\Omega$ satisface las hipótesis del lema \ref{Lema:HazTopologico} y por lo tanto, $(V,h,\Omega)$ es un haz topológico con fibra $T$ isomorfa a un toro. Como $\tilde{\Foli_{\alpha}^{4}}$ es transversal a $\tilde{\Foli_{\infty}^{4}}$, si fijamos una fibra no singular $T_{a}=h^{-1}(a)$ obtenemos el grupo de holonomía global $G_{\alpha}$ de $T_{\alpha}$ asociado a la foliación $\tilde{\Foli_{\alpha}^{4}}$ y al haz fibrado $(V,h,\Omega)$ (ver sección \ref{Sec:HolonomiaGlobal}). Como el grupo fundamental de $\Omega=\C\setminus\{0,1\}$ está generado por dos lazos $\gamma_{1}$ y $\gamma_{2}$, el grupo de holonomía global $G_{\alpha}$ está generado por dos elementos $f_{\alpha}$ y $g_{\alpha}$. Antes de continuar veamos quién es la fibra de este haz topológico.
\begin{Proposicion}
\label{Prop:FibraDelHaz}
Sea $(V,h,\Omega)$ el haz topológico de arriba, si $\Gamma=\mathbb{Z}\oplus j\mathbb{Z}$ entonces, $h^{-1}(a)\simeq\C/\Gamma$.
\end{Proposicion}
\begin{proof}
Si fijamos una foliación $\tilde{\Foli_{\alpha}^{4}}$ ,como el grupo fundamental de $\Omega$ está generado por dos elementos ${\gamma_{1},\gamma_{2}}$, entonces  el grupo de holonomía $G_{\alpha}$ está generado por dos biholomorfismos $f_{\alpha},g_{\alpha}\colon T_{a}\rightarrow T_{a}$. La foliación $\tilde{\mathcal{F}}_{\alpha}^{4}$ sólo tiene nueve puntos singulares y de manera local, alrededor de cada uno de estos puntos, la foliación está generada por el campo vectorial $3u\tfrac{\partial}{\partial u}-v\tfrac{\partial}{\partial v}$ (proposición \ref{Prop:2}). Así, la transformación de holonomía asociada a la separatriz local $\{v=0\}$ es $v\mapsto e^{\tfrac{2\pi i}{3}}v=jv$ y tiene orden tres. Si tomamos a $T_{a}$ como una transversal a la separatriz local $\{v=0\}$, entonces $f_{\alpha}^{3}=g_{\alpha}^{3}=Id$ y tanto $f_{\alpha}$ como $g_{\alpha}$ fijan un punto. Un toro que tenga un automorfismo con las propiedades anteriores es biholomorfo a $\C/\Gamma$ donde $\Gamma=\mathbb{Z}\oplus j\mathbb{Z} \}$ (ver apéndice \ref{ApendiceToros}).
\end{proof}

\section{El grupo de holonomía global de $\tilde{\Foli_{\alpha}^{4}}$.}

Para probar que una foliación $\Foli_{\alpha}^{4}$ tiene una primera integral racional vamos a usar agunas propiedades del grupo de holonomía global que se construyó en la sección anterior. En esta sección probaremos algunas propiedades de este grupo. 

\begin{Lema}
\label{Lema:GrupoHolonomia}
Sea $G_{\alpha}$ el grupo de holonomía global asociado al haz fibrado $(V,h,\Omega)$ y la foliación $\tilde{\Foli_{\alpha}^{4}}$. Entonces en una carta adecuada de $\C$, $G_{\alpha}$ está generado por $f_{\alpha}(z)=jz$ y $g_{\alpha}(z)=jz+A(\alpha)$.
\end{Lema}
\begin{proof}
Si fijamos coordenadas de una cubierta universal $P\colon \C\rightarrow T_{a}$ entonces en esta cubierta, $f_{\alpha}(z)=jz+a(\alpha)$ y $g_{\alpha}=jz+b(\alpha)$ (ver apéndice \ref{ApendiceToros}). Notemos que $z_{0}(\alpha):=\tfrac{a(\alpha)}{1-j}$ es un punto fijo de $f_{\alpha}$, por lo tanto, si $k_{\alpha}(z)=z+z_{0}(\alpha)$ entonces $k_{\alpha}^{-1}\circ f_{\alpha}\circ k_{\alpha}(z)=jz$. En estas nuevas coordenadas, $k_{\alpha}^{-1}\circ g_{\alpha}\circ k_{\alpha}(z)=jz+b(\alpha)-a(\alpha)=jz+A(\alpha)$. 
\end{proof}

A continuación veremos cómo se ve un subrgupo de biholomorfismos de $\C/\Gamma$ generado por dos elementos $f(z)=jz$ y $g(z)=jz+A$.

\begin{Proposicion}
\label{Prop:DescripcionDelGrupo}
Sea $G$ un subgrupo de biholomorfismos de $\C/\Gamma$ generado por $f(z)=jz$ y $g(z)=jz+A$, entonces $G=\{j^{i}z+A;\, i\in\{0,1,2\},\, d\in\mathbb{Z}\oplus j\mathbb{Z} \}$.
\end{Proposicion}
\begin{proof}
Si $G_{1}=\{j^{i}z+A;\, i\in\{0,1,2\},\, d\in\mathbb{Z}\oplus j\mathbb{Z} \}$, es claro que $G\subset G_{1}$. Para demostrar que $G_{1}\subset G$ basta ver que $z+A$ y $z+jA$ están en $G$ pues en este caso, si $d=m+jn$ tenemos que $\mu(z)=z+dA=z+(m+nj)A=z+mA+njA\in G$ y si a $\mu$ le anteponemos $f$ y $f^{2}$ obtenemos las transformaciones $jz+dA$ y $j^{2}z+dA$ respectivamente. Así pues, $g\circ f^{2}=j(j^{2}z)+A=z+A$ y $f\circ g\circ f=j(j^{2}z+A)=z+jA$
\end{proof}

Las siguientes equivalencias nos van a ayudar a probar la existencia de primeras integrales racionales.

\begin{Corolario}
\label{Coro:Equivalencias}
Sea $G$ un grupo como el de la proposición anterior, entonces son equivalentes:
\begin{enumerate}
\item $G$ es finito.
\item $G$ tiene órbita finita en $\C/\Gamma$.
\item Existe $n\in\mathbb{N}$ tal que $nA\in\Gamma$.
\end{enumerate}
\end{Corolario}
\begin{proof}
$1\Rightarrow 2$. Es claro.

\noindent $2\Rightarrow 3$ Sea $\mu(z)=z+A$, fijemos $z_{0}\in\C/\Gamma$ y consideremos al conjunto $\{z_{0},\mu(z_{0}),\mu^{2}(z_{0}),\ldots \}$. Como la órbita de $z_{0}$ es finita tienen que existir $m,n\in\mathbb{N},\, m<n$ tales que $\mu^{m}(z_{0})=\mu(z_{0})^{n}$ (mod $\Gamma$). Esto quiere decir que $mA=nA$ (mod $\Gamma$) y por lo tanto $(n-m)A\in\Gamma$.

\noindent $3\Rightarrow 1$. Para probar este úlitmo inciso y para uso posterior, vamos a calcular explícitamente el número de elementos de $G$. Sea $k\in\mathbb{N}$ el natural más pequeño que satisface $kA\in\Gamma$ y $\mu(z)=j^{i}z+dA\in G$, encontraremos una manera equivalente de escribir a $\mu$. Si $d=(p+jq)$ y dividimos a $p$ y $q$ entre $k$ obtenemos $p=ka +\tilde{p}$ y $q=kb+\tilde{q}$ con $0\leq\tilde{p},\tilde{q}<k$ y entonces podemos escribir a $\mu$ como $\mu(z)=j^{i}+k(a+jb)A+(\tilde{p}+j\tilde{q})A$; módulo $\Gamma$, $k(a+jb)A=0$, puesto que $kA\in\Gamma$ por lo tanto, $\mu(z)$ induce la misma transformación que $\tilde{\mu}(z)=j^{i}z+(\tilde{p}+j\tilde{q})A$ y además $0\leq\tilde{p},\tilde{q}<k$. Lo anterior quiere decir que toda transformación $\mu(z)=j^{i}z+dA$ la podemos escribir con una $d=m+jn$ tal que $0\leq m,n<k$. Con esto no es difícil convencerse de que $G$ tiene $3k^{2}$ elementos.
\end{proof}

Si $\tilde{L}$ es una hoja de la foliación $\tilde{F}_{\alpha},\, \alpha\neq\infty$, la construcción del grupo de holonomía global (ver sección \ref{Sec:HolonomiaGlobal}) nos asegura que la función $h\colon \tilde{L}\setminus h^{-1}(0,1,\infty)\rightarrow\C\setminus\{\, 0,1 \, \}$ es una aplicación cubriente. Para ver bajo qué condiciones la hoja $L=\Pi(\tilde{L})$ es algebraica, vamos a necesitar una expresión local de la función $h\colon \tilde{L}\rightarrow\overline{\C}$ alrededor de los puntos $p\in\tilde{L}\cap h^{-1}(0,1,\infty)$.

\begin{Lema}
\label{Lema:ExpresionLocalDeLaAplicacionCubriente}
Sea $\tilde{L}$ una hoja de la foliación $\tilde{F}_{\alpha}, \alpha\neq\infty$ que no pasa por ninguno de los nueve puntos silla $\tilde{q}_{i}(\alpha)$ de la foliación, entonces, alrededor de cada punto $p\in\tilde{L}\cap h^{-1}(0,1,\infty)$ existe una carta local $z$ alrededor de $p$ tal que en esta carta la función $h\colon \tilde{L}\rightarrow\overline{\C}$ adquiere la expresión $h(z)=1+z^{3}$.
\end{Lema}

\begin{proof}
Como hemos venido haciendo, basta probar la afirmación para $p\in h^{-1}(1)=D_{1}\cup\tilde{l}_{1}\cup\tilde{l}_{2}\cup\tilde{l}_{3}$ (ver observación \ref{Obs:FibrasSingulares}) y despúes usar el corolario \ref{Coro2Prop1} y el lema \ref{Lema:Jalando} para ver que la afirmación también se vale en cada punto del conjunto $\tilde{L}\cap h^{-1}(0,\infty)$. Como la hoja $\tilde{L}$ no pasa por ninguno de los nueve puntos silla $\tilde{q}_{i}\in\tilde{l}_{i}$ y las rectas $\tilde{l}_{i}$ son invariantes para la foliación $\tilde{\Fol{\alpha}}$, entonces $\tilde{L}\cap h^{-1}(1)\subset D_{1}$. Usando la carta $(x,u)$ de $M$ en la cual $\Pi(x,u)=(x,ux)$ tenemos que $h(x,u)=H_{\infty}\circ\Pi(x,u)=\tfrac{u^{3}x^{3}-1}{x^{3}-1}$ y así, $h(x,u)-1=\tfrac{x^{3}(u^{3}-1)}{x^{3}-1}$. En esta carta el divisor $D_{1}$ queda descrito por $\{\, x=0\, \}$ y como $\tilde{L}$ interseca de manera transversal al divisor $D_{1}$ (proposición \ref{Prop:Transversalidad}) podemos parametrizar a $\tilde{L}$ alrededor de $p\in\tilde{L}\cap h^{-1}(1)$ con una función $x\mapsto (x,u(x))$ donde $u\colon (\C,0)\rightarrow\C$ es holomorfa. Así, $h-1\colon \tilde{L}\rightarrow\overline{\C}$ adquiere la expresión $h(x,u(x))-1=\tfrac{x^{3}((u(x))^{3}-1)}{x^{3}-1}$ alrededor del punto $p$. Como $u(0)\notin\{\, 1,j,j^{2}\, \}$ pues estos valores de $u$ se corresponden con las intersecciones del divisor $D_{1}$ con las rectas $\tilde{l}_{i}$, concluimos que la función $h-1\colon \tilde{L}\rightarrow\overline{\C}$ tiene orden tres alrededor del punto $p\in\tilde{L}\cap h^{-1}(1)$ y por lo tanto existe una carta coordenada $z$ alrededor de $p\in\tilde{L}$ tal que, $h(z)=1+z^{3}$. 
\end{proof}

\begin{Proposicion}
\label{Prop:EquivalenciasIntegrabilidad}
Sea $G_{\alpha}=\langle jz,jz+A(\alpha)\rangle$ el grupo de holonomía global asociado a $\tilde{F}_{\alpha}$, entonces son equivalentes:
\begin{enumerate}
\item $\Foli_{\alpha}^{4}$ tiene una primera integral racional.
\item $\Foli_{\alpha}^{4}$ tiene una hoja algebraica que no está contenida en las líneas invariantes de $\mathcal{L}$.
\item Existe un natural $n\in\mathbb{N}$ tal que $nA(\alpha)\in\Gamma$.
\end{enumerate}
\end{Proposicion} 
\begin{proof}
$1\Rightarrow 2$. Es claro.

\noindent $2\Rightarrow 3$ Sea $L$ una hoja algebraica de $\Foli_{\alpha}^{4}$ que no está contenida en las líneas de $\mathcal{L}$. Por el teorema de Bézout tenemos que $L\cap H^{-1}(a)$ es un conjunto finito. Si $\tilde{L}$ es el transformado estricto de $L$ ($\mathit{i.e.}$ $\tilde{L}=\overline{\Pi^{-1}(L\setminus\mathcal{P})}$ donde $\mathcal{P}$ son los puntos que explota $\Pi^{-1}$), entonces $\tilde{L}\cap T_{a}$ es un conjunto finito pues $T_{a}=\Pi^{-1}(H^{-1}(a))$, pero $\tilde{L}\cap T_{a}$ es una órbita de $G_{\alpha}$ y por el corolario anterior, existe $n\in\mathbb{N}$ tal que $nA(\alpha)\in\Gamma$.

\noindent $3\Rightarrow 1$ Sea $\tilde{L}$ una hoja de $\tilde{\Foli}_{\alpha}$ tal que $\Pi(\tilde{L})$ no está contenida en las nueve rectas de la configuración y tampoco pasa por los nueve puntos silla $q_{i}(\alpha)$, veremos que podemos cubrir a $\tilde{L}$ con una cantidad finita de abiertos relativamente compactos y por lo tanto $\tilde{L}$ es compacta. Por el lema anterior, en cada punto $p_{i}\in\tilde{L}\cap h^{-1}(0,1,\infty)$ existe una bola $B_{i}$ tal que $h(B_{i})$ es abierto en $\overline{\C}$ y como $h\colon\tilde{L}\setminus{h^{-1}(0,1,\infty)}\rightarrow\C\setminus\{\, 0,1\, \}$ es una aplicación cubriente, para cada punto $x\in\tilde{L}\setminus h^{-1}(0,1,\infty)$ existe un abierto $U_{x}$ relativamente compacto tal que $h(U_{x})$ es abierto en $\overline{\C}$. Los conjuntos $h(B_{i})$ y $h(U_{x})$ forman una cubierta abierta de $\overline{\C}$ así que podemos extraer una subcubierta finita $\{\, V_{i} \, \}$ (donde $V_{i}=h(U_{x})$ ó $V_{i}=h(B_{k})$). Por hipótesis $G_{\alpha}$ tiene órbita finita en $\C/\Gamma$ así que existe un $m\in\mathbb{N}$ tal que $\#\tilde{L}\cap h^{-1}(b)\leq m$ para toda $b\in \C\setminus\{\, 0,1,\infty\,\}$ (ver observación \ref{Obs:Interseccion}). Por continuidad lo mismo sucede con las intersecciones de $h^{-1}(0),\, h^{-1}(1)$ y $h^{-1}(\infty)$ con $\tilde{L}$. Así, para un conjunto abierto $V_{i},\, V_{i}=h(U_{x})$, $h^{-1}(V_{i})$ está formado por un número finito de abiertos relativamente compactos y por lo tanto $h^{-1}(V_{i})$ también es relativamente compacto. Los conjuntos $h^{-1}(V_{i})$ junto con las bolas $B_{i}$ que son un número finito, forman una cubierta finita de $\tilde{L}$ por abiertos relativamente compactos. Una subvariedad compacta de una variedad proyectiva es algebraica (referencia...) y por lo tanto la hoja $L=\Pi(\tilde{L})$ de la foliación $\Foli_{\alpha}$ es algebraica. Como la foliación $\mathcal{F}_{\alpha}$ tiene un infinidad de hojas algebraicas, tiene una primera integral racional (teorema \ref{Teo:Darboux}).
\end{proof}

        \section{El grado de la primera integral de $\Foli_{\alpha}$.}    
            En esta sección veremos que existe un subconjunto denso $E$ de parámetros $\alpha\in\overline{\C}$ tal que si $\alpha\in E$, entonces la foliación $\Foli_{\alpha}$ tiene una primera integral racional. En caso de que dicha primera integral exista sabemos que el grupo de holonomía global $G_{\alpha}$ de la foliación $\tilde{\Foli}_{\alpha}=\Pi^{*}(\Foli_{\alpha})$ es finito (proposición \ref{Prop:EquivalenciasIntegrabilidad} y corolario \ref{Coro:Equivalencias}). Vamos a acotar por abajo el grado de la primera integral en términos de la cardinalidad del grupo de holonomía global y mostraremos que para una natural $k_{0}$ fijo, los parámetros $\alpha\in E$ tales que la primera integral de la foliación $\Foli_{\alpha}$ es de grado menor que $k_{0}$, forman un subonjunto finito.\\

Por la proposición \ref{Prop:EquivalenciasIntegrabilidad} sabemos que la foliación $\Foli_{\alpha}$ tiene primera integral racional si y sólo si existe $n\in\mathbb{N}$ tal que $nA(\alpha)\in\mathbb{Z}\oplus j\mathbb{Z}=\Gamma$, donde $g_{\alpha}(z)=jz+A(\alpha)$ es uno de los dos generadores del grupo de holonomía global $G_{\alpha}$. La condición $nA(\alpha)\in\mathbb{Z}\oplus j\mathbb{Z}$ es equivalente a que $A(\alpha)\in\mathbb{Q}\oplus j\mathbb{Q}$. El conjunto $A(\alpha)\in\mathbb{Q}\oplus j\mathbb{Q}$ es un subconjunto denso del toro $\C/\Gamma$. Veremos que la función $A\colon\C\rightarrow\C/\Gamma$ es holomorfa, no constante y por lo tanto localmente abierta. Esto implica que la preimagen de $\mathbb{Q}\oplus j\mathbb{Q}$, es densa en $\C$. Así pues, vamos a estudiar algunas propiedades de la función $A\colon\C\rightarrow \C/\Gamma$.\\

Primero veamos que $A\colon\C\rightarrow\C/\Gamma$ es holomorfa. La transformación $g_{\alpha}=jz+A(\alpha)$ está generada por un lazo $\gamma\in\pi_{1}(\C\setminus\{0,1\},a)$. Si fijamos este lazo obtenemos una función $\Phi_{\gamma}\colon\C\times T_{a}\rightarrow T_{a}$, donde $T_{a}=h^{-1}(a)\simeq\C/\Gamma$ es una transversal a todas las foliaciones $\tilde{\Foli}_{\alpha},\, \alpha\in\C$. Esta función está definida como $\Phi_{\gamma}(\alpha,p)=g_{\alpha}$, esto quiere decir que para $\alpha\in\C$ fijo, $\Phi_{\gamma}(\alpha,z)=jz+A(\alpha)$ es uno de los dos generadores del grupo de holonomía global $G_{\alpha}$. Como las soluciones de la ecuación diferencial que genera a cada una de estas foliaciones en una vecindad de cada punto dependen de manera holomorfa de las condiciones iniciales y del parámetro $\alpha$ (teorema \ref{Teo:ExistenciaUnicidad}), tenemos que $A(\alpha)$ es holomorfa.\\

Para ver que $A(\alpha)$ es no constante veremos que los grupos de holonomía de las foliaciones $\tilde{\Foli}_{0}$ y $\tilde{\Foli}_{1}$ tienen distinto número de elementos y por lo tanto $A(0)\neq A(1)$.\\

Por la proposición \ref{Prop:2} tenemos que $$H_{1}=\frac{(x-j^{2})(y-j)(y-j^{2}x)}{(x-j)(y-j^{2})(y-jx)}=\frac{p_{1}}{q_{1}},$$ es una primera integral racional de $\Foli_{1}$. Por el teorema de Bézout tenemos que $\{p_{1}=0\}$ interseca a $\{q_{1}=0\}$ en nueve puntos y cada uno de estos puntos está en todas las curvas de nivel de $H_{1}$. Con un razonamiento similar, la curva de nivel $H_{\infty}^{-1}(a)=\Pi(T_{a})$ siempre pasa por los nueve puntos de intersección de $\{p_{\infty}=y^{3}-1=0\}$ y $\{q_{\infty}=x^{3}-1=0\}$. Con esto no es difícil convencerse (ver figura...) que $H^{-1}_{1}(c)$ interseca a $H^{-1}_{\infty}(a)$ en los seis puntos $(1,j),(1,j^{2}),(j,1),(j,j),(j^{2},1)$ y $(j^{2},j^{2})$. Así, de los nueve puntos de intersección de $H^{-1}_{1}(c)$ y $H^{-1}_{\infty}(a)$, sólo tres de ellos no son alguno de los doce puntos que explota la transformación $\Pi\colon M\rightarrow\CP$. Las intersecciones en cada uno de estos seis puntos es transversal y entonces, al explotar, estas intersecciones desaprecen y por lo tanto el transformado estricto de $H^{-1}_{1}(c)$ bajo $\Pi$ (que es una hoja de la foliación $\tilde{\Foli_{1}}$) interseca a $T_{a}=\Pi^{-1}(H^{-1}_{\infty}(a))$ en tres puntos. Esto implica que el grupo de holonomía global $G_{1}$ sólo tiene tres elementos y así $f_{1}(z)=g_{1}(z)=jz$. Por lo tanto, $A(0)=0$.\\

A continuación veremos que $A(0)\neq 0$ (mod $\Gamma$). Notemos primero que $$H_{0}(x,y)=\frac{x^{3}(y^{3}-1)}{y^{3}(x^{3}-1)}=\frac{p_{0}}{q_{0}},$$ es una primera integral racional de $\Foli_{0}$. En efecto, $$q_{0}^{2}dH_{0}=q_{0}dp_{0}-p_{0}dq_{0}=3x^{2}y^{2}[x(x^{3}-1)\, dy-y(y^{3}-1)\, dx].$$

La recta $\{x=0\}$ que es parte de la curva de nivel cero de $H_{0}$ corta a $H_{\infty}(a)$ en tres puntos $(0,y_{k})$, donde $y_{1},y_{2},y_{3}$ son las raíces de $a=y^{3}-1$. $H_{0}^{-1}(0)$ tiene multiplicidad tres a lo largo de la recta $\{x=0\}$ así que para $b$ cercana a $0$, $H_{0}^{-1}(b)$ interseca a $H_{\infty}^{-1}(a)$ en nueve puntos  pues cada uno de los puntos de intersección $(0,y_{k})$, al variar $b$, se separa en tres puntos cercanos a $(0,y_{k})$ y además, para $b$ suficientemente pequeña ninguno de estos nueve puntos coincide con los puntos que explota $\Pi$. Por lo tanto, al explotar, el transformado estricto de $H_{0}^{-1}(b)$ interseca a la tansversal $T_{a}=h^{-1}(a)$ en al menos nueve puntos y así, el grupo $G_{0}$ tiene al menos nueve elementos. Esto quiere decir que $A(0)\neq 0$. Ahora podemos probar el siguiente teorema.

\begin{Teorema}
\label{Teo:DensidadPrimerasIntegrales}
Para la familia $\Foli_{\alpha}$ de foliaciones de $\CP$ existe un subconjunto denso de parámetros $E\subset\overline{\C}$ tal que si $\alpha\in E$, la foliación $\Foli_{\alpha}$ tiene una primera integral racional.
\end{Teorema}
\begin{proof}
Como la función $A\colon\C\rightarrow\C/\Gamma$ es holomorfa y no constante, $E=A^{-1}(\mathbb{Q}\oplus j\mathbb{Q})$ es un subconjunto denso de $\C$. Si $\alpha\in\ E$ tenemos que $A(\alpha)\in\mathbb{Q}\oplus j\mathbb{Q}$ y por lo tanto existe $n\in\mathbb{N}$ tal que $nA(\alpha)\in\Gamma=\mathbb{Z}\oplus j\mathbb{Z}$. De acuerdo al teorema \ref{Prop:EquivalenciasIntegrabilidad} la foliación $\Foli_{\alpha}$ tiene una primera integral racional.
\end{proof}

Si $\alpha\in E\subset\overline{\C}$ entonces la foliación $\Foli_{\alpha}$ tiene una primera integral racional de grado $d(\alpha)$ y para el grupo de holonomía global $G_{\alpha}$ de la foliación $\tilde{\Foli}_{\alpha}$ existe un $k=k(\alpha)\in\mathbb{N}$ tal que  $G_{\alpha}$ tiene $3k(\alpha)^{2}$ elementos (corolario \ref{Coro:Equivalencias} y proposición \ref{Prop:EquivalenciasIntegrabilidad}).

\begin{Lema}
\label{Lema:Cota}
Si el grupo de holonomía global $G_{\alpha}$ tiene $k(\alpha)^{2}$ elementos, entonces el grado $d(\alpha)$ de la primera integral de la foliación $\Foli_{\alpha}$ cumple $k(\alpha)^{2}\leq d(\alpha)$.
\end{Lema} 
\begin{proof}
Sea $L$ una hoja de la foliación $\Foli_{\alpha}$ y $\tilde{L}$ su transformado estricto bajo $\Pi$, por el teorema de Bézout tenemos que,
$$3d(\alpha)=\#\{L\cap H_{\infty}^{-1}(a)\}\geq\#\{\tilde{L}\cap h_{\infty}^{-1}(a)\}=\#G_{\alpha}=3k(\alpha)^{2}.$$
\end{proof}

\begin{Teorema}
\label{Teo:GradosGrandes}
Dado un natural $n_{0}\in\mathbb{N}$ el subconjunto $E_{n_{0}}:=\{\alpha\in E;\, k(\alpha)\leq n_{0}\, \}\cup\{\infty \, \}$ es finito. Es decir, las foliaciones $\Foli_{\alpha}$ de $\CP$ con primera integral racional de grado $d(\alpha)\leq n_{0}$ son un número finito. 
\end{Teorema}
\begin{proof}
Si el conjunto $E_{n_{0}}$ fuera infinito tendríamos que para algún número $r\in\{\, 1,\ldots,n_{0} \, \}$, el conjunto $E_{r}:=\{\, \alpha\in E_{n_{0}};\, k(\alpha)=p \, \}\cup\{\, \infty\, \}$ es infinito. Del corolario \ref{Coro:Equivalencias} tenemos que para todo $\alpha,\beta\in E_{r}$, $A(\alpha)=A(\beta)$ (mod $\Gamma$) y entonces el subconjunto $E_{r}$ tiene una sucesión convergente $\{\, a_{i} \, \}_{i\in\mathbb{N}}$ tal que $A(a_{i})=A(\alpha)=A(\beta)$ para toda $i\in\mathbb{N}$ esto implica que $A(z)$ es constante lo cual es una contradicción.
\end{proof}

\section{El género de las hojas de $\Fol{\alpha}$.}

Cuando una curva algebraica es suave hay una conocida fórmula que relaciona el grado de la curva algebraica con su género. En el caso de que la curva algebraica $C$ sea singular, debemos desingularizarla para así obtener una superficie de Riemann compacta $S$ y una aplicación holomorfa $\varphi\colon S\rightarrow C$ que es un biholomorfimso fuera de $S\setminus\varphi^{-1}(sing(C))$, donde $sing(C)$ es el conjunto de puntos singulares de la curva $C$ y luego calcular el género de la superficie $S$.

En nuestro caso, si $\alpha\in\{\, 1,j,j^{2},\infty \, \}$ las proposiciones \ref{Prop:3} y \ref{Prop:FibraDelHaz} nos dicen que las hojas irreducibles que no están contenidas en las nueve líneas que deja invariante la foliación $\Fol{\alpha}$ son toros biholomorfos a $\C/\mathbb{Z}\oplus j\mathbb{Z}=\C/\Gamma$. Cuando $\alpha\notin\{\, 1,j,j^{2},\infty \, \}$ la proposición \ref{Prop:2} nos asegura que al explotar mediante $\Pi\colon M\rightarrow \CP$ a los doce puntos radiales de la foliación $\Fol{\alpha}$, cualquier hoja $L$ de la foliación que no pase por ninguno de los nueve puntos silla $q_{i}(\alpha)$ se transforma en una superficie de Riemann $\tilde{L}$. En el caso particular en que la foliación $\Fol{\alpha}$ tiene primera integral racional ($\alpha\in E$),  vamos a construir una aplicación cubriente $\pi\colon \tilde{L}\rightarrow \C/\Gamma$ donde $\tilde{L}$ es una hoja ``genérica'' de la foliación $\tilde{\Fol{\alpha}}$. Esto implica que la hoja $L=\Pi({\tilde{L}})$ de la foliación $\Fol{\alpha},\, \alpha\in E$ tiene género uno pues por la proposición \ref{Prop:EquivalenciasIntegrabilidad} la hoja $\tilde{L}$ es una superficie de Riemann compacta.\\

Para poder definir la aplicación cubriente vamos necesitar algunas definiciones. 
\begin{defn}
\label{Def:PuntoGenerico}
Sea $G=\langle jx,jx+A\rangle$ un subrgupo de biholomorfismos de $\C/\Gamma$ generado por dos elementos, decimos que $b\in\C/\Gamma$ es \emph{genérico} para el grupo $G$ si el único elemento $f\in G$ que fija a $b$ es la identidad. Es decir, $b\in\C/\Gamma$ es genérico si su estabilizador, $G_{b}$ es trivial.
\end{defn}

En el caso particular en que $\alpha\in E$, el grupo $G_{\alpha}$ que actúa en $\C/\Gamma$ es finito, así, hay un número finito de estabilizadores y cada estabilizador fija un número finito de puntos. Por lo tanto, el conjunto de puntos de $\C/\Gamma$ que no son genéricos para $G_{\alpha}$ es un conjunto finito y así, el conjunto de puntos genéricos es denso y no numerable.

\begin{defn}
\label{Def:HojaGenerica}
 Diremos que una hoja $\tilde{L}$ de la foliación $\tilde{\Fol{\alpha}}$ es \emph{genérica} si $\tilde{L}\cap T_{a}$ tiene un punto genérico, donde $T_{a}=h^{-1}(a)\simeq\C/\Gamma$ es una transversal a todas las hojas de la foliación $\tilde{\Fol{\alpha}}$.
\end{defn}

Notemos que la condición de ser una hoja genérica no depende del punto $b\in\tilde{L}\cap T_{a}$. En efecto, si denotamos por $L$ a $\tilde{L}\setminus h^{-1}(0,1,\infty)$, la restricción de $h$ a $L$ es una aplicación cubriente y así, $L\cap T_{a}=\tilde{L}\cap T_{a}$ es la fibra $h^{-1}(a)$ de esta aplicación cubriente. Como la acción de monodromía de toda aplicación cubriente es transitiva, todos los puntos tienen estabilizadores conjugados. Así, si un punto $b\in L\cap T_{a}$ es genérico, todos los puntos en $\tilde{L}\cap T_{a}$ son genéricos.

Como en el caso en que $\alpha\in E$ el grupo de holonomía global $G_{\alpha}$ es finito, los puntos $b\in T_{a}$ son genéricos son densos y no numerables. Por lo tanto, si $\alpha\in E$ el conjunto de hojas genéricas de la foliación $\Fol{\alpha}$ es denso y no numerable.\\

Una hoja $\tilde{L}_{1}$ de la foliación $\tilde{\Fol{1}}$ es biholomorfa a $\C/\Gamma$. Dado $\alpha\in E$ y una hoja genérica $\tilde{L}$ de la foliación $\Fol{\alpha}$, vamos a construir una aplicación cubriente $\pi\colon\tilde{L}\rightarrow\tilde{L}_{1}$. Denotemos por $h_{0}$ y $h_{1}$ a las restricciones de $h=H\circ\Pi$ a las hojas $\tilde{L}$ y $\tilde{L}_{1}$ respectivamente y denotemos por $L$ y $L_{1}$ a $\tilde{L}\setminus\{\, 0,1,\infty \, \}$ y $\tilde{L}_{1}\setminus\{\, 0,1,\infty \, \}$.

Escojamos un punto $b\in L\cap T_{a}$ y un punto $b_{1}\in L_{1}\cap T_{a}$ y definamos $\pi(b)=b_{0}$, vamos a definir $\pi$ de tal manera que $h_{0}=h_{1}\circ\pi$. Sean $x\in\tilde{L}$ y $\tilde{\gamma}\colon [0,1]\rightarrow L$ un camino en $L$ que una $b$ con $x$. Al aplicar $h_{0}$, $\tilde{\gamma}$ se convierte en una curva $\gamma\colon [0,1]\rightarrow\C\setminus\{\, 0,1\, \}$ que podemos levantar a una curva $\hat{\gamma}\colon [0,1]\rightarrow  L_{1}$ mediante la aplicación cubriente $h_{1}$. Definimos $\pi(x)=\hat{\gamma}(1)$.

Veamos que $\pi(x)$ no depende del camino $\tilde{\gamma}$ que une a $b$ con $x$. Como es usual, basta ver que si $\tilde{\gamma}$ es un camino cerrado basado en $b$, $\hat{\gamma}(1)=b_{1}$. La transformación de holonomía $k\colon\C/\Gamma\rightarrow\C/\Gamma$ del grupo $G_{\alpha}$ asociada a la curva $\gamma=h_{0}\circ\tilde{\gamma}$ deja fijo al punto $b$, y como la hoja $\tilde{L}$ es genérica, la transformación $k$ tiene que ser la identidad. Así, si escribimos a $\gamma$ como producto de los dos generadores del grupo fundamental de $\C\setminus\{\, 0,1\, \}$, $\gamma=\gamma_{1}^{m_{1}}\circ\gamma_{2}^{n_{1}}\circ\cdots\circ\gamma_{1}^{m_{r}}\circ\gamma_{2}^{m_{r}}$ podemos escribir a $k$ como un producto de los generadores $f_{\alpha}$ y $g_{\alpha}$ del grupo de holonomía global $G_{\alpha}$ de la foliación $\tilde{\Fol{\alpha}}$, obteniendo:

$$k(z)=f_{\alpha}^{m_{1}}\circ g_{\alpha}^{n_{1}}\circ\cdots\circ f_{\alpha}^{m_{r}}\circ g_{\alpha}^{n_{r}}=cz+d=z,$$

\noindent donde $c=j^{m_{1}+\cdots+m_{r}}$. Como $c=1$ tenemos que $3$ divide a $m_{1}+\cdots+m_{r}$. En la sección anterior vimos que en el grupo $G_{1}$, los dos generadores $f_{1}$ y $g_{1}$ coinciden ($f_{1}(z)=jz$). Así, la transformación de holonomía del grupo $G_{1}$ asociada al lazo $\gamma$ es $K(z)=j^{m_{1}+\cdots+m_{r}}=j^{3p}=z$ y por lo tanto $\hat{\gamma}(b)=b_{1}$.

Hemos definido $\pi\colon L\rightarrow L_{1}$ de tal forma que $h_{0}=h_{1}\circ\pi$, y como $h_{1}$ es invertible de manera local, localmente tenemos que $\pi=h_{1}^{-1}\circ h_{0}$. Por lo tanto, es holomorfa y más aún, es un homeomorfismo local.\\

Ahora vamos a extender a $\pi$ al conjunto $\tilde{L}\cap h^{-1}(0,1,\infty)$. El conjunto $h^{-1}(0,1,\infty)$ está formado por tres divisores $D_{1},D_{2}$ y $D_{3}$ que se obtienen de explotar los puntos, en coordenadas homogéneas, $[1:0:0],[0:1:0]$ y $[0:0:1]$ y por el transformado estricto $\tilde{\mathit{l}_{k}}$ de las nueve rectas $\mathit{l}_{k}$ que deja invariante toda foliación $\Fol{\alpha}$ de la familia. Como la foliación $\tilde{\Fol{\alpha}}$ deja invariante al transformado estricto de estas nueve rectas y la hoja $\tilde{L}$ no pasa por ningún punto silla $\tilde{q_{k}(\alpha)}$ (estas hojas no son genéricas), entonces las intersecciones tanto de $\tilde{L}$ como de $\tilde{L}_{1}$ con el conjunto $h^{-1}(0,1,\infty)$ están contenidas en el conjunto $D_{1}\cup D_{2}\cup D_{3}$. Vamos a definir $\pi$ para los puntos en $\tilde{L}\cap D_{1}$, donde $D_{1}$ es el divisor obtenido al explotar el punto $(0,0)$, para definirla en los otros divisores podemos usar el corolario \ref{Coro2Prop1} y la definición de $\pi$ para los puntos en $\tilde{L}\cap D_{1}$.\\

La hoja $\tilde{L}_{1}$ queda descrita por $\tilde{h}_{1}^{-1}(e)=(H_{1}\circ\Pi)^{-1}(e)$, $e\in\C\setminus\{\, 0,1 \, \}$, donde $H_{1}(x,y)=\tfrac{(x-j^{2})(y-j)(y-j^{2}x)}{(x-j)(y-j^{2})(y-jx)}$ es una primera integral de la foliación $\Fol{1}$ de $\CP$ y $\Pi\colon M\rightarrow \CP$ es la transformación que explota los doce puntos radiales de la foliación $\Fol{\alpha}$. Usando la carta $(x,u)$ de $M$ en la cual $\Pi(x,u)=(x,ux)$ tenemos que $\tilde{h}_{1}(x,u)=\tfrac{(x-j^{2})(ux-j)(u-j^{2})}{(x-j)(ux-j^{2})(u-j)}$. Al evaluar en el divisor $D_{1}=\{\, x=0\, \}$ obtenemos $\tilde{h}_{1}(0,u)=\tfrac{u-j^{2}}{u-j}$ y por lo tanto, la intersección de $\tilde{L}_{1}=h^{-1}(e)$ con el divisor $D_{1}=\{\, x=0\, \}$ es el punto $(0,\tfrac{j^{2}-je}{1-e}):=p_{0}$. Así, la hoja $\tilde{L}_{1}$ interseca en único punto $p_{0}$ y para todo $q\in\tilde{L}\cap D_{1}$ definimos $\pi(q)=p_{0}$.\\

Veamos que $\pi$ también es holomorfa y un homeomorfismo local en los puntos $q\in\tilde{L}\cap h^{-1}(1)$. Por el lema \ref{Lema:ExpresionLocalDeLaAplicacionCubriente} para cada punto $q\in\tilde{L}\cap h^{-1}(1)$ y $p_{0}\in\tilde{L}$ existen coordenadas $z$ y $w$ de $\tilde{L}$ y $\tilde{L}_{1}$ alrededor de $q$ y $p$ respectivamente tales que $h_{0}(z)=1+z^{3}$ y $h_{1}(w)=1+w^{3}$. Como $h_{0}=h_{1}\circ\pi$ fuera del conjunto $D_{1}\cap\tilde{L}$ tenemos que $z^{3}=(w(\pi(z)))^{3}$ para toda $z\neq 0$. Así, $w(\pi(z))=j^{i}z$ y por lo tanto $\pi$ es holomorfa y un homemorfismo local alrededor de cada punto $q\in\tilde{L}\cap h^{-1}(1)$.\\

Para ver que $\pi$ es una aplicación cubriente basta mostrar que cumple la propiedad de levantamiento de curvas. Sea $\tilde{\gamma}\colon [0,1]\rightarrow\tilde{L}_{1}$ una curva en la hoja $\tilde{L}_{1}$ y consideremos la curva $\gamma:=h_{1}\circ\tilde{\gamma}$. Usando a $h_{0}$ podemos levantar a la curva $\gamma$ a una curva $\hat{\gamma}\colon [0,1]\rightarrow\tilde{L}_{0}$. Afirmamos que $\hat{\gamma}$ es un levantamiento de $\tilde{\gamma}$. En efecto, como $\hat{\gamma}$ es un levantamiento de $\gamma$ tenemos que $h_{1}\circ\tilde{\gamma}=\gamma=h_{0}\circ\hat{\gamma}$, pero por construcción, $h_{0}=h_{1}\circ\pi$ y así, $h_{1}\circ\tilde{\gamma}=(h_{1}\circ\pi)\circ\hat{\gamma}$ y como $h_{1}$ es un homeomorfismo local tenemos que punto a punto $\tilde{\gamma}=\pi\circ\hat{\gamma}$. Hemos probado así, la siguiente proposición.

\begin{Proposicion}
\label{Prop:GeneroDeLasSoluciones}
Si la foliación $\Fol{\alpha}$ de $\CP$ tiene primera integral racional, entonces casi cualquier hoja de la foliación tiene género uno.
\end{Proposicion}

Como vimos en la sección anterior, podemos escoger un parámetro $\alpha$ de tal forma que el grado de la primera integral de la foliación $\Fol{\alpha}$ sea arbitrariamente grande, pero como afirma la proposición anterior, el género de las soluciones es uno.
    
\appendix
\chapter{Transformaciones entre toros complejos.}
\label{ApendiceToros}        
            En esta pequeña sección vamos a estudiar transformaciones entre toros complejos y clasificaremos los automorfismos de un toro que fijan un punto (isomorfismo de grupos).

\begin{defn}
\label{latiz}
Sean $\omega,\tilde{\omega}\in\C$ dos números complejos que son linealmente independientes como vectores en $\mathbb{R}^{2}$. Al conjunto $L=\omega\mathbb{Z}\oplus\tilde{\omega}\mathbb{Z}$ lo llamaremos la \emph{retícula} generada por $\omega$ y $\tilde{\omega}$
\end{defn}

Observemos que por definición una retícula $L$ es un subgrupo aditivo de $\C$, como $\C$ es conmutativo podemos considerar el grupo cociente $X_{L}:=\C/L$. Este conjunto es bien sabido que es homeomorfo a un toro y con la proyección canónica $\Pi\colon\C\rightarrow L$ podemos hacer de $X_{L}$ una superficie de Riemann.\\

Sean $X,Y$ dos toros con retículas asociadas $L$ y $M$ respectivamente. Si $F\colon X\rightarrow Y$ es una función holomorfa, entonces, por la fórmula de Riemann-Hurwitz, $F$ no tiene puntos de ramifiación y por lo tanto es una aplicación cubriente. Como la proyección canónica $\Pi_{L}\colon\C\rightarrow X$ es la  cubriente universal de $X$ tenemos que $F\circ\Pi\colon\C\rightarrow Y$ es una aplicación cubriente y como $\Pi_{L}\colon\C\rightarrow Y$ es la cubriente universal de $Y$ debe de existir una función holomorfa $G$ que hace conmutar el siguiente diagrama.\\

%Diagrama  

A continuación obtendremos una expresión para la función $G$. Para cualquier retícula $M$ y cualquier número complejo $a\in\C$, la traslación $z\mapsto a+z$ respeta las clases de equivalencia de $\C/M$ y por lo tanto la traslación baja a una traslación en el cociente $\C/M$. Así, con una traslación adecuada, siempre podemos suponer que $G(0)=0$. Si $l\in L$, por la conmutatividad del diagrama tenemos que $G(z-l)=G(z)$ (mod$M$) y así, para $l$ fijo la función $G(z-l)-G(z)\colon\C\rightarrow M\subset\C$ toma valores en $M$. Como $M$ es discreto y $G$ es continua tenemos que $G(z-l)-G(z)$ es constante y por lo tanto $G(z-l)'=G(z)'$. Esto quiere decir que todos los valores de $G'$ son tomados en una región contenida en el paralelogramo formado por dos generadores de $M$ y por lo tanto $G'$ es acotada y por el teorema de Liuoville, $G'$ es constante. Así, $G(z)=\gamma z$ para algún $\gamma\in\C$. Como $G(z-l)=G(z)$ (mod$M$) tenemos que $G(l)=G(0)=0$ (mod$M$) y entonces $\gamma L\subset M$. Si $F$ es un isomorfismo tenemos que $\gamma^{-1}M\subset L$ y entonces $M\subset\gamma L$, por lo tanto concluímos que $\gamma L = M$.\\

Ahora vamos a clasificar los automorfismos de un toro en sí mismo. Sea $F\colon X\rightarrow X$ un biholomorfismo que fija al origen (automorfsimo de grupos), en la cubriente universal esta transformación adquiere la expresión $G(z)=\gamma z$ y además se cumple que $\gamma L=L$. Afirmamos que $\gamma$ tiene que ser una raíz de la unidad. En efecto, si $|\gamma|<1$ y $l\in L$ es un elemento de la retícula de longitud mínima, $|\gamma l|<|\gamma|$, y por lo tanto $\gamma l\notin L$, esto claramente no pude suceder. Si $|\gamma|>1$ podemos hacer un argumento similar. Si el argumento de $\gamma$ no es conmesurable con $\pi$, la órbita de un punto en la circunferencia unitaria se hace densa en ella y en consecuencia no puede ocurrir que $\gamma L = L$. Por lo tanto, $\gamma$ es una raíz de la unidad.

Si $l\in L\setminus \{ 0\}$ y $\gamma\in\C\setminus\mathbb{R}$, entonces $l$ y $\gamma l$ generan a $L$ y en consecuencia $\gamma^{2} l$ debe de ser una combinación entera de ellos dos.
\begin{equation*}
\gamma^{2} l =m\gamma l +nl,
\end{equation*}
\noindent entonces,
\begin{equation*}
\gamma^{2}-m\gamma-nl=0.
\end{equation*}
\noindent Por lo tanto,
\begin{equation}
\label{TorosPosibles}
\gamma=\frac{m\pm \sqrt{m^{2}-4n}}{2}.
\end{equation}

\noindent Como $\gamma$ es una raíz de la unidad los únicos valores posibles para $m$ son $-2,-1,-0,1$ y $2$. Esto nos dice que $\gamma$ puede ser una raíz cuadrada, cuarta ó sexta de la unidad. Ahora que sabemos que valores puede tomar $\gamma$, la condición $\gamma L =L$ nos obligará a poner algunas restricciones a $L$. Si $\gamma$ es una raíz cuadrada de la unidad no hay ningun problema pues $-L=L$. Cuando $\gamma$ es una raiz cuarta de la unidad los generadores de $L$ son ortogonales (y en consecuencia $L\simeq \mathbb{Z}\oplus i\mathbb{Z}$), si $\gamma$ es una raíz sexta entonces los generadores forman un ángulo de $\tfrac{\pi}{3}$ ($L\simeq \mathbb{Z}\oplus e^{\tfrac{\pi i}{3}}\mathbb{Z}$).

Denotemos por $j=e^{\tfrac{2\pi i}{3}}$, $L=\mathbb{Z}\oplus e^{\tfrac{\pi i}{3}}\mathbb{Z}$
y $\Gamma=\mathbb{Z}\oplus j\mathbb{Z}$. Observemos que $z\mapsto jz$ lleva a la retícula $L$ en la retícula $\Gamma$ y en consecuencia $\C/L\simeq\C/\Gamma$. El párrafo anterior nos dice que si un toro $X$ tiene un automorfismo de orden tres, entonces $X\simeq\C/\Gamma$ ($j$ es una raíz sexta de la unidad y tiene orden tres).


\begin{thebibliography}{9}
\bibitem{IlyaYako}
Yu. S. Ilyashenko, S. Yakovenko,
\emph{Lectures on analytic differential equations},
American Mathematical Society.

\bibitem{Mumford}
D. Mumford, 
\emph{Algebraic geometry},
Springer-Verlag.

\bibitem{CamachoAlcides}
C. Camacho, A. Lins Neto,
\emph{Teoria geométrica das folheaçóes},

\bibitem{FischerGerd}
G. Fischer,
\emph{Plane algebraic curves},
American Mathematical Society.

\bibitem{MirandaR}
R. Miranda,
\emph{Algebraic curves and riemann surfaces},
American Mathematical Society.

\bibitem{Mattei-Moussu}
J.F. Mattei, R.Moussu,
\emph{Holonomie et intégrales premières},
Ann. Sc. Ec. Norm. Su. 13 (1980) 469-523.

\bibitem{Articulo}
A. Lins Neto,
\emph{Some exapmles for the Poincaré and Painlevé problems},
Ann. Sc. Ec. Norm. Su. 4 (2002) 231-266.
\end{thebibliography}




\end{document}

