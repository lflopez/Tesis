\section{Ecuaciones diferenciales y foliaciones.}

Sea $U\subset\C^{2}$ un abierto y $F=(F_{1},F_{2})\colon U\rightarrow\C^{2}$ un campo vectorial holomorfo. La ecuación diferencial (autónoma) asociada a $F$ se define como:

\begin{equation}
\label{EcuacionDiferencial}
\begin{aligned}
\frac{dx}{dt} &=F_{1}(x,y) \\
\frac{dy}{dt} &=F_{2}(x,y),\ \ t\in\C.
\end{aligned}
\end{equation}

Denotaremos por $(\dot{x},\dot{y})$ a $(\tfrac{dx}{dt},\tfrac{dy}{dt})$.\\

Una solución de esta ecuación diferencial es una curva parametrizada $\varphi\colon D\subset\C\rightarrow\C^{2}$ que satisface $\frac{d\varphi(t)}{dt} = F(\varphi(t)),\ \forall t\in V\subset\C$.\\

Un resultado fundamental en la teoría de ecuaciones diferenciales es el teorema de existencia y unicidad de soluciones. A continuación enunciamos este resultado:

\begin{Teorema}
\label{Teo:ExistenciaUnicidad}
\cite{IlyaYako} Para cualquier ecuación diferencial (\ref{EcuacionDiferencial}) y todo punto $(x_{0},y_{0})\in U$ existe un abierto $V\subset U$ y un disco $D_{\epsilon}=\{\left|t-t_{0}\right|<\epsilon \}\subset\C$, tal que la solución con condición inicial $(x,y,t)\in V\times D_{\epsilon}$ existe y es única.

Más aún, la solución depende de manera holomorfa de la condición inicial, y si $F$ depende de manera holomorfa de algunos otros parámetros, la solución también depende de manera holomorfa de estos parámetros.
\end{Teorema}

El teorema anterior nos asegura que alrededor de cada punto $x\in U$, existe una función analítica $\varphi\colon V\times D_{\epsilon}\rightarrow U$ tal que si fijamos $x_{0}\in V$, $\varphi(x_{0},t)$ es la solución de la ecuación diferencial (\ref{EcuacionDiferencial}) con condición incial $\varphi(x_{0},0)=x_{0}$. A esta función se le suele llamar \emph{flujo de la ecuación diferencial} (\ref{EcuacionDiferencial}).

A pesar de contar con el teorema de existencia y unicidad para ecuaciones diferenciales holomorfas, en principio, un campo vectorial puede ser muy complicado y las soluciones de la ecuación diferencial que determina pueden ser imposibles de escribir explícitamente. Es por eso que con frecuencia buscamos cambios de coordenadas (continuos, diferenciables ó analíticos) que simplifiquen el campo vectorial.

\begin{defn}
\label{Def:EquivalenciaAnalitica}
Decimos que dos ecuaciones diferenciales determinadas, respectivamente, por dos campos vectoriales $F$ y $F'$ definidos en los conjuntos abiertos $U$ y $U'$ son \emph{analíticamente equivalentes} si existe un biholomorfismo $H\colon U \rightarrow U'$ que cumple la siguiente relación:
\begin{equation}
\label{EquivalenciaAnalitica}
\Big(\frac{\partial H}{\partial x}\Big)F(x)=F'(H(x)).
\end{equation}
\end{defn}

Lo anterior quiere decir que la diferencial del biholomorfismo $H$ lleva el vector $F(x)$, anclado en el punto $x$, en el vector $F'(H(x))$ anclado en el punto $H(x)$.\\

Es muy común que en los textos sobre foliaciones ó ecuaciones diferenciales en variedades analíticas de dimensión dos, se usen las palabras foliación, campo vectorial y 1-forma para referirse al mismo objeto. Esta costumbre está justificada por los teoremas (\ref{Teo:FoliacionesGeneradas}) y (\ref{Teo:CamposEquivalentes}) que básicamente nos dicen que cuando en una ecuación diferencial nos olvidamos del tiempo y la parametrización de las soluciones, nos quedamos con una foliación. No es el propósito de esta tesis probar estos resultados, pero es bueno entender bien lo que éstos dicen para comprender como es que se relacionan estos objetos geométricos aparentemente distintos.\\
\begin{defn}
\label{Def:PuntoSingular}
A un punto $x\in U$ tal que $F(x)=(0,0)$ se le conoce como \emph{punto singular}. A lo largo de este trabajo denotaremos por $\Sigma$ al conjunto de puntos singulares de $F$.
\end{defn}
 Otro hecho importante de la teoría de ecuaciones diferenciales es que fuera de los lugares con puntos singulares, todas las ecuaciones diferenciales tienen el mismo comportamiento local.

\begin{Teorema}
\label{Teo:Rectificacion}
Todo ecuación diferencial definida por un campo vectorial holomorfo $F$ es, en una vecindad de un punto no singular, analíticamente equivalente a la ecuación diferencial determinada por el campo vectorial constante $\tilde{F}(x)=(1,0)$.
\end{Teorema}
\begin{proof}
  El flujo del campo vectorial $\tilde{F}$ es $\tilde{\varphi}((x,y,t))=(x+t,y)$. Observemos que para toda $(x,y)$, $\tilde{\varphi}((x,y,-x))\in\Pi':=\{x=0\}$. Sea $\Pi$ una transversal al vector $F((x_{0},y_{0}))$ en el punto $(x_{0},y_{0})$. Si $h\colon\Pi'\rightarrow\Pi$ es un isomorfismo lineal y $\varphi((x,y,t))$ es el flujo del campo vectorial $F$, entonces la función $H=\varphi(h[\tilde{\varphi}((x,y,-x))],x)$ lleva las soluciones del campo vectorial $\tilde{F}$ en las soluciones del campo vectorial $F$. Además, $H$ manda a la transversal $\Pi'$ en la transversal $\Pi$ y el vector $(1,0)$ en el vector $F((x_{0},y_{0}))\neq 0$, esto nos permite concluir que $H$ es invertible.   
\end{proof}
Al teorema anterior se le conoce como el teorema de rectificación. Las soluciones de la ecuación diferencial definida por el campo $F'=(1,0)$ son $x=x_{0}+t,y=y_{0}$, y como afirma el teorema, si no hay puntos singulares, localmente podemos escribir a las soluciones de esta manera; esto motiva la siguiente definición:

\begin{defn}
\label{Def:FoliacionEstandar}
La \emph{foliación estándar} del disco $B:=\{(x,y)\in\C^{2} \mid \left|x\right|<1, \left|y\right|<1 \}$ es la partición de $B$ por conjuntos $L_{y}=\{ \left|x\right|<1 \}\times\{y \} $ que llamaremos \emph{hojas de la foliación}, $\mathit{i.e.}$:
\begin{equation}
B=\bigsqcup_{\left|y\right|<1}L_{y}.
\end{equation}
\end{defn}

En la mayor parte de este trabajo trabajaremos en $\CP$, es por eso que necesitamos la siguiente definición:

\begin{defn}
\label{Def:FoliacionNoSingular}
Una \emph{foliación no singular} $\mathcal{F}$ en una dos variedad analítica $M$, es una partición de $M$ en hojas $L_{\alpha}$ de tal forma que todo punto $x\in M$ tiene una vecindad $B'$ y un biholomorfismo $H\colon B'\rightarrow B $ que manda las hojas locales $L_{\alpha}\cap B'$ de $\mathcal{F}$ en las hojas de la foliación estándar. A la pareja $(H,B')$ la llamaremos una \emph{carta distinguida} de la foliación $\mathcal{F}$.
\end{defn}

Es decir, una foliación no singular en una dos variedad analítica $M$, es una partición de $M$ que localmente se ve como la foliación estándar. El teorema (\ref{Teo:Rectificacion}) nos asegura que las soluciones de una ecuación diferencal definida por un campo vectorial holomorfo sin puntos singulares, generan una foliación no singular.\\

Como en los puntos que no son singulares una ecuación diferencial tiene un comportamiento local sencillo, el verdadero interés es estudiar el comportamiento de una ecuación diferencial en una vecindad de un punto singular y, eventualmente, hacer un análisis global del comportamiento de una ecuación cuando este sea posible.

\begin{defn}
Una \emph{foliación singular holomorfa} en una dos variedad analítica $M$ es una foliación no singular en $M\setminus\Sigma$ donde $\Sigma$ es un conjunto de puntos aislados al cual llamaremos el \emph{conjunto singular de la foliación}.
\end{defn}

Todo campo vectorial $F$ en $M$, con conjunto singular $\Sigma$, define, mediante las soluciones de la ecuación diferencial que determina,  una foliación no singular $\mathcal{F}$ en $M\setminus\Sigma$. Pero en principio, $\Sigma$ podría ser un conjunto muy grande (por ejemplo, $\Sigma$ podría ser toda una curva analítica) como para definir una foliación singular en M. El siguiente resultado afirma que para foliaciones en dos variedades analiticas, siempre es posible encontrar un conjunto de puntos aislados $\Sigma'\subset\Sigma$ y una foliación no singular $\tilde{\mathcal{F}}$ de $M\setminus\Sigma'$ de tal forma que las hojas de $\tilde{\mathcal{F}}$ coincidan con las hojas de $\mathcal{F}$ en $M\setminus\Sigma$.

\begin{Teorema}
\label{Teo:ExtensionFoliaciones}
Sea $F$ un campo vectorial definido en una dos variedad analítica $M$. Si $\Sigma$ es el conjunto singular de $F$, entonces existe un conjunto de puntos aislados $\Sigma'\subset\Sigma$ y una foliación no singular holomorfa $\tilde{\mathcal{F}}$ de $M\setminus\Sigma'$ cuya restricción a $M\setminus\Sigma$ coincide con la foliación generada por el campo vectorial $F$.
\end{Teorema}

El teorema anterior básicamente dice que en las foliaciones singulares generadas por campos vectoriales holomorfos siempre podemos suponer que el conjunto singular $\Sigma$, es un conjunto de puntos aislados. El resultado recíproco también es cierto.

\begin{Teorema}
\label{Teo:FoliacionesGeneradas}
Sea $\Sigma\subset M$ un conjunto aislado de puntos de la dos variedad analítica $M$, $\mathcal{F}$ una foliación no singular holomorfa en $\Sigma\setminus M$ que no se puede extender a ningún subconjunto de $\Sigma$.

Entonces, en una vecindad $U$ de cada punto $a\in\Sigma$, la foliación $\mathcal{F}$ está generada por un campo vectorial holomorfo $F$ con conjunto singular $\Sigma\cap U$.
\end{Teorema} 

Así como definimos equivalencia analítica entre campos vectoriales, ahora damos la definición correspondiente para foliaciones.

\begin{defn}
Decimos que dos foliaciones $\mathcal{F}$ y $\mathcal{F}'$ definidas en las variedades $M$ y $M'$ son \emph{analíticamente equivalentes} si existe un biholomorfismo $H\colon M\rightarrow M'$ que manda las hojas de $\mathcal{F}$ en las hojas de $\mathcal{F}'$ y el conjunto singular $\Sigma$ de $\mathcal{F}$ en el conjunto singular $\Sigma'$ de $\tilde{\mathcal{F}}$.
\end{defn}

El siguiente teorema relaciona campos vectoriales equivalentes con foliaciones equivalentes y foliaciones equivalentes con campos vectoriales ``casi'' equivalentes.

\begin{Teorema}
\label{Teo:CamposEquivalentes}
Sean $\dot{x=F(x)}$ y $\dot{x}=F'(x)$ dos ecuaciones diferenciales definidas por los campos vectoriales $F$ y $F'$ respectivamente. Si las ecuaciones diferenciales son analíticamente equivalentes, entonces las foliaciones que generan son analíticamente equivalentes.

Recíprocamente, si las foliaciones $\mathcal{F}$ y $\mathcal{F}'$ que generan los campos vectoriales $F$ y $F'$ respectivamente, son analíticamente equivalentes, entonces existe una función holomorfa $\rho$ que no se anula fuera del conjunto singular de $\mathcal{F}$ y cumple:
\begin{equation}
\rho(x)\Big( \frac{\partial H}{\partial x}\Big)F(x) = F'(H(x)).
\end{equation} 
\end{Teorema}

El teorema anterior nos dice que, cuando hablamos de foliaciones, ya no importa tanto el vector tangente (y por lo tanto la parametrización de la solución) si no todo el subespacio uno dimensional que es tangente a la hoja de la foliación (la multiplicación por la función $\rho$ es lo que nos permite concluir esto). Una manera de codificar estos subespacios tangentes es usando 1-formas.\\

A todo campo vectorial $F=(F_{1},F_{2})$ le podemos asociar la 1-forma $F_{1}\, dy - F_{2}\, dx $. Observemos que nuestro campo vectorial anula a esta 1-forma, pero como la 1-forma es lineal, también anula a todo el subespacio generado por el vector $(F_{1},F_{2})$. Así, el kernel de la 1-forma está formado por todos los subespacios tangentes a las soluciones de la ecuación diferencial determinada por el campo vectorial $F$.\\

Una de las ventajas de usar 1-formas es que si tenemos una función holomorfa $H$ entre dos variedades analíticas $M$ y $N$, podemos ``jalar'' cualquier 1-forma $\omega$ en $N$ a una 1-forma $H^{*}(\omega)$ en $M$. Esto lo podemos lograr usando la diferencial de $H$ para empujar vectores tangentes a $M$ en vectores tangentes a $N$ y después evaluarlos en $\omega$, $\mathit{i.e.}$:

\begin{equation}
H^{*}(\omega)(v):=\omega\Big(\Big(\frac{\partial H}{\partial x}\Big)v\Big),\ v\in TM.
\end{equation}

\section{Holonomía local.}

Una herramienta muy útil al estudiar una ecuación diferencial es el análisis del comportamiento de la dinámica transversal de sus soluciones. Esta dinámica la obtenemos estudiando el grupo de holonomía asociado a una hoja determinada de la foliación. A continuación veremos cómo asociar este grupo a una ecuación diferencial.\\

Una transversal a una hoja $L$ de una foliación $\mathcal{F}$ de $M$ en el punto $a$, es la imagen de una función holomorfa $\tau\colon (\C,0)\rightarrow (M,a)$ que corta de manera transversal a $L$.\\

Si en un punto $a\in L$ tomamos una carta distinguida $(H=(h_{1},h_{2}),B')$ del punto $a$ tal que $H(L)=L_{0}=\{(x,0)\}\subset B$ entonces, para todo punto $b\in L\cap B'$, si consideramos las transversales locales a $L$ en los puntos $a$ y $b$ definidas por $\tau:=H^{-1}(\,\{\,(h_{1}(a),y); \left|y\right|<1 \,\}\,)$ y $\tau':=H^{-1}(\,\{\,(h_{1}(b),y); \left|y\right|<1 \,\}\,)$, podemos definir una transformación entre las transversales $\tau$ y $\tau'$.

En efecto, si en la hoja $L$ tomamos un curva $\gamma\colon [0,1]\rightarrow L$ que una a los puntos $a$ y $b$ entonces, la curva imagen $\tilde{\gamma}:=H(\gamma)$ que une a los puntos $(h_{1}(a),0)$ y $(h_{1}(b,0)$ puede ser levantada a curvas $\tilde{\gamma_{\alpha}}$ en las hojas $L_{\alpha}$ de la foliación estándar para así, mediante el biholomorfismo $H^{-1}$, obtener curvas $\gamma_{\alpha}$ que unan a las transversales $\tau$ y $\tau'$.\\

Así, para un punto $x\in\tau$, existe una curva $\gamma_{\alpha}\colon[0,1]\rightarrow L_{\alpha}$ tal que $\gamma_{\alpha}(0)\in\tau$ y $\gamma_{\alpha}(1)\in\tau'$ y por lo tanto, podemos definir la  \emph{transformación de correspondencia} entre $\tau$  y $\tau'$ asociada a $\gamma$ como $\Delta^{\gamma}_{\tau,\tau'}(x)=\gamma_{\alpha}(1)$. Como las soluciones de una ecuación diferencial dependen de manera holomorfa de la condición inicial, la transformación de correspondencia es una función holomorfa.\\

Si tomamos un tercer punto $c\in\gamma([0,1])$ y una tercera transversal $\tau''$ a $L$ por el punto $c$, entonces se satisface la identidad:

\begin{equation}
\label{IdentidadCorrespondencia}
\Delta^{\gamma}_{\tau,\tau''}=\Delta^{\gamma}_{\tau',\tau''}\circ\Delta^{\gamma}_{\tau,\tau'},
\end{equation}

\noindent además, la transformación de correspondencia sólo depende de la clase de homotopía de $\gamma$. Estas dos observaciones nos van a permitir definir una transformación de correspondencia entre dos transversales a una hoja $L$ que sean unidas por un camino $\gamma$.\\  

Si tenemos una curva $\gamma\colon [0,1]\rightarrow L$ y dos transversales $\tau,\tau'$ a $L$ en los puntos $\gamma(0)$ y $\gamma(1)$ respectivamente, podemos cubrir a $\gamma([0,1])$ con un número finito de abiertos  distinguidos $U_{j}$ de tal forma que en cada $U_{j}$ la foliación es equivalente a la foliación estándar. En cada uno de estos abiertos $U_{j}$ podemos poner transversales auxiliares $\tau_{j}$ y obtener un mapeo de correspondencia entre transversales consecutivas. Si componemos todos estos mapeos, al final obtenemos una transformación $\Delta_{\gamma}$ entre las transversales $\tau$ y $\tau'$.\\

Si $\gamma^{-1}$ es la curva $\gamma$ recorrida en sentido opuesto tenemos que $\Delta_{\gamma^{-1}}\circ\Delta_{\gamma}=Id$ y entonces, la transformación $\Delta_{\gamma}$ es un biholomorfismo.\\

Como una transversal no es más que una imagen biholomorfa de una vecindad $U\subset\C$ podemos tomar cartas de cada una de ellas y  así, si los puntos $a,a'\in L$ por los cuales pasan las transversales $\tau$ y $\tau'$ son distintos, reparametrizando las transversales, siempre es posible encontrar coordenadas de tal forma que el mapeo de correspondencia asociado a un camino que una ambos puntos, se vea como la identidad, es por eso que vamos a considerar caminos cerrados y usaremos una sola transversal $\tau$.

\begin{defn}
Sea $a\in L$, $\tau$ una transversal a $L$ por el punto $a$ y $\gamma\in\pi_{1}(L,a)$ un camino cerrado. La \emph{transformación de holonomía} $\Delta_{\gamma}\colon (\tau,a)\rightarrow(\tau,a)$ es la transformación de correspondencia a lo largo del camino cerrado $\gamma$. Tomando una carta de $\tau$ alrededor del punto $a$, podemos pensar a la transformación de holonomía como un biholomorfismo de una vecindad $U\subset\C$ del origen en ella misma, es decir, $\Delta_{\gamma}\in \mathrm{Diff}(\C,0)$.
\end{defn} 

Con las construcciones anteriores, por cada elemento $\gamma$ del grupo fundamental de la hoja $L$ obtenemos un biholomorfismo $\Delta_{\gamma}\in\mathrm{Diff}(\C,0)$. Al grupo formado por todos estos biholomorfismos lo llamaremos grupo de holonomía de la foliación $\mathcal{F}$ a lo largo de la hoja $L$.

\begin{Ejemplo}
\label{Ej:HolonomiaLineal}
La ecuación diferencial lineal
\begin{equation}
\begin{aligned}
\dot{x} &= \lambda_{1}x\\
\dot{y} &= \lambda_{2}y
\end{aligned}
\end{equation}
tiene por solución a 
\begin{equation}
\begin{aligned}
x(t) &= c_{1}e^{\lambda_{1}t}\\
y(t) &= c_{2}e^{\lambda_{2}t},\ \  c_{1},c_{2}\in\C.
\end{aligned}
\end{equation}
Vamos a calcular el grupo de holonomía asociado a la hoja $L=\{y=0\}\setminus\{0\}$ y la transversal $\{x=1\}$.

Un generador del grupo fundamental de $L$ es el lazo $\gamma=\{\left|x\right| = 1\}$. Si consideramos el segmento de recta $\{t\tfrac{2\pi i}{\lambda_{1}} \mid t\in[0,1]\}\subset\C$, su imagen bajo la solución con condición inicial (1,0) coincide con el lazo $\gamma$, y su imagen bajo cualquier otra solución que tenga condición inicial en la recta $\{x=1\}$ termina de nuevo en esta transversal pues, $x(\tfrac{2\pi i}{\lambda_{1}})=c_{1}e^{2\pi i}=c_{1}$ y si la condición inicial está en la transversal, entonces $c_{1}=1$ y así, $x(\tfrac{2\pi i}{\lambda_{1}})=1$. Por lo tanto, el mapeo de holonomía es: 

\begin{equation}
y \mapsto e^{2\pi i \frac{\lambda_{2}}{\lambda_{1}}}y.
\end{equation}   
\end{Ejemplo}

\marginpar{referencia}

La transformación de holonomía se puede definir en cualquier transversal a una hoja de la foliación, podemos cambiar de transversal usando, otra vez, a las hojas de la foliación y así obtenemos un biholomorfismo entre las transversales. Por lo tanto, los grupos de holonomía asociados a ambas transversales son conjugados el uno del otro. 

\section{Holonomía global.}

La transformación de holonomía que se definió en el sección anterior es un concepto local, la transformación de holonomía fija un punto (la intersección de la transversal $\tau$ con la hoja  $L$ a la cual se le está calculando el grupo de holonomía) y la transformación sólo está definida en una vecindad de ese punto.\\

Sería agradable poder, en algunos casos, definir esta transformación de manera global, es decir, en toda la transversal $\tau$. Para definir la transformación de holonomía (local), usábamos las hojas de la foliación para movernos de una transveral $\tau$ a otra transversal $\tau'$, pero para poder hacer esto, necesitábamos que la transversal $\tau$ a la hoja $L$ siguiera cortando de manera transversal a hojas suficientemente cercanas a $L$. Por lo tanto, si queremos definir una transformación de holonomía global vamos a necesitar una curva analítica que sea transversal a todas las hojas de la foliación.\\

Además, vamos a necesitar que, cuando nos movamos de un punto $x$ a un punto $y$ en una misma hoja $L$, la transversal a $L$ en el punto $x$ sea analíticamente equivalente a la tansversal a $L$ en el punto $y$. Es por eso que hacemos la siguiente definición.

\begin{defn}
\label{Def:FoliacionTransversal}
Sea $E$ una dos variedad analítica y $\Pi\colon E\rightarrow B$ un haz fibrado con fibra $F$. Diremos que una foliación $\mathcal{F}$ de $E$ es \emph{transversal a las fibras de $(E,\Pi)$} si:
\begin{enumerate}
\item Para todo $p\in E$, la hoja $L_{p}$ de $\mathcal{F}$ corta de manera transversal a $F_{\Pi(p)}$.

\item Si $L$ es una hoja de $\mathcal{F}$ entonces $\Pi\colon L\rightarrow B$ es una aplicación cubriente.
\end{enumerate}
\end{defn}

Si tenemos que la foliación $\mathcal{F}$ es transversal a las fibras de $(E,\Pi)$, por cada elemento $[\gamma]$ del grupo fundamental $\pi_{1}(B,b)$ del espacio base $B$, podemos asociar un biholomorfismo de la fibra $\varphi_{[\gamma]}\colon F\rightarrow F$ de la siguiente manera.\\

Sea $\gamma\colon [0,1]\rightarrow B$ es una lazo tal que $\gamma(0)=\gamma(1)=b$ y sea $y\in\Pi^{-1}(b)=F$. Si $L_{y}$ es la hoja de $\mathcal{F}$ que pasa por el punto $y$, entonces, como $\Pi\colon L \rightarrow B $ es una aplicación cubriente, podemos levantar a $\gamma$ a una única curva $\tilde{\gamma}\colon [0,1] \rightarrow L_{y}$ tal que, $\tilde{\gamma}(0)=y$ y $\Pi\circ\tilde{\gamma}=\gamma$. Esto quiere decir que $\Pi(\tilde{\gamma}(1))=\gamma(1)=b$ y por lo tanto $\tilde{\gamma}(1)\in\Pi^{-1}(b)=F$.\\

Así, para un punto $y\in\Pi^{-1}(b)$ podemos definir $\varphi_{\gamma}(y)=\tilde{\gamma}(1)$. Como el punto final de $\tilde{\gamma}$ sólo depende de la clase de homotopía de $\gamma$, la asociación anterior asigna a cada elemento $[\gamma]\in\pi_{1}(B,b)$ una transformación $\varphi_{[\gamma]}\colon F\rightarrow F$.\\

Gracias a que las soluciones de una ecuación diferencial dependen de manera analítica de las condiciones iniciales (Teorema \ref{Teo:ExistenciaUnicidad}), $\varphi_{[\gamma]}$ es una transformación analítica y es un biholomorfismo ya que $\varphi_{[\gamma^{-1}]}$ es una inversa analítica de $\varphi_{\gamma}$.\\

\begin{defn}
\label{Def:HolonomiaGlobal}
Sea $\mathcal{F}$ una foliación transversal a las fibras de $(E,\Pi,B)$. Si $[\gamma]\in\pi_{1}(B,b)$ y $F=\Pi^{-1}(b)$, llamaremos \emph{transformación de holonomía global} asociada a $[\gamma]$, a la transformación $\varphi_{[\gamma]}\colon F\rightarrow F$ construida arriba. 
\end{defn}

Si tenemos una separatriz $L$ de la foliación $\mathcal{F}$, y $p\in\overline{L}$ es un punto singular de $\mathcal{F}$, podemos obtener la transformación de holonomía local asociada a $L$, al restringir, a una vecindad $V\subset F$ suficientemente pequeña de $L\cap F$, la transformación de holonomía global asociada a un lazo $\gamma$ que rodee a $\Pi(p)$. Esto se debe a que el lazo $\gamma$ se levanta a $L$ como un lazo $\tilde{\gamma}$ que rodea al punto singular $p$ en la separatriz $L$ (localmente, la separatriz es un disco sin un punto). Y la transformación de holonomía local, también puede construirse levantando un lazo de $L$ a curvas en hojas de $\mathcal{F}$ suficientemente cercanas a $L$. Es precisamente la imposibilidad de levantar lazos de $L$ a cualquier hoja de $\mathcal{F}$ lo que hace a la holonomía de la sección anterior, una transformación local.\\

Lo anterior nos permite heredar algunas propiedades de la transformación de holonomía local a la transformación de holonomía global. Por ejemplo, si la transformación de holonomía local  $f$ tiene orden finito ($\mathit{i.e.},\, f^{n}=Id\,$ para algún $n\in\mathbb{N}$), entonces la transformación de holonomía global que coincide en un abierto con $f$, también tiene orden finito y los órdenes de ambas coinciden.\\

\marginpar{prueba caso particular}

Más adelante necesitaremos el siguiente resultado,

\begin{Lema}
\label{Lema:HazTopologico}
Sea $V$ una dos variedad analitica y $H\colon V \rightarrow\Omega\subset\C$ una función analítica sin puntos críticos tal que existe una foliación no singular $\mathcal{F}$ que corta de manera transversal a todas las curvas de nivel de $H$ y, para cualesquiera $a,b\in \Omega$, $H^{-1}(a)$ es compacta y topológicamente equivalente a $H^{-1}(b)$. Entonces, $(V,H,\Omega)$ es un haz topológico.
\end{Lema}
\begin{proof}
Para exhibir a las trivializaciones locales alrededor de un punto $a\in \Omega$ vamos a usar el flujo de el campo vectorial que genera a la foliación $\mathcal{F}$ en una vecindad de cada punto $z\in H^{-1}(a)$.\\
Alrededor de este punto $z$, existe una vecindad $\tilde{W_{z}}\subset V$, un disco $D_{\epsilon}$ y una función $g\colon \tilde{W_{z}}\times D_{\epsilon}$ tal que para $z_{0}\in W_{z}:=\tilde{W_{z}\cap H^{-1}(a)}$, $g(z_{0},t)$ es la hoja de la foliación $\mathcal{F}$ que pasa por $z_{0}$.\\
Sea $U_{z}=H(\tilde{W_{z}})$ y consideremos la función $H\circ g\colon W_{z}\times D_{\epsilon}\subset H^{-1}(a)\times D_{\epsilon}\rightarrow U_{z}$. Como  para toda $p\in\Omega$, $H^{-1}(p)$ corta de manera transversal a todas las hojas de la foliación $\mathcal{F}$ tenemos que,

$$\frac{\partial (H\circ g)}{\partial t}=\nabla H\frac{\partial g}{\partial t}\neq 0.$$

Así, dado $b\in U_{z}$ existe una función analítica $\varphi_{z}(b,\_)\colon W_{z}\subset H^{-1}(a)\rightarrow H^{-1}(b)$ (haciendo un abuso de notación) tal que alrededor de $z\in W_{z}$, $H\circ g(x,\varphi_{z}(b,x))=b$.\\

Si para cada $z\in H^{-1}(a)$ hacemos la construcción anterior, como $H^{-1}(a)$ es compacto, podemos encontrar una cubierta finita $W_{1},\ldots,W_{n}$ de $H^{-1}(a)$ tal que si $b\in U:=\bigcap_{i=1}^{n} H(W_{i})$, entonces existe una función analítica e inyectiva $\varphi(b,\_)\colon H^{-1}(a)\rightarrow H^{-1}(b)$. En efecto, para $x\in H^{-1}(a)$ se tiene que $x\in W_{k}$ para alguna $k$ y así podemos definir $\varphi(b,x)=\varphi_{k}(b,x)$. La función anterior está bien definida pues si $x\in W_{i}\cap W_{k},\, i\neq k$ el teorema de existencia unicidad nos asegura que las funciones $\varphi_{i}(b,\_)$ y $\varphi_{k}(b,\_)$ coinciden en el conjunto $W_{i}\cap W_{k}$ y por lo mismo, tenemos que $\varphi(b,\_)$ es inyectiva. Como $H^{-1}(a)$ es compacta, la función analítica $\varphi(b\_)$ también es suprayectiva y es por lo tanto, un biholomorfismo.\\


Como todas las curvas de nivel son homeomorfas a una superficie $F$, para cada $a\in\Omega$ hemos encontrado una vecindad $U\subset\Omega$ y un biholomorfismo $\varphi\colon U\times F\rightarrow H^{-1}(U)$ que manda a los puntos $(b,x)\in U\times F$ en el conjunto $H^{-1}(b)$. Esto quiere decir que $\varphi^{-1}$ es una trivialización local de $(V,H,\Omega)$.\\
\end{proof}

\noindent Por último, observemos que si fijamos $x_{0}\in H^{-1}(a)$ y $L_{x_{0}}$ es la hoja de $\mathcal{F}$ que pasa por $x_{0}$, entonces $\varphi(t,x_{0})\subset L_{x_{0}}\cap H^{-1}(U)$. Esto quiere decir que,
$$\varphi^{-1}(L_{x_{0}}\cap H^{-1}(U))=\bigsqcup_{x\in L_{x_{0}}\cap H^{-1}(U)}U\times \{x\}.$$ Como de manera local tenemos que $H=\Pi_{1}\circ \varphi^{-1}$ entonces, si $L$ es una hoja de la foliación $\mathcal{F}$, $H\colon L\rightarrow \Omega$ es una aplicación cubriente.

\section{El plano proyectico complejo $\C\mathbb{P}^{2}$.}

El espacio proyectivo complejo $\C\mathbb{P}^{n}$ es el conjunto de todos los subespacios uno dimensionales de $\C^{n+1}\setminus\{0\}$. Si en este conjunto consideramos la relación de equivalencia $x\sim y$ sí y sólo sí $x=\lambda y, \lambda\in\C\setminus\{0\}$, entonces $\C\mathbb{P}^{n} = \C^{n+1}\setminus\{0\} / \sim$.\\

En el caso $n=2$ obtenemos el plano proyectivo complejo, y en el caso $n=1$ obtenemos la recta proyectiva compleja que es nada más y nada menos que una esfera.\\

Denotaremos por $[x_{0}:y_{0}:z_{0}]$ a la clase de $(x_{0},y_{0},z_{0})$ $\mathit{i.e.}$ $[x_{0}:y_{0}:z_{0}]:=\{(x,y,z)\in\C^{3}\setminus\{0\} \mid \exists\lambda\in\C,(x_{0},y_{0},z_{0})=\lambda (x,y,z)\}$. A esta manera de denotar los puntos de $\CP$ se le conoce como coordendas homogéneas.\\

Con la topología cociente, $\CP$ tiene una estructura natural de  dos variedad analítica dada por las siguientes cartas: los abiertos que usaremos son $U_{x}=\{[x:y:z] \mid x\neq 0\},U_{y}=\{[x:y:z] \mid y\neq 0\},U_{z}=\{[x:y:z] \mid z\neq 0\}$ y los homemorfismos son:

$$
\begin{matrix}
\psi_{x} \colon & U_{x} &  \rightarrow & \C^{2}\\
& [x:y:z] &  \mapsto &  (\frac{y}{x},\frac{z}{x})
\end{matrix}
$$

$$
\begin{matrix}
\psi_{y} \colon & U_{y} &  \rightarrow & \C^{2}\\
& [x:y:z] &  \mapsto &  (\frac{x}{y},\frac{z}{y})
\end{matrix}
$$

$$
\begin{matrix}
\psi_{z} \colon & U_{z} &  \rightarrow & \C^{2}\\
& [x:y:z] &  \mapsto &  (\frac{x}{z},\frac{y}{z})
\end{matrix}
$$\\

Podemos pensar al mapeo $\psi_{z}$ como si a cada punto $[x:y:z]$ de $U_{z}\subset\CP$ lo mandáramos al representante $(\tfrac{x}{z},\tfrac{y}{z},1)$; a esta carta sólo le falta cubrir a las clases $[x:y:0]$. Si nos quedamos solamente con las primeras dos coordenadas $[x:y]$, estos puntos forman un $\C\mathbb{P}^{1}$, que como ya hemos mencionado es una esfera.\\

Así, podemos pensar a $\CP$ como un plano complejo $\C^{2}$ al cual, le hemos pegado una recta proyectiva, la recta al infinito.\\ 
 
La mayoría del tiempo trabajaremos en las coordenadas $\psi_{z}(U_{z})=\C^{2}$, pero a veces necesitaremos ir a las otras dos cartas para ver el comportamiento de algunas cosas que pasan en la recta al infinito.\\

Los cambios de coordenadas de $\psi_{x}(U_{x})$ a $\psi_{z}(U_{z})$ y de $\psi_{y}(U_{y})$ a $\psi_{z}(U_{z})$ están dados por:

\begin{equation} 
%\tag{$\dagger$}
\label{CambiosDeCoordenadas}
\begin{aligned}
\psi_{z} \circ \psi^{-1}_{x}(u,v) = (\frac{1}{u},\frac{v}{u})\\
\psi_{z} \circ \psi^{-1}_{y}(u,v) = (\frac{u}{v},\frac{1}{v}).
\end{aligned}
\end{equation}


\section{El teorema de linealización de Poinacaré.}
\label{sec:LinenalizacionDePoincare}
Como mencionamos anteriormente, un problema fundamental de la teoría de ecuaciones diferenciales es simplificar, mediante un cambio de coordenadas, un campo vectorial $F$. Si este campo vectorial tiene parte lineal, una pregunta natural es cuándo este campo vectorial es analíticamente equivalente a su parte lineal. El teorema de linealización de Poincaré nos dice bajo qué condiciones un campo vectorial es analíticamente equivalente a su parte lineal.\\

Sea $F(x)=Ax+V_{2}(x)+\cdots+V_{m}(x)+\cdots$ donde $A=(\tfrac{\partial F}{\partial x})(0)$ es la parte lineal del campo vectorial en el origen y los $V_{i}$ son campos vectoriales homogéneos de grado $i$. Una manera de atacar el problema de linealización es intentar eliminar el término $V_{2}$ mediante un biholomorfismo $H_{2}$. Una vez logrado ésto procedemos a eliminar el término $V_{3}$ con un biholomorfismo $H_{3}$ y así sucesivamente.\\

Supongamos que tenemos un campo vectorial  $F=(F_{1},F_{2})$ del siguiente estilo:

\begin{equation}
\label{CampoVectInicial}
\begin{aligned}
F_{1} &= \lambda_{1}x +ax^{\alpha}y^{\beta}+O(||(x,y)||^{\alpha+\beta})\\
F_{2} &= \lambda_{2}y + O(||(x,y)||^{\alpha+\beta}).
\end{aligned}
\end{equation}

\noindent El objetivo es encontrar un biholomorfimso $H\colon (\C^{2},0)\rightarrow(\C^{2},0)$ que conjugue este campo vectorial con un campo vectorial $\tilde{F}=(\tilde{F}_{1},\tilde{F}_{2})$ de la forma:

\begin{equation}
\label{CampoVectFinal}
\begin{aligned}
\tilde{F}_{1} &= \lambda_{1}x +O(||(x,y)||^{\alpha+\beta})\\
\tilde{F}_{2} &= \lambda_{2}y + O(||(x,y)||^{\alpha+\beta}).
\end{aligned}
\end{equation}

\noindent Para lograr esto, el biholomorfismo $H$ y los campos vectoriales tienen que statisfacer la igualdad:
\begin{equation}
\label{NuevaEquivalenciaAnalitica}
\Big(\frac{\partial H}{\partial x}\Big)F(x)=\tilde{F}(H(x)).
\end{equation}


\noindent Si hacemos $H(x,y)=(x,y)+c(x^{\alpha}y^{\beta},0)$ el lado izquierdo de la ecuación (\ref{NuevaEquivalenciaAnalitica}) es:

\begin{equation*}
(Id+c\begin{pmatrix} \alpha x^{\alpha -1}y^{\beta} & \beta x^{\alpha}y^{\beta -1} \\ 0 & 0 \end{pmatrix})\begin{pmatrix}\lambda_{1}x +ax^{\alpha}y^{\beta}+O(||(x,y)||^{\alpha+\beta})\\ \lambda_{2}y \end{pmatrix}.
\end{equation*} 

\noindent La primera entrada de este vector es,

\begin{equation*}
\lambda_{1}x +ax^{\alpha}y^{\beta}+cx^{\alpha}y^{\beta}(\alpha\lambda_{1} +\beta\lambda_{2}) +O(||(x,y)||^{\alpha+\beta}).
\end{equation*} 
%\begin{pmatrix}\lambda_{1}(x+p(x,y)) + O(||(x,y)||^{\alpha+\beta}) \\ \lambda_{2}y\end{pmatrix}.

\noindent La primera coordenada del campo vectorial $\tilde{F}(H(x,y))=\tilde{F}((x,y)+c(x^{\alpha}y^{\beta},0))$ es,

\begin{equation*}
\lambda_{1}(x+cx^{\alpha}y^{\beta}) + O(||(x,y)||^{\alpha+\beta}).
\end{equation*}

\noindent Si queremos que el bihlomorfismo $H$ conjugue a los campos vectoriales $F$ y $\tilde{F}$, los monomios de grado uno y de grado $\alpha + \beta$ de ambos campos vectoriales deben ser iguales. Así,

\begin{equation}
a+c(\alpha\lambda_{1}+\beta\lambda_{2})=\lambda_{1}c.
\end{equation}

En caso de que $\lambda_{1}-\alpha\lambda_{1}+\beta\lambda_{2}\neq 0$ podemos despejar a $c$ y obtener:

%\begin{equation*}
%\lambda_{1}x +(a+c(\alpha\lambda_{1} +\beta\lambda_{2}))x^{\alpha}y^{\beta} +O(||(x,y)||^{\alpha+\beta +1}) = \lambda_{1}x + \lambda_{1}cx^{\alpha}y^{\beta}+O(||(x,y)||^{\alpha+\beta +1}).
%\end{equation*}

\begin{equation}
c = \frac{a}{\lambda_{1} -(\alpha\lambda_{1} + \beta\lambda_{2})}.
\end{equation}

Con nuestra elección de $H$, es claro que la segunda coordenada de ambos vectores de la ecuación (\ref{NuevaEquivalenciaAnalitica}). Por lo tanto, si $\lambda_{1}-\alpha\lambda_{1}+\beta\lambda_{2}\neq 0$, existe un biholomorfismo $H$ que conjuga a los campos vectoriales $F$ y $\tilde{F}$.
\begin{defn}
\label{Def:Resonancias}
Diremos que la pareja $\lambda=(\lambda_{1},\lambda_{2})\in\C^{2}$ es \emph{resonante} si existen números naturales $\alpha_{1},\alpha_{2}$, $\alpha_{1}+\alpha_{2}>1$ tales que:
\begin{equation}
\label{Denominadores}
\lambda_{j}=\alpha_{1}\lambda_{1}+\alpha_{2}\lambda_{2}.
\end{equation}
\end{defn}

En el cáclulo anterior si remplazamos el campo vectorial $F$ por un campo vectorial de la forma,

\begin{equation}
\begin{aligned}
F_{1} &= \lambda_{1}x +V_{\alpha+\beta}(x,y)+O(||(x,y)||^{\alpha+\beta})\\
F_{2} &= \lambda_{2}y + O(||(x,y)||^{\alpha+\beta}),
\end{aligned}
\end{equation}

\noindent donde $V_{\alpha+\beta}(x,y)$ es un polinomio homogéneo de grado $\alpha+\beta$ en el cual aparece el monomio $ax^{\alpha}y^{\beta}$, no es difícil convencerse de que $H(x,y)=(x,y)+(cx^{\alpha}y^{\beta},0)$ conjuga a este campo vectorial con:
 
\begin{equation}
\begin{aligned}
\tilde{F}_{1} &= \lambda_{1}x + V_{\alpha+\beta}-ax^{\alpha}y^{\beta}+O(||(x,y)||^{\alpha+\beta})\\
\tilde{F}_{2} &= \lambda_{2}y + O(||(x,y)||^{\alpha+\beta}).
\end{aligned}
\end{equation}

Una vez eliminado el monomio $ax^{\alpha}y^{\beta}$, si la pareja $(\lambda_{1},\lambda_{2})$ es no resonante, podemos eliminar los demás monomios de grado $\alpha+\beta$ en la primera coordenada. Si en la segunda coordenada también hay monomios de grado $\alpha+\beta$, con un procedimiento similar podemos eliminarlos. \\ 
Si los valores propios de $A$ son no resonantes entonces podemos eliminar cualquier término no lineal de $F$ mediante un biholomorfismo pero, si queremos eliminar todos los términos no lineales de un jalón, tendremos que componer una infinidad de biholomorfismos y el resultado no siempre será un biholomorfismo sino simplemente una serie de potencias formal. Esto se debe a que los números $\lambda_{j}-\alpha_{1}\lambda_{1}-\alpha_{2}\lambda_{2}$ pueden ser muy pequeños y en consecuencia, al dividir por ellos, podemos afectar la convergecia de la serie de potencias que linealiza (de manera formal) al campo vectorial. Es por eso que si queremos que la serie de potencias converja, tenemos que pedir algo adicional a los valores propios de $A$.

\begin{defn}
\label{Def:DominioPoincare}
Sean $\lambda_{1},\lambda_{2}\in\C$, si la cerradura convexa del conjunto $\{ \lambda_{1},\lambda_{2}\}\subset\C$ no contiene al origen, diremos que $\lambda=(\lambda_{1},\lambda_{2})$ está en el \emph{dominio de Poincaré}. En caso contrario, diremos que $\lambda$ está en el \emph{dominio de Siegel}.
\end{defn}

Si pensamos a los número complejos $\lambda_{1},\lambda_{2}$ como vectores en $\mathbb{R}^{2}$, estar en el dominio de Siegel se traduce a que $\lambda_{1}$ y $\lambda_{2}$ sean linealmente dependientes y de sentidos opuestos.\\

Con las dos definiciones anteriores podemos enunciar el teorema de linealización de Poincaré.
\marginpar{referencia}
\begin{Teorema}
Si los valores propios de la parte lineal del campo vectorial $F$ son no resonantes y están en el dominio de Poincaré, entonces $F$ es analíticamente equivalente a su parte lineal.
\end{Teorema}

\begin{Ejemplo}
\label{Ej:LinealizacionRadial}
Si los valores propios de la parte lineal de un campo vectorial $F$ son iguales y distintos de cero, entonces el campo vectorial $F$ es analíticamente equivalente a su parte lineal. Esto se debe a que si $\lambda_{1}=\lambda_{2}$ entonces la ecuación (\ref{Denominadores}) sólo tiene solución con $\alpha_{1}=1$ y $\alpha_{2}=0$ y en consecuencia la pareja $(\lambda_{1},\lambda_{2})$ es no resonante. Además como los valores propios coinciden, su cerradura convexa es un único punto $\lambda_{1}\neq 0$. 
\end{Ejemplo}

\section{Explosión de singularidades.}  

En la sección anterior enunciamos bajo qué condiciones un campo vectorial, con parte lineal, es analíticamente equivalente a su parte lineal pero, ¿y si nuestro campo vectorial no tiene parte lineal? Cuando esto sucede, una técnica muy utilizada es la explosión de singularidades. A continuación describimos este proceso.\\

Intuitivamente, si un campo vectorial no tiene parte lineal en el origen, el campo vectorial aplasta todos los subespacios 1-dimensionales que salen del origen. Es por eso que vamos a intentar ``separar'' a estos subespacios 1-dimensionales.\\

Consideremos la proyección canónica de $p\colon\C^{2}\setminus\{0\}\rightarrow\C\mathbb{P}^{1}$. Esta proyección asocia a cada punto $(x,y)\in\C^{2}\setminus\{0\}$ su clase de equivalencia $[x:y]\in\C\mathbb{P}^{1}$, o lo que es lo mismo, $p$ envía a todo un subespacio 1-dimensional menos el origen en un solo punto de $\C\mathbb{P}^{1}$.\\

La gráfica de esta función $graf(p)$ es un subconjunto del producto $\C^{2}\times\C\mathbb{P}^{1}$ y está determinada por los puntos de la forma $((x,y),[x:y])$. En esta gráfica, todos los subespacios 1-dimensionales de $\C^{2}$ ya están ``separados'' pues a subespacios 1-dimensional distintos de $\C^{2}$ les corresponden puntos distintos en $\C\mathbb{P}^{1}$ .\\

La proyección $\Pi\colon\C^{2}\times\C\mathbb{P}^{1}\rightarrow\C^{2}$ es una función biyectiva entre $graf(p)$ y $\C^{2}\setminus\{0\}$ y además la imagen inversa del origen es \emph{el divisor excepcional} $E:=\Pi^{-1}(0)=\{0\}\times\C\mathbb{P}^{1}\simeq\C\mathbb{P}^{1}$. Si con una transformación queremos alterar la parte lineal de un campo vectorial, no podemos utilizar transformaciones analíticas invertibles, ya que éstas preservan la parte lineal del campo vectorial.

Observemos que si $M:=graf(p)\cup E$, entonces, $\Pi\colon M\rightarrow\C^{2}$ es una función biyectiva entre $M\setminus E$ y $\C^{2}\setminus\{0\}$, la imagen inversa del origen es toda una curva y además, en $M$ ya están ``separados'' los subespacios 1-dimensionales de $\C^{2}$.\\

Por todo lo anterior, la pareja $(M,\Pi)$ es un buen candidato para desingularizar un campo vectorial. Pero todo el proceso de desingularización lo debemos llevar acabo de manera analítica, es por eso que debemos darle a $M$ una estructura de variedad analítica y mostrar que con esa estructura, $\Pi$ es una función analítica.\\   

Si usamos la carta $u=\tfrac{w}{z}$ de $\C\mathbb{P}^{1}$, $M$ queda descrita por los puntos que satisfacen $u=\tfrac{y}{x}\Rightarrow y=ux$ (esta última ecuación también inculye a los puntos de $E$).

En estas coordenadas la función $\varphi((x,y),[z:w])=(x,\tfrac{w}{z})=(x,u)$ restringida a $M$ es una carta de $M$. Como en los puntos de $M$ se cumple que $y=ux$, la función inversa de esta carta es $\varphi^{-1}(x,u)=((x,ux),u)=((x,ux),[z:w])$.\\

De manera análoga, en la otra carta de $\C\mathbb{P}^{1}$, $v=\tfrac{x}{y}$, $M$ queda descrita por $x=vy$ y la función $\phi((x,y),[z:w])=(y,\tfrac{w}{x})=(y,v)$ es otra carta de $M$ cuya inversa es $\phi^{-1}(y,v)=((vy,y),v)$\\

Como las cartas $((x,y),u)$ y $((x,y),v)$ cubren a todo $\C^{2}\times\C\mathbb{P}^{1}$, las cartas $(x,u)$ y $(y,v)$ cubren a $M$ y el cambio de coordenadas de $(x,u)$ a $(y,v)$ está dado por:

\begin{equation}
\label{CambiosCoordenadasExplosion}
\phi\circ\varphi^{-1}(x,u)=(ux,\tfrac{1}{u}).
\end{equation}

Las dos cartas anteriores hacen de $M$ una dos variedad analítica y en la carta $(x,u)$ la proyección $\Pi\colon M\rightarrow\C^{2}$ adquiere la forma $\Pi(x,u)=(x,ux)$ ya que en esta carta $y=ux$. Mientras que en la otra carta, $\Pi(y,v)=(vy,y)$.\\

Lo anterior prueba que, con la estructura analítica que le hemos dado a $M$, $\Pi$ es una función analítica y entonces podemos usarla para jalar 1-formas en $\C^{2}$ (y en consecuencia foliaciones y campos vectoriales) a $M$.\\

Si tenemos una 1-forma $\omega$ con punto singular aislado en el origen, esta 1-forma define una foliación no singular $\mathcal{F}$ en $\C^{2}\setminus\{0\}$. Así, $\Pi^{*}(\omega)$ define una foliación no singular $\Pi^{*}(\mathcal{F})$ en $M\setminus E$. Pero gracias al Teorema (\ref{Teo:ExtensionFoliaciones}) podemos extender a $\Pi^{*}(\mathcal{F})$ a todo $M$ como una foliación singular con puntos singulares aislados en el divisor excepcional $E$. Al proceso anterior se le conoce como \emph{explosión de la 1-forma} $\omega$ ó \emph{explosión de la foliación} $\mathcal{F}$. A esta técnica de explosión de singularidades también se le conoce como \emph{desingularización}.

\begin{Ejemplo}
\label{Ej:BlowUpLineal}
Haremos el blow-up de la 1-forma $\omega=x\, dy-\lambda y\, dx,\lambda\in\C$ en la carta $(x,u)$.\\

$\Pi^{*}(\omega)=x\, d(ux)-\lambda ux\, dx=x(u\, dx +x\, du)-\lambda ux\, dx=x(x\, du-(\lambda-1)u\, dx)$.\\

$\Pi^{*}(\omega)$ define una foliación no singular en $M\setminus E$ y como en esta carta $E=\{x=0\}$, la función $x$ no se anula en $M\setminus E$. Así, podemos multiplicar a la 1-forma $\Pi^{*}(\omega)$ por $x$ para obtener una nueva 1-forma $\omega_{1}=x\, du-(\lambda-1)u\, dx$ que define la misma foliación que $\Pi^{*}(\omega)$ en $M\setminus E$ pero tiene singularidades aisladas en el divisor excepcional $E$.
\end{Ejemplo}

%%La 1-forma que explotamos en el ejemplo anterior se corresponde con el campo vectorial:

%\begin{equation}
%\begin{aligned}
%\dot{x}&=x\\
%\dot{y}&=\lambda y.
%\end{aligned}
%\end{equation}

%Y el campo vectorial que obtenemos en la carta $(u,x)$ al explotar es:

%\begin{equation}
%\begin{aligned}
%\dot{x}&=x\\
%\dot{u}&=(\lambda-1) u.
%\end{aligned}
%\end{equation}

Un caso particular del ejemplo anterior que utilizaremos más adelante es cuando $\lambda=1$. En este caso $\Pi^{*}(\omega)=x^{2}du$, pero en $M\setminus E$ la función $x^{2}$ tampoco se anula y por lo tanto podemos multiplicar por ella para así obtener la 1-forma $\omega_{1}=du$.

Esta 1-forma no tiene puntos singulares y las hojas de la foliación que determina quedan descritas por $u=cte$. Usando la otra carta $(y,v),\ v=\tfrac{x}{y}$ podemos ver que, en la parte de $M$ que la carta $(x,u)$ no nos permite ver, tampoco hay puntos singulares. Es decir, al explotar la foliación generada por $\omega=x\, dy-y\, dx$ obtenemos una foliación en $M$ sin puntos singulares.\\

En este caso particular, $\lambda=1$, el campo vectorial que se corresponde con la 1-forma $\omega=x\, dy-y\, dx$ es el campo vectorial radial:

\begin{equation}
\begin{aligned}
\dot{x}&=x\\
\dot{y}&=y.
\end{aligned}
\end{equation}

Una observación muy importante es que todo el proceso de desingularización puede llevarse acabo de manera local (sólo hay que restringir toda la construcción a una vecindad del origen $(\C^{2},0)$). Si tenemos una foliación $\mathcal{F}$ con un número finito de singularidades $\Sigma$ en una dos variedad analítica $M$, la observación anterior nos permite hacer una explosión local en cada uno de los puntos singulares para así obtener una nueva variedad $M'$ y una función holomorfa $\Pi\colon M' \rightarrow M$ que satisface las siguientes propiedades:

\begin{enumerate}
\item Si $p\in\Sigma$ entonces $\Pi^{-1}(p):=E_{p}\simeq\C\mathbb{P}^{1}$.
\item $\Pi$ es un biholomorfimso entre $M'\setminus \bigcup_{p\in\Sigma} E_{p}$ y $M\setminus\Sigma$.
\end{enumerate}

La variedad $M'$ se puede construir explotando un punto $p_{1}\in\Sigma$, obteniendo así una variedad $M_{1}$, un divisor $E_{p_{1}}$ y un mapeo $\Pi_{1}\colon M_{1}\rightarrow M$. Como $\Pi_{1}$ es un biholomorfismo entre $M_{1}\setminus E_{p_{1}}$ y $M\setminus \{p_{1}\}$, alguna vecindad de otro punto $p_{2}\in\Sigma\setminus \{p_{1}\}$ se mapea de manera biholomorfa a la nueva variedad $M_{1}$. Así, podemos aplicar el procedimiento anterior al punto $\Pi^{-1}_{1}(p_{2})$ para obtener otra variedad $M_{2}$, otro divisor $E_{p_{2}}$ y otro mapeo $\Pi_{2}\colon M_{2}\rightarrow M_{1}$ que es un biholomorfismo entre $M_{2}\setminus E_{p_{2}}$ y $M_{1}\setminus \{\Pi^{-1}_{1}(p_{2})\}$.\\ 

Si repetimos este procedimiento con todos los puntos restantes de $\Sigma$, como $\Sigma$ es un conjunto finito, al final obtendremos una variedad $M_{n}$, un divisor $E_{p_{n}}$ y un mapeo $\Pi_{n}\colon M_{n}\rightarrow M_{n-1}$ que es un biholomorfismo entre $M_{n}\setminus E_{p_{n}}$ y $M_{n-1}\setminus \{\Pi^{-1}_{n-1}(p_{n})\}$. Si llamamos $M'$ a $M_{n}$ y $\Pi$ a $\Pi_{1}\circ\cdots\circ\Pi_{n}$, la variedad $M'$ y el mapeo $\Pi$ satisfacen las propiedades deseadas.   
 
\section{El grado de una foliación dicrítica en $\C\mathbb{P}^{2}$.}

Dada una foliación holomorfa del plano proyectivo complejo $\CP$ se tiene que, como consecuencia del teorema de Chow (el cual afirma que todo subconjunto analítico de una variedad proyectiva es algebraico, ver \cite{Mumford}) , ésta es generada, en cualquier carta afín, por un campo vectorial polinomial \cite[p.~477]{IlyaYako}.\\

Si en la carta afín $(x,y)$, la foliación está generada por el campo vectorial:

\begin{equation}
\label{EcuacionEnCP2}
\begin{aligned}
\dot{x} &=p_{1}(x,y)+\cdots+p_{n}(x,y)\\
\dot{y} &=q_{1}(x,y)+\cdots+q_{n}(x,y),
\end{aligned}
\end{equation}

\noindent donde los $p_{k},q_{k}$ son polinomios homogéneos de grado $k$, entonces, para obtener un campo vectorial que genere a la foliación en una vecindad de la recta al infinito, consideramos el cambio de coordenadas $x=\tfrac{1}{u}$ y $y=\tfrac{v}{u}$. En estas coordenadas el campo vectorial se expresa como:

\begin{equation}
\begin{aligned}
\dot{u} &=\frac{1}{u^{n+1}}p_{n}(1,v)+\cdots+\frac{1}{u^{2}}p_{1}(1,v)\\
\dot{v} &=\frac{1}{u_{n+2}}(vp_{n}(1,v)-q_{n}(1,v))+\cdots+\frac{1}{u^{3}}(vp_{1}(1,v)-q_{1}(1,v)).
\end{aligned}
\end{equation}

Si llamamos $h_{k+1}=yp_{k}(x,y)-xq_{k}(x,y)$ podemos escribir la ecuación anterior como:

\begin{equation}
\label{EcuacionEnCP2Infinito}
\begin{aligned}
\dot{u} &=\frac{1}{u^{n+1}}p_{n}(1,v)+\cdots+\frac{1}{u^{2}}p_{1}(1,v)\\
\dot{v} &=\frac{1}{u_{n+2}}h_{n+1}(1,v)+\cdots+\frac{1}{u^{3}}h_{2}(1,v).
\end{aligned}
\end{equation}

En esta carta, la recta al infinito queda descrita por $\{u=0\}$ y entonces, antes de obtener la expresión final del campo vectorial cerca del infinito, podemos multiplicar por una potencia de $u$ adecuada para eliminar los polos. Así, tenemos dos casos distintos:

\begin{enumerate}

\item Si $h_{n+1}\neq 0$, entonces diremos que la foliación es \emph{no dicrítica} y así, podemos multiplicar por $u^{n+2}$ para obtener el campo vectorial:
\begin{equation}
\label{EcuacionNoDicritica}
\begin{aligned}
\dot{u} &=up_{n}(1,v)+\cdots+u^{n}p_{1}(1,v)\\
\dot{v} &=h_{n+1}(1,v)+uh_{n}(1,v)+\cdots+u^{n-1}h_{2}(1,v).
\end{aligned}
\end{equation}

Observemos que en este caso, al hacer el cambio de coordenadas, el grado de los polinomios que definen la foliación en la nueva carta es uno más que el grado de los polinomios que definen la foliación en la carta inicial. Además, la recta al infinito $\{u=0\}$ es invariante y tiene singularidades en los puntos $(0,v_{j})$ donde $v_{j}$ es una raíz del polinomio $h_{n+1}(1,v)$.

\item Si $h_{n+1}\equiv 0$, diremos que la foliación es \emph{dicrítica} y entonces basta multiplicar por $u^{n+1}$ para obtener:

\begin{equation}
\label{EcuacionDicriticaInfinito}
\begin{aligned}
\dot{u} &=p_{n}(1,v)+\cdots+u^{n-1}p_{1}(1,v)\\
\dot{v} &=h_{n}(1,v)+\cdots+u^{n-2}h_{2}(1,v).
\end{aligned}
\end{equation}

En este caso volvemos a obtener un campo vectorial polinomial del mismo grado que el campo vectorial original pero ahora la recta al infinito ya no es invariante. En efecto, la foliación corta de manera transversal a la recta al infinito salvo en las raíces del polinomio $p_{n}(1,v)$, en estos puntos tenemos tangencias que se vuelven puntos singulares si $p_{n}(1,v)$ y $h_{n}(1,v)$ tienen raíces en común.
\end{enumerate}

En el caso dícritico, el polinomio $h_{n+1}$ se anula y esto se traduce a que

\begin{equation}
\label{CondicionDicritica}
yp_{n}(x,y)=xq_{n}(x,y).
\end{equation}

Si evaluamos esta expresión en los puntos $(1,y)$ obtenemos $yp_{n}(1,y)=q_{n}(1,y)$ y entonces, el polinomio $p_{n}(1,y)$ es de grado estrictamente menor a $n$ o lo que es lo mismo, $p_{n}(x,y)$ no tiene monomios de la forma $ay^{n}$. Un razonamiento similar nos permite concluir que $q_{n}(x,y)$ no tiene monomios de la forma $bx^{n}$.

Como el polinomio $p_{n}$ es homogéneo de grado $n$ y no hay monomios de la forma $ay^{n}$, todos los monomios de $p_{n}$ tienen una potencia de $x$ y por la misma razón, todos los monomios del polinomio $q_{n}$ tienen una potencia de $y$. Así, podemos escribir a $p_{n}$ y $q_{n}$ como

\begin{equation}
\begin{aligned}
p_{n}(x,y) &=xf(x,y)\\
q_{n}(x,y) &=yg(x,y).
\end{aligned}
\end{equation}

 \noindent Además, los polinomios $f$ y $g$ son homogéneos de grado $n-1$. Si insertamos estas dos últimas igualdades en (\ref{CondicionDicritica}) obtenemos:

\begin{equation}
\label{A}
xyf(x,y)=xyg(x,y)
\end{equation}

\noindent Entonces, al cancelar el factor $xy$ de ambos lados de la igualdad (\ref{A}) se tiene que,

\begin{equation}
f(x,y)=g(x,y).
\end{equation}

Todo lo anterior quiere decir que, una foliación dicrítica siempre la podemos escribir de la forma:

\begin{equation}
\label{EcuacionDicritica}
\begin{aligned}
\dot{x} &=p_{1}(x,y)+\cdots+p_{d}(x,y)+xg(x,y)\\
\dot{y} &=q_{1}(x,y)+\cdots+q_{d}(x,y)+yg(x,y),
\end{aligned}
\end{equation}

\noindent con $g(x,y)$ un polinomio homogéneo de grado $d$.\\

En el caso de una foliación no dicrítica, vimos que el grado de los polinomios que definen a la foliación no son invariantes bajo cambios de coordenadas. Es por eso que si queremos asociar un grado a una foliación de $\C\mathbb{P}^{2}$ debemos encontrar otra manera de mirar a una foliación de $\C\mathbb{P}^{2}$.\\

Si en $\C^{3}$ consideramos una 1-forma $\Omega=A\, dx+B\, dy+C\, dz$, donde $A,B,C\in\C[x,y,z]$ son polinomios homogéneos de grado $d+1$, el conjunto $\{\Omega=0\}$ define una distribución de planos en $\C^{3}$. Podemos pensar que $\Omega$ asocia a un punto $(x_{0},y_{0},z_{0})$ el kernel de la transformación lineal $A(x_{0},y_{0},z_{0})x+B(x_{0},y_{0},z_{0})y+C(x_{0},y_{0},z_{0})z$.\\

Si queremos que $\Omega$, al proyectar en la carta afín de $\C\mathbb{P}^{2},\ z=1$ siga asociando a cada punto $[x_{0}:y_{0}:z_{0}]$ ya no un plano sino una recta, el plano original debe contener a la dirección determinada por $[x_{0}:y_{0}:z_{0}]$. Por ejemplo, si $\Omega$ asocia al punto $(0,0,1)$ el plano $x+z=0$, al proyectar en el plano $z=1$, el plano $x+z=0$ se proyecta en la recta $x=-1$ y esta recta ni siquiera pasa por el punto $(0,0)$ que es el representante de la clase $[0:0:1]$ en el plano $z=1$.\\

Una manera de evitar lo anterior es pedir que:

\begin{equation}
\label{EulerFieldCondition}
xA(x,y,z)+yB(x,y,z)+zC(x,y,z)=0 \,\, \forall x,y,z\in\C^{3}.
\end{equation}

\noindent Lo anterior es equivalente a pedir que la distribución de planos que define $\{\Omega=0\}$ contenga al campo vectorial radial:

\begin{equation}
\label{RadialVectorField}
V=x\frac{\partial}{\partial x}+y\frac{\partial}{\partial y}+z\frac{\partial}{\partial z}.
\end{equation}

\noindent Así, cualquier 1-forma $\Omega=A\, dx+B\, dy+C\, dz$ con coeficientes polinomiales homogéneos de grado $d+1$ que satisface la condición (\ref{EulerFieldCondition}), define una foliación de $\C\mathbb{P}^{2}$ que en la carta afín $z=1$ adquiere la forma:

\begin{equation}
\label{FormaAfin}
\omega=A(x,y,1)\, dx+B(x,y,1)\, dy.
\end{equation}

\noindent Observemos que el coeficiente $C$ desaparece por que si $z=1$, entonces $dz=0$.\\

Recíprocamente, si en una carta afín de $\C\mathbb{P}^{2}$ una foliación está generada por la 1-forma $\omega=p(x,y)\, dx+q(x,y)\, dy$ con $p,q\in\C[x,y]$ polinomios de grado $d$, entonces podemos levantar $\omega$ a una 1-forma $\Omega$ de $\C^{3}$ con coeficientes polinomiales homogéneos que satisface la identidad (\ref{EulerFieldCondition}). En efecto, si escogemos los coeficientes $A$ y $B$ como:

\begin{align}
A(x,y,z) &=z^{d+1}p(\tfrac{x}{z},\tfrac{y}{z})\\
B(x,y,z) &=z^{d+1}q(\tfrac{x}{z},\tfrac{y}{z}).
\end{align}

\noindent Entonces, la identidad (\ref{EulerFieldCondition}) obliga a que:

\begin{equation}
C(x,y,z) = z^{-1}(xA(x,y,z)+yB(x,y,z)).
\end{equation}

\noindent Es decir, para obtener a los polinomios homogéneos $A$ y $B$ de los polinomios $p$
y $q$ respectivamente, homogeneizamos a los polinomios $p$ y $q$ y después los multiplicamos por $z$.\\

Una observación muy importante que nos va a permitir definir el grado de una foliación en $\C\mathbb{P}^{2}$ es que, el grado de una 1-forma polinomial $\Omega$ de $\C^{3}$ es invariante bajo transformaciones del grupo general lineal $GL(3,\C)$ y en consecuencia es invariante bajo el grupo de transformaciones proyectivas $PGL(3,\C)$ de $\C\mathbb{P}^{2}$.  

\begin{defn}
\label{GradoDeUnaFoliacion}
Sea $\mathcal{F}$ una foliación de $\C\mathbb{P}^{2}$. Si en coordenadas homogéneas esta foliación está generada por la 1-forma $\Omega=A\, dx+B\, dy+ C\, dz$ con coeficientes polinomiales homogéneos de grado $d+1$ diremos que la foliación $\mathcal{F}$ tiene \emph{grado} $d$. 
\end{defn} 

Un resultado que será muy importante, es que toda foliación de grado $d$ tiene, contando multiplicidades, $d^{2}+d+1$ puntos singulares. Es por eso que apesar de que la foliación esté generada, en coordenadas homogéneas, por una 1-forma de grado $d+1$ diremos que la foliación tiene grado $d$. Cabe mencionar que algunos autores no hacen esta convención y definen el grado de una foliación como el grado de la 1-forma que genera a la foliación en coordenadas homogéneas.\\

A continuación veremos algunas propiedades que cumplen los polinomios $A,B,C$ de la 1-forma $\Omega=A\, dx+B\, dy+C\, dz$ que nos van a ayudar a probar que el número de puntos singulares que tiene la foliación generada por $\{ \Omega=0\}$ es $d^{2}+d+1$.\\

Si tenemos una foliación dada por la 1-forma $\Omega=A\, dx+B\, dy+C\, dz$ entonces podemos escribir a $A$ y a $B$ de la siguiente manera:
\begin{equation}
\label{CoeficientesRespectoAz}
\begin{aligned}
A &=a_{d+1}+a_{d}z+\cdots+a_{1}z^{d}+a_{0}z^{d+1}\\
B &=b_{d+1}+b_{d}z+\cdots+b_{1}z^{d}+b_{0}z^{d+1},
\end{aligned}
\end{equation}

\noindent donde los $a_{k},b_{k}\in\C[x,y]$ son polinomios homogéneos de grado $k$. Entonces tenemos que:

\begin{equation}
xA+yB = (xa_{d+1}+yb_{d+1})+(xa_{d}+yb_{d})z+\cdots+(xa_{1}+yb_{1})z^{d}+(xa_{0}+yb_{0})z^{d+1}.
\end{equation}

\noindent De la indentidad (\ref{EulerFieldCondition}) tenemos que $-zC=xA+yB$. Es decir que $z$ divide al polinomio $xA+yB$ y en consecuencia tenemos que:

\begin{equation}
\label{CondicionImportante}
 xa_{d+1}+yb_{d+1}\equiv 0.
\end{equation}

\noindent Razonando de manera análoga a como hicimos con la ecuación (\ref{CondicionDicritica}) y suponiendo que alguno de los polinomios $a_{d+1},b_{d+1}$ no es idénticamente cero, podemos concluir que:

\begin{equation} 
\begin{aligned}
\label{RadialHomogeneo}
a_{d+1} &= -yg(x,y)\\
b_{d+1} &= xg(x,y),
\end{aligned}
\end{equation}

\noindent con $g(x,y)$ un polinomio homogéneo de grado $d$.\\ %Además, de la ecuación (\ref{EulerFieldCondition}) se sigue que:

%\begin{equation}
%C=(xa_{d}+yb_{d})+(xa_{d-1}+yb_{d-1})z\cdots+(xa_{1}+yb_{1})z^{d-1}+(xa_{0}+yb_{0})z^{d}.
%\end{equation}

Si miramos a la foliación definida por $\{ \Omega=0 \}$ en la carta $z=1$ obtenemos la 1-forma $\omega=A(x,y,1)\, dx+B(x,y,1)\, dy$, y si usamos las ecuaciones (\ref{CoeficientesRespectoAz}) y (\ref{RadialHomogeneo}) obtenemos que $\omega$ adquiere la forma:

\begin{equation}
\omega=(a_{0}+a_{1}\cdots+a_{d}-yg)\, dx+(b_{0}+b_{1}+\cdots+b_{d}+xg)\, dy.
\end{equation} 

\noindent Esta 1-forma genera la misma foliación que el campo vectorial:

\begin{equation}
\begin{aligned}
\dot{x} &= b_{0}+b_{1}+\cdots+b_{d}+xg\\
\dot{y} &= -a_{0}-a_{1}+\cdots-a_{d}+yg.
\end{aligned}
\end{equation}

Podemos resumir lo anterior en el siguiente lema:

\begin{Lema}
Sea $\mathcal{F}$ una foliación de grado $d$ en $\C\mathbb{P}^{2}$. Entonces en cualquier carta afín $\mathcal{F}$ está generada por un campo vectorial de la forma:
\begin{equation}
\begin{aligned}
\dot{x} &= b_{0}+b_{1}+\cdots+b_{d}+xg\\
\dot{y} &= a_{0}+a_{1}+\cdots+a_{d}+yg.
\end{aligned}
\end{equation}
Donde $a_{k},b_{k}\in\C[x,y]$ son polinomios homgéneos de grado $k$ y $g\in\C[x,y]$ es un polinomio homogéneo de grado $d$ o g(x,y) es el polinomio cero.
\end{Lema}

Es fácil convencerse, usando la expresión (\ref{EcuacionEnCP2Infinito}), que el caso no dicrítico se da cuando $g(x,y)\equiv 0$ y el caso dicrítico se da cuando $g(x,y)$ no se anula idénticamente.\\

Como mencionamos anteriormente, podemos pensar que toda foliación en una dos variedad analítica $M$ sólo tiene singularidades aisladas, y en caso de que $M$ sea compacta, sólo hay un número finito de ellas. Así, como $\C\mathbb{P}^{2}$ es compacto, cualquier foliación de él tiene un número finito de singularidades y esto nos permite que siempre podamos tomar una carta afín en la cual la recta al infinito no tenga singularidades. Vamos a usar esta observación en la prueba del siguiente teorema:

\begin{Teorema}
Sea $\mathcal{F}$ una foliación de grado $d$. Entonces, contando multiplicidades, $\mathcal{F}$ tiene $d^{2}+d+1$ puntos singulares.
\end{Teorema}

\begin{proof}
Si escogemos una carta afín en la cual la recta al infinito no tenga puntos singulares, en esta carta afín el campo vectorial que genera a la foliación adquiere la forma:
\begin{equation*}
\begin{aligned}
\dot{x} &= b_{0}+b_{1}+\cdots+b_{d}+xg\\
\dot{y} &= a_{0}+a_{1}+\cdots+a_{d}+yg.
\end{aligned}
\end{equation*}

Las dos curvas algebraicas que definen esta ecuación son de grado $d+1$ así que, por el Teorema de Bézout, estas dos curvas se intersecan, contando multiplicidades,  en $d^{2}+2d+1$ puntos singulares. Cualquiera de estos puntos de intersección que esté en nuestra carta afín va a ser un punto singular de la foliación, pero los puntos de intersección de estas dos curvas en la recta al infinito, gracias a la manera en que escogimos nuestras coordenadas, no son puntos singulares de la foliación. Por lo tanto, para probar el teorema basta demostrar que las dos curvas algebraicas que definen la ecuación se intersecan $d$ veces en la recta al infinito.\\

Para ver en que puntos la curva $\{A=a_{0}+\cdots+a_{d}+yg=0\}$ interseca a la recta al infintio tenemos que homogeneizar al polinomio $A$ y después evaluar en $z=0$. Si homogeneizamos al polinomio $A$ obtenemos el polinomio $a_{0}z^{d}+\cdots+a_{d}z+yg$, y al evaluar en $z=0$ vemos que esta curva corta a la recta al infinito en el punto $[1:0:0]$ y en los puntos determinados por las raíces del polinomio $g$. De manera análoga la curva $\{B=b_{0}+b_{1}+\cdots+b_{d}+xg\}$ corta a la recta al infinito en los puntos $[0:1:0]$ y en los puntos determinados por las raíces de $g$. Como $g$ es un polinomio de grado $d$, las dos curvas se interesecan $d$ veces en la recta al infinito.

\end{proof}
