En el otro caso, cuando $\alpha\in\{1,j,j^{2},\infty \}$ los 9 puntos singulares que no están en $\mathcal{P}$ degeneran en 3 puntos.
\\

Para $\alpha=1$:


$$(1,\alpha),(\alpha,1),(\frac{1}{\alpha},\frac{1}{\alpha}) \rightarrow (1,1).$$

$$(j,\alpha j^{2}),(\alpha j,j^{2}),(\frac{j}{\alpha},\frac{j^{2}}{\alpha}) \rightarrow (j,j^{2}).$$

$$(j^{2},\alpha j),(\alpha j^{2},j),(\frac{j^{2}}{\alpha},\frac{j}{\alpha}) \rightarrow (j^{2},j).$$


Para $\alpha=j$:


$$(1,\alpha),(\alpha j^{2},j),(\frac{j}{\alpha},\frac{j^{2}}{\alpha}) \rightarrow (1,j).$$

$$(\alpha,1),(j,\alpha j^{2}),(\frac{j^{2}}{\alpha},\frac{j}{\alpha}) \rightarrow (j,1).$$

$$(\alpha j,j^{2}),(j^{2},\alpha j),(\frac{1}{\alpha},\frac{1}{\alpha}) \rightarrow (j^{2},j^{}).$$


Para $\alpha=j^{2}$:


$$(1,\alpha),(\alpha j,j^{2}),(\frac{j^{2}}{\alpha},\frac{j}{\alpha}) \rightarrow (1,j^{2}).$$

$$(\alpha j^{2},j),(j,\alpha j^{2}),(\frac{1}{\alpha},\frac{1}{\alpha}) \rightarrow (j,j).$$

$$(\alpha ,1),(j^{2},\alpha j),(\frac{j}{\alpha},\frac{j^{2}}{\alpha}) \rightarrow (j^{2},j^{}).$$


Para $\alpha=\infty$:


$$(1,\alpha), (j,\alpha j^{2}),(j^{2}, \alpha j) \rightarrow [0:1:0].$$

$$(\alpha,1),(\alpha j^{2},j),(\alpha j, j^{2}) \rightarrow [1:0:0].$$

$$(\frac{1}{\alpha},\frac{1}{\alpha}),(\frac{j}{\alpha},\frac{j^{2}}{\alpha}),(\frac{j^{2}}{\alpha},\frac{j}{\alpha}) \rightarrow (0,0).$$
\\

Por lo tanto, si $\alpha\in\{1,j,j^{2},\infty \}$, $\Fol[4]{\alpha}$  tiene 12 puntos singulares, a saber, los 12 puntos de $\mathcal{P}$. Y el análogo a la proposición \ref{Prop:2} es la siguiente proposición:
\\

\begin{Proposicion}
\label{Prop:3}
Si $\alpha\in\{1,j,j^2,\infty\}$ las singularidades de $\Fol[4]{\alpha}$ son los 12 puntos de $\mathcal{P}$, 9 de ellos son de tipo radial y los otros 3 están contenidos en alguno de los conjuntos $\mathcal{P}_i$ de la proposición 1. Además, la foliación tiene una primera integral racional $H_\alpha=\tfrac{P_\alpha}{Q_\alpha}$ donde $P_\alpha$ y $Q_\alpha$ son producto de 3 líneas de $\mathcal{L}$. Estas líneas las podemos escoger de la siguiente manera, si $p_1,p_2,p_3\in\mathit{P}_i$ son las singularidades de $\Fol[4]{\alpha}$ que no son de tipo radial, sean $\mathit{l}_1, \mathit{l}_2, \mathit{l}_3,$ las rectas de $\mathcal{L}$ que pasan por alguno de estos 3 puntos y $\mathit{l}_4, \mathit{l}_5, \mathit{l}_6,$ rectas de $\mathcal{L}$ que pasan por algun otro de esos 3 puntos, entonces $H_\alpha=\tfrac{\mathit{l}_1\mathit{l}_2\mathit{l}_3}{\mathit{l}_4\mathit{l}_5\mathit{l}_6}$.
\end{Proposicion}

\begin{proof}
La única parte de la proposición que falta demostrar es que $\Fol[4]{\alpha}$ tiene una primera integral racional. Primero observemos que si $S$ es una transformación proyectiva como las del corolario \ref{Coro1Prop1}, por el lema \ref{Lema:Jalando}, $S^*(\Fol[4]{\alpha})=\Fol[4]{\beta}$ y además $\beta\in\{1,j,j^2,\infty\}$ pues en caso contrario $S^*(\Fol[4]{\alpha})$ tendría 21 puntos singulares mientras que $\Fol[4]{\alpha}$ tiene solamente 12 puntos singulares.
\\

%Aqui o antes se podría hablar mas de la observacion anterior
Por la observación anterior, basta encontrar una primera integral para $\Fol[4]{\infty}$ y después jalar esta integral a las demás foliaciones.
\\

Sea $H_\infty = \tfrac{y^3-1}{x^3-1} = \tfrac{(y-1)(y-j)(y-j^2)}{(x-1)(x-j)(x-j^2)} = \tfrac{P}{Q}$. Entonces, $\tfrac{dH}{H} = \tfrac{dP}{P} - \tfrac{dQ}{Q} = \tfrac{3y^2dy}{y^3-1} - \tfrac{3x^2dx}{x^3-1} = \tfrac{3}{(y^3-1)(x^3-1)}(x^3-1)y^2dy - (y^3-1)x^2dx$. Por lo tanto, $H_\infty$ es una primera integral de $\Fol[4]{\infty}$ y así, $H_\infty - 1$ y $\tfrac{1}{H_\infty} - 1$ también son primeras integrales de $\Fol[4]{\infty}$:

$$H_\infty = \frac{(y-1)(y-j)(y-j^2)}{(x-1)(x-j)(x-j^2)}$$
$$H_\infty - 1 = \frac{(y-x)(y-jx)(y-j^2x)}{(x-1)(x-j)(x-j^2)}$$
$$\frac{1}{H_\infty} - 1 = \frac{(y-x)(y-jx)(y-j^2x)}{(y-1)(y-j)(y-j^2)}$$

En la primera de las integrales los puntos que se escogieron son $[1:0:0]$ y $[0:1:0]$, en la segunda se escogieron $(0,0)$ y $[0:1:0]$ y en la última $(0,0)$  $[1:0:0]$.
%Al final (osea aqui) puedes explicar rapidamente como le harias en el caso de que alfa sea 1,j o j^2
\end{proof}

Con la proposición \ref{Prop:2} podemos probar que la familia de grado cuatro cumple el primer inciso del teorema principal %\ref{Teo:Prin}
.\\

En efecto, el conjunto $A^{4}$ que le quitaremos a $\overline\C$ para que la familia tenga singularidades no degeneradas de tipo analítico fijo es $A^{4} = \{1,j,j^{2},\infty\}$. La tabla \ref{Tab:PuntSingGrad4} muestra que las singularidades de $\Fol[4]{\alpha}$ se pueden escribir como funciones holomorfas $p_{j}\colon \overline\C\setminus A^{4}\rightarrow\CP$ y las primeras integrales que se obtuvieron en la proposición \ref{Prop:2} muestran que los puntos singulares $p_{j}(s),p_{j}(t)$ son localmente analíticamente equivalentes. 
